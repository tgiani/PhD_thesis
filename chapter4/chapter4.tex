\chapter{Theoretical error in PDFs determination}
\label{sec:th_error}
In order to get an accurate estimate of the uncertainties affecting the Standard Model predictions, theoretical 
uncertainties need to be taken into account. 
For hadron collider processes these are dominated by those due to missing higher order corrections in QCD calculations
(MHOU) and to PDFs. 
It is clear how MHOUs will also have an affect on the PDFs themselves, being present in the perturbative predictions
of the particular processes used for PDFs determination: besides from contributing to the overall
size of PDFs uncertainty, MHOU might affect the relative weights different points have in a fit:
points accurately described by the current perturbative predictions (up to NNLO) should weight more
than those poorly described. 
%
As described in Chapter~\ref{ch:nnpdf_methodology}, PDFs uncertainties only accounts for statistical and 
systematic errors affecting the experimental data entering the analysis, and typically do not include any source 
of theory uncertainty.

%
In this Chapter, based on Refs.~\cite{AbdulKhalek:2019bux,AbdulKhalek:2019ihb}, we describe how to set up 
a general formalism for the inclusion of theoretical uncertainties in PDFs determinations,
 and we specify it to the case of MHOUs.

 \section{Theory error as a covariance matrix}
 In this Section we will show how, by adopting a Bayesian approach and assigning to the 
 expected true value of the theory a Gaussian prior probability distribution, any missing theoretical
 contribution can be accounted for by adding a contribution to the experimental covariance matrix used in the PDFs fit.

%
Denoting as $D$ the vector of experimental data entering the analysis and as $\mathcal{T}$ the corresponding vector of 
"true" unknown values - whose determination is the goal of the experiment - we assume that the experimental results
are Gaussianly distributed about this hypothetical true values $\mathcal{T}$ 
\begin{align}
    \label{eq:conditional_prob}
    P\left(D|\mathcal{T}\right) \propto
    \exp\left(-\frac{1}{2}\left(D-\mathcal{T}\right)^T C^{-1} \left(D-\mathcal{T}\right)\right)\,.
\end{align}
The true values $\mathcal{T}$ are unknown, however we can compute the theory predictions $T$ 
for each experimental data using a theory framework which is 
generally incomplete, for example because it is based on the fixed-order truncation of a perturbative
expansion \footnote{In addition to MHOUs other effects which could be neglected in the predictions are 
higher twist and nuclear effects}. Furthermore $T$ depend on PDFs which are evolved up to 
the physical data scales using again an incomplete theory. The vectors $\mathcal{T}$ and $T$ would coincide
if the theory were exact and the PDFs were known with certainty.
%
Writing the difference between the true and the actual value of the theory predictions as
\begin{align}
    \Delta = \mathcal{T} - T\,
\end{align}
we can consider this difference as an additional unknown systematic error, accounting for the incomplete theory.
If we assume, in the same spirit as when estimating experimental systematic, that the true values $T$ are 
Gaussianly distributed about the theory predictions $T$
\begin{align}
    \label{eq:gaussian_hyp}
    P\left(\mathcal{T}|T\right) \propto 
    \exp\left(-\frac{1}{2}\left(\mathcal{T}-T\right)^T S^{-1} \left(\mathcal{T}-T\right)\right)\,,
\end{align}
then the prior probability distribution of $\Delta$ will be given by
\begin{align}
    P\left(\Delta\right) \propto \exp\left(-\frac{1}{2}\Delta^T S^{-1} \Delta\right)\,. 
\end{align}
Eq.~\ref{eq:conditional_prob} can be rewritten as
\begin{align}
    P\left(D|\mathcal{T}\right) = P\left(D, \Delta|T\right)  \propto 
    \exp\left(-\frac{1}{2}\left(D - T - \Delta\right)^T C \left(D-T-\Delta\right)\right)\,,   
\end{align}
so that, upon marginalization over $\Delta$, we get the conditional probability of the data $D$ 
given the theory predictions $T$
\begin{align}
    \label{eq:log_likelihood}
    P&\left(D|T\right) \propto 
    \int d\Delta\, P\left(\Delta\right)P\left(D, \Delta|T\right)  \nonumber\\
    &=\int d\Delta\, \exp\left(-\frac{1}{2}\left(D - T - \Delta\right)^T C^{-1} \left(D-T-\Delta\right) 
    -\frac{1}{2}\Delta^T S^{-1} \Delta \right) \nonumber \\
    &\propto \exp\left(-\frac{1}{2} \left(D-T\right)^T\left(C + S\right)^{-1} \left(D-T\right) \right)\,,
\end{align}
where in the last line we have performed explicitly the Gaussian integral over $\Delta$.
Eq.~\ref{eq:log_likelihood} defines the likelihood which is usually minimized in a Gaussian fit and shows how
theoretical uncertainties can be treated simply as another form of experimental systematic: 
it is an additional uncertainty to be taken into account when trying to find the truth from the data
using a specific theory setting, and it can be accounted for by mean of an additional contribution $S$ to
the experimental covariance matrix $C$.

%
The problem is then to estimate the theory covariance matrix $S$. The Gaussian hypothesis Eq.~\ref{eq:gaussian_hyp}
implies that 
\begin{align}
    \int d\mathcal{T} \, P\left(\mathcal{T}|T\right) \left(\mathcal{T}-T\right)_i \left(\mathcal{T}-T\right)_j = 
    \langle \Delta_i \,\Delta_j \rangle = S_{ij}\,,
\end{align}
showing how in general we need a way to estimate the shifts $\Delta_i$, in a way that takes into account
the theoretical correlations between different points within the same dataset, between different datasets
measuring the same physical process and between datasets corresponding to different processes
\footnote{Theory correlations will be present even for entirely different processes, through the universal parton distributions.}.

\section{MHOU from scale variations}
The most commonly used method to estimate the theory corrections due to MHOUs is scale variations. 
In the following we briefly revise its key ingredients and fix the conventions and terminology used in this work,
considering the simple case of electroproduction process, like DIS, 
and referring to Ref.~\cite{AbdulKhalek:2019ihb} for a complete and formal discussion of the topic.

%
Considering the problem of PDFs determination and remembering the factorized expression
for high-energy processes cross-sections, there are two independent source of MHOU: the 
perturbative expression of the partonic cross-section and the perturbative expression of the 
anomalous dimensions that determine the evolution of parton distributions.
These will be associated with two independent unphysical scales, which here will be denoted as renormalization scale
$\mu_r$ and factorization scale $\mu_f$.
Using RG equations for hard cross-sections and for PDFs it is possible to obtain an estimate of the MHOUs 
varying independently the two unphysical scales entering the problem.

Considering a generic structure function $F$, 
and denoting as $\overline{F}$ the corresponding scale-dependent theory prediction\footnote{the dependence of the structure 
function on $\mu_r^2$ and $\mu_f^2$ cancel order by order in 
perturbation theory, leaving the total physical observable independent on any unphysical scale.}
we have 
\begin{align}
\label{eq:scale_var_F}
    \overline{F}\left(Q^2,\mu_r^2, \mu_f^2\right) &= 
    \overline{C}\left(\alpha_s\left(\mu_r^2\right),\frac{\mu_r^2}{Q^2}\right)\otimes 
    q\left(\alpha_s\left(\mu_f^2\right),\frac{\mu_f^2}{Q^2}\right) \nonumber\\
    &=\overline{C}\left(\alpha_s\left(t+k_r\right),k_r\right)\otimes q\left(\alpha_s\left(t+k_f\right),k_f\right)\,,
\end{align}
where, following the notations Ref.~\cite{AbdulKhalek:2019ihb}, we have introduced the notations
$t = \log Q^2/\Lambda^2$, $k_r = \log \mu_r^2/Q^2$ and $k_f = \log \mu_f^2/Q^2$.
In order to estimate the MHOUs due to the truncation of the perturbative expansion of the coefficient function 
$C$ we can fix a specific renormalization scheme and keep $\mu_f = Q$, but varying
the renormalization scale $\mu_r^2$ used in the computation of the coefficient function itself.
The scale-dependent structure function $\overline{F}$ will then be given by
\begin{align}
    \overline{F}\left(Q^2,\mu_r^2\right) = 
    \overline{C}\left(\alpha_s\left(\mu_r^2\right),\frac{\mu_r^2}{Q^2}\right)\otimes q\left(Q^2\right) 
    &=\overline{C}\left(\alpha_s\left(t+k_r\right),k_r\right)\otimes q\left(t\right)\,.
\end{align}
Using the RG invariance of the physical cross section
\begin{align}
    \label{eq:RG_invariance}
    \mu_r^2\frac{d}{d\mu_r^2}\overline{F}\left(Q^2,\mu_r^2\right) = \frac{d}{dk_r}\overline{F}\left(t,k_r\right)=0\,,
\end{align}
it is easy to show that
the renormalization scale dependent Wilson coefficients $\overline{C}$ can be written as
\begin{align}
    \label{eq:wilson_coeff_scale_var}
    \overline{C}\left(\alpha_s\left(t+k_r\right),k_r\right) =
    C\left(t+k_r\right) &-  k\frac{d}{dt}C\left(t+k_r\right) 
    + \frac{1}{2} k_r^2\, \frac{d^2}{dt^2}C\left(t+k_r\right) + ...
\end{align}
where $C\left(t\right) \equiv \overline{C}\left(\alpha_s\left(t\right),0\right)$.
In other words, thanks to the RG invariance we can write the renormalization scale dependent Wilson coefficients
in terms of their values at the physical scale $Q$.
The log derivatives appearing in Eq.~\ref{eq:wilson_coeff_scale_var} can be easily evaluated using the 
perturbative expression of $C$ 
\begin{align}
    C\left(t\right) = 
    c_0 + \alpha_s\left(t\right)c_1 + \alpha_s^2\left(t\right)c_2 + \alpha_s^3\left(t\right)c_3 + ...\,,
\end{align}
and of the $\beta$ function expansion Eq.~\ref{eq:beta_function_expansion} getting
\begin{align}
    \label{eq:scale_varied_wilson_coeff}
    \overline{C}&\left(\alpha_s\left(t+k_r\right),k_r\right) = c_0 
    + \alpha_s\left(t+k_r\right)c_1 + \alpha_s^2\left(t+k_r\right)\left(c_2 + k_r\beta_0 c_1\right) \nonumber \\
    &+ \alpha_s^3\left(t+k_r\right)\left(c_3 + k_r\beta_0\left(\beta_1c_1 +2c_2 - k_r\beta_0 c_1\right)\right)\, + ...
\end{align} 

%
In the same way, starting again from Eq.~\ref{eq:scale_var_F}, we can fix $\mu_r=Q$
and vary the scale at which the PDFs are evaluated $\mu_f$ getting
\begin{align}
    \overline{F}\left(Q^2,\mu_f^2\right) = 
    \overline{C}\left(\alpha_s\left(Q^2\right)\right)\otimes q\left(\alpha_s\left(\mu_f^2\right),\frac{\mu_f^2}{Q^2}\right) 
    &=\overline{C}\left(t\right)\otimes q\left(\alpha_s\left(t+k_f\right),k_f\right)\,,
\end{align}
Using the RG invariance Eq.~\ref{eq:RG_invariance} with respect to $\mu_f$ we get 
\begin{align}
    \label{eq:scale_vare_pdf_1}
    \overline{q}\left(\alpha_s\left(t+k_f\right),k_f\right) = q\left(t+k_f\right) &-  k_f\frac{d}{dt}q\left(t+k_f\right) 
    + \frac{1}{2} k_f^2\,\frac{d^2}{dt^2}q\left(t+k_f\right) + ...
\end{align}
where in analogy with what done for the Wilson coefficients we have defined 
$q\left(t\right) \equiv \overline{q}\left(\alpha_s\left(t\right),0\right)$.
Using the evolution equation
\begin{align}
    \frac{d}{d\mu_f^2}\, q\left(\mu_f^2\right) = \gamma\left(\alpha_s\left(\mu_f^2\right)\right)q\left(\mu_f^2\right)\,,
\end{align}
Eq.~\ref{eq:scale_vare_pdf_1} can be rewritten as 
\begin{align}
    \label{eq:scale_var_pdf_2}
    \overline{q}\left(\alpha_s\left(t+k_f\right),k_f\right) = &q\left(t+k_f\right) - k\gamma q\left(t+k_f\right) \nonumber\\
    &+ \frac{1}{2}k^2\left(\gamma^2 + \frac{d}{dt}\gamma\right)q\left(t+k_f\right) + ...\,,
\end{align}
which can be further simplified using the perturbative expansion of the anomalous dimension
and the expression for the $\beta$ function\footnote{In Ref.~\cite{AbdulKhalek:2019ihb} it is explicitly shoe that an 
alternative way of obtaining Eq.~\ref{eq:scale_var_pdf_2} consists in varying the renormalization scale of the anomalous dimension.
MHOUs due to PDFs evolution can therefore be estimated varying either the PDFs scale or the scale of the anomalous dimension.}.

%
Eqs.~\ref{eq:scale_varied_wilson_coeff},\ref{eq:scale_var_pdf_2} allow to perform scale variations for a single process,
varying independently the two unphysical scales $\mu_r$ and $\mu_f$. 
