\chapter{Jets data}
\label{ch:jets}
In this Chapter we present a systematic analysis for the inclusion 
of jets cross-sections in a global parton distributions determination,
assessing the impact of jets and dijets production measurements on the PDFs at various perturbative orders.

\section{Jets observables and available experimental data}
Considering precision QCD studies, such as for example the determination of PDFs, 
there is a number of unsettle theoretical issues regarding the choice of the most suitable jets observable
to be used in the analysis. 
On one hand, the simplest inclusive observable one could consider, 
the single-inclusive jets cross-section~\cite{Ellis:1990ek,Aversa:1988fv}, turns out to be non-unitary,
with a possible unitary alternative offered by the dijets cross-section which, despite appearing to be 
especially well suited for PDFs determination~\cite{Giele:1994xd}, at NLO displays a significant scale dependence. 
Thanks to the recent NNLO computation for these observables, this last problem has essentially been settled,
with the scale dependence of dijets cross-section being under control at NNLO.
On the other hand, the single-jet inclusive cross section
shows a scale dependence which is not reduced when going to NNLO~\cite{Currie:2017ctp}, 
showing how the perturbative behaviour, the scale dependence~\cite{Currie:2018xkj} and even the definition~\cite{Cacciari:2019qjx} 
of this observable are non-trivial.  

%
In this study we address these issues in the context of PDFs determination, studying the impact of both single-inclusive
jets and dijets cross-sections with different scale choices in a global parton distributions fit. 
This allows us to asses which observable and scale choice leads to better PDFs compatibility with other data, better fit quality,
and more stringen constraint on the PDFs, and provides a guideline for the inclusion of jets observables in a future
global fit (NNPDF4.0).