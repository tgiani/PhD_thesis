\chapter{Theoretical error in PDFs determination}
\label{sec:th_error}
In order to get an accurate estimate of the uncertainties affecting the Standard Model predictions, theoretical 
uncertainties need to be taken into account. 
For hadron collider processes these are dominated by those due to missing higher order corrections in QCD calculations
(MHOU) and to PDFs. 
It is clear how MHOUs will also have an affect on the PDFs themselves, being present in the perturbative predictions
of the particular processes used for PDFs determination: besides from contributing to the overall
size of PDFs uncertainty, MHOU might affect the relative weights different points have in a fit:
points accurately described by the current perturbative predictions (up to NNLO) should weight more
than those poorly described. 
%
As described in Chapter~\ref{ch:nnpdf_methodology}, PDFs uncertainties only accounts for statistical and 
systematic errors affecting the experimental data entering the analysis, and typically do not include any source 
of theory uncertainty.

%
In this Chapter, based on Refs.~\cite{AbdulKhalek:2019bux,AbdulKhalek:2019ihb}, we describe how to set up 
a general formalism for the inclusion of theoretical uncertainties in PDFs determinations,
 and we specify it to the case of MHOUs.

 \section{Theory error as a covariance matrix}
 In this Section we will show how, by adopting a Bayesian approach and assigning to the 
 expected true value of the theory a Gaussian prior probability distribution, any missing theoretical
 contribution can be accounted for by adding a contribution to the experimental covariance matrix used in the PDFs fit.

%
Denoting as $D$ the vector of data entering the analysis and as $T$ the corresponding vector of theoretical 
predictions, which are functions of the free parameters $\theta$ of our model,
the figure of merit which is minimized during a Gaussian fit is defined as the probability of
the data D given the model parameters, namely the likelihood
\begin{align}
    P\left(D|\theta\right) = 
    \exp\left(-\frac{1}{2}\left(D-T\left(\theta\right)\right)^T C \left(D-T\left(\theta\right)\right)\right)\,.
\end{align}
As mentioned above, the predictions $T\left(\theta\right)$ are computed using a theory framework which is 
generally incomplete: