\chapter{Jets data}
\label{ch:jets}
In this Chapter, base on Ref.~\cite{AbdulKhalek:2020jut}, we present a systematic analysis for the inclusion 
of jets cross-sections in a global parton distributions determination,
assessing the impact of jets and dijets production measurements on the PDFs at various perturbative orders.

% 
There is a number of unsettle theoretical issues regarding the choice of the most suitable jets observable
to be considered for precision QCD studies, such as for example the determination of PDFs,
On one hand, the simplest inclusive observable, 
the single-inclusive jets cross-section~\cite{Ellis:1990ek,Aversa:1988fv}, turns out to be non-unitary.
A possible alternative free from this issue is offered by the dijets cross-section which however, despite appearing to be 
especially well suited for PDFs determination~\cite{Giele:1994xd}, at NLO displays a significant scale dependence. 
Thanks to the recent NNLO computation for these observables, this last problem has essentially been settled,
with the scale dependence of dijets cross-section being under control at NNLO.
On the other hand, the single-jet inclusive cross section
shows a scale dependence which is not reduced when going to NNLO~\cite{Currie:2017ctp}, 
showing how the perturbative behaviour, the scale dependence~\cite{Currie:2018xkj} and even the definition~\cite{Cacciari:2019qjx} 
of this observable are non-trivial.  

%
In this study we address these issues in the context of PDFs determination, studying the impact of both single-inclusive
jets and dijets cross-sections with different scale choices in a global parton distributions fit. 
This allows us to asses which observable and scale choice leads to better PDFs compatibility with other data, better fit quality,
and more stringen constraint on the parton distributions, and provides a guideline for the inclusion of jets observables in a future
global fit (NNPDF4.0).

\section{Jets data from ATLAS and CMS}
The ATLAS and CMS collaborations have performed a number of measurements of single-inclusive and 
dijets cross sections, with center of mass energies ranging from $\sqrt{s}=2.76$ to $13$ TeV.
In this work we will consider single-inclusive and dijets data at $\sqrt{s}=7$ and $8$ TeV.
%
Whereas recent global PDFs determinations include some jets data, like for instance NNPDF3.1, which includes ATLAS and CMS 
single-inclusive data with $\sqrt{s}=2.76$ and $7$ TeV, this is the first time that the full LHC-Run I jet dataset is being 
considered. In particular dijets data have not been included in any other previous analysis.

%
The specific features of the data considered here are summarized in Table~\ref{tab:input_datasets}:
for each dataset we reported the centre of mass energy $\sqrt{s}$, 
the integrated luminosity $\mathcal{L}$, the jet radius $R$,
the measured differential distribution and the number of datapoints $n_{\text{dat}}$.
The relevant kinematic variables are defined as follows.
For single-inclusive jets we denote as $p_T$ and $y$ the transverse momentum and rapidity.
For dijets, $m_{jj}$ is the invariant dijet mass, $y^*$ and $|y_{\text{max}}|$ are the absolute rapidity difference
and the maximum absolute rapidity of the two leading jets of the event, defined as $y^*=|y_1-y_2|/2$
and $|y_{\text{max}}|= \text{max}\left(|y_1|,|y_2|\right)$ respectively.
Finally, considering the dijets triple differential distribution,
$p_{T,\rm avg} = \left(p_{T_1}+p_{T_2}\right)/2$ is the average transverse momentum of the two leading jets and 
$y_b = |y_1+y_2|/2$ is the boost of the dijets system.

%
The ATLAS 7 TeV data for single-inclusive jet, given as distributions differential in transverse momentum 
$p_T$ and rapidity $y$, cover the kinematic range $100\,\, \text{GeV} \leq p_T \leq 1.992\,\, \text{TeV}$,
$0 \leq |y| \leq 3$, while the ATLAS 8 TeV data cover the same range in rapidity but with an extended transverse momentum
kinematic coverage $70\,\, \text{GeV} \leq p_T \leq 2.5\,\, \text{TeV}$.
In our default fit we include only the central rapidity bin ($y_{jet} \leq 0.5$) of ATLAS 7 TeV, 
for ease of comparison with the NNPDF3.1 analysis of Ref.~, where the same choice was adopted due to
the difficulty in achieving a good description of the complete set of rapidity bins using
the default experimental covariance matrix\footnote{In Refs.~\cite{Ball:2017nwa,Nocera:2017zge} this choice was validated,
showing how PDFs determined from each rapidity bin in turn are indistinguishable.}.
The CMS 7 TeV data are available for $100\,\, \text{GeV} \leq p_T \leq 2.0\,\, \text{TeV}$,
$0 \leq |y| \leq 2.5$, and the CMS 8 TeV cover the extended
ranges $74\,\, \text{GeV} \leq p_T \leq 2.5\,\, \text{TeV}$ and $0 \leq |y| \leq 3.0$.

%
Moving to dijets cross-sections, in the case of ATLAS 7 TeV the measurements are double-differential in
$m_{jj}$ and $|y^*|$, with
$260\,\,\text{GeV}\leq m_{jj} \leq 4.27\,\,\text{TeV}$ and $0 \leq y^* \leq 3.0$, while for CMS 7 TeV
the distributions are differential in  $m_{jj}$ and $|y_{\text{max}}|$, with
$200\,\,\text{GeV}\leq m_{jj} \leq 5\,\,\text{TeV}$ and $0 \leq |y_{\text{max}}| \leq 2.5$.  
Finally the CMS 8 TeV data are triple-differential in $p_{T,\rm avg}$, $y_b$ and $y^*$ with
ranges $133\,\,\text{GeV} \leq p_{T,\rm avg} \leq 1.78\,\,\text{TeV}$ and $0\leq y_b,\,y^* \leq 3$.
Note that ATLAS dijets measurements are currently available at 7 and 13 TeV bit not at 8 TeV.

%
In addition to the datasets listed in Table~\ref{tab:input_datasets} ATLAS and CMS have performed
measurements at $\sqrt{s}=13$ TeV for both single-inclusive jet~\cite{Aaboud:2017wsi,Khachatryan:2016wdh}
and dijets~\cite{Aaboud:2017wsi,Sirunyan:2020uoj}. These however have smaller integrated luminosities, for this reason
we do not include these datasets in the analysis. 
%
Finally several measurements for multijets production are also available, with ATLAS providing differential distributions for
three jets cross-sections at 7 TeV~\cite{Aad:2014rma} and four jets cross-sections at 8 TeV~\cite{Aad:2015nda} 
and CMS for three jets at 7 TeV~\cite{CMS:2014mna}.
However theoretical predictions for these observables are currently available only up to NLO, and therefore they will
not be considered here.

%
For all the measurements considered here, the complete set of systematic uncertainties and correlations available from
{\tt HepData} have been used.

%-------------------------------------------------------------------------------
\begin{table}[!t]
    \centering
    \scriptsize
    \renewcommand{\arraystretch}{1.90}
    %-------------------------------------------------------------------------------
\begin{tabularx}{\textwidth}{XXcccccc}
\toprule
  Experiment 
& Measurement   
& $\sqrt{s}$ [TeV]
& $\mathcal{L}$ [fb$^{-1}$] 
& $R$
& Distribution  
& $n_{\rm dat}$ 
& Reference   \\
\midrule
  ATLAS  
& Inclusive jets  
& 7 
& 4.5
& 0.6
& $d^2\sigma/dp_Td|y|$  
& 140  
& \cite{Aad:2014vwa}  \\
  CMS  
& Inclusive jets  
& 7
& 4.5
& 0.7
& $d^2\sigma/dp_Td|y|$  
& 133 
& \cite{Chatrchyan:2012bja}  \\
  ATLAS  
& Inclusive jets  
& 8
& 20.2
& 0.6
& $d^2\sigma/dp_Td|y|$  
& 171  
& \cite{Aaboud:2017dvo}  \\
  CMS  
& Inclusive jets  
& 8 
& 19.7
& 0.7
& $d^2\sigma/dp_Td|y|$  
& 185  
& \cite{Khachatryan:2016mlc}  \\
\midrule
  ATLAS  
& Dijets  
& 7 
& 4.5
& 0.6
& $d^2\sigma/dm_{jj}d|y^{*}|$  
& 90 
& \cite{Aad:2013tea}  \\
  CMS  
& Dijets  
& 7 
& 4.5
& 0.7
& $d^2\sigma/dm_{jj}d|y_{\rm max}|$  
& 54  
& \cite{Chatrchyan:2012bja}  \\
  CMS  
& Dijets  
& 8 
& 19.7
& 0.7
& $d^3\sigma/dp_{T,\rm avg}dy_b dy^{*}$  
& 122  
& \cite{Sirunyan:2017skj}  \\
\bottomrule
\end{tabularx}
%------------------------------------------------------------------------------ 

    \vspace{0.3cm}
    \caption{\small The LHC single-inclusive jet and dijet cross-section data
       that will be used  in this study. For each dataset we indicate the experiment,
       the measurement, the center of mass energy $\sqrt{s}$, the luminosity 
       $\mathcal{L}$, the jet radius $R$, the measured distribution, the number of 
       datapoints $n_{\rm dat}$ and the reference.}
    \label{tab:input_datasets}
\end{table}
%-------------------------------------------------------------------------------
    


\section{Theoretical calculations}
In this section we present the main features of the theoretical calculations used to perform
our phenomenological study, discussing scale choices and QCD corrections up to NNLO.
\comment{I would exclude discussion of EW corrections, maybe I could put in an appendix}

%
As mentioned at the beginning of this chapter, even when considering NNLO predictions
the single-inclusive jet cross-sections are in general
rather sensible to the choice of central scale.
Three possible choice are given by the individual jet transverse momentum $p_T$, 
the leading jet transverse momentum $p_{T,1}$ and the scalar sum of the transverse momenta of all the partons
in the event 
\begin{align}
    \hat{H}_T = \sum_{i\in\text{partons}} p_{T,i}\,.
\end{align}
Predictions obtained from different scales choices may differ,even at NNLO, by amounts which are comparable to their
scale dependence.
In Ref.~\cite{Currie:2018xkj} the scales $\mu = \hat{H}_T $ and $\mu = 2p_T$ were singled out as optimal choices,
according to a number of criteria, such as perturbative convergence and scale uncertainty as error estimate.
Here we will consider results for $\mu = \hat{H}_T$, which will be compared to those found using $\mu=p_T$,
which was the baseline choice adopted in previous NNPDF dterminations.

%
Turning to dijets observables, also here different scale choices are possible. As mentioned before, at NLO theoretical predictions
computed with different choices differ significantly, however the problem is alleviated at NNLO, with $\mu = m_{jj}$
emerging as preferred choice \comment{Refs}. This scale will be adopted here. 




\section{Results}