\chapter{Jets data}
\label{ch:jets}
In this Chapter, base on Ref.~\cite{AbdulKhalek:2020jut}, we present a systematic analysis for the inclusion 
of jets cross-sections in a global parton distributions determination,
assessing the impact of jets and dijets production measurements on the PDFs at various perturbative orders.

%
Considering precision QCD studies, such as for example the determination of PDFs, 
there is a number of unsettle theoretical issues regarding the choice of the most suitable jets observable
to be considered. 
On one hand, the simplest inclusive observable, 
the single-inclusive jets cross-section~\cite{Ellis:1990ek,Aversa:1988fv}, turns out to be non-unitary.
A possible alternative free from this issue is offered by the dijets cross-section which however, despite appearing to be 
especially well suited for PDFs determination~\cite{Giele:1994xd}, at NLO displays a significant scale dependence. 
Thanks to the recent NNLO computation for these observables, this last problem has essentially been settled,
with the scale dependence of dijets cross-section being under control at NNLO.
On the other hand, the single-jet inclusive cross section
shows a scale dependence which is not reduced when going to NNLO~\cite{Currie:2017ctp}, 
showing how the perturbative behaviour, the scale dependence~\cite{Currie:2018xkj} and even the definition~\cite{Cacciari:2019qjx} 
of this observable are non-trivial.  

%
In this study we address these issues in the context of PDFs determination, studying the impact of both single-inclusive
jets and dijets cross-sections with different scale choices in a global parton distributions fit. 
This allows us to asses which observable and scale choice leads to better PDFs compatibility with other data, better fit quality,
and more stringen constraint on the parton distributions, and provides a guideline for the inclusion of jets observables in a future
global fit (NNPDF4.0).

\section{Jets data from ATLAS and CMS}
The ATLAS and CMS collaborations have performed a number of measurements of single-inclusive and 
dijets cross sections, with center of mass energies ranging from $\sqrt{s}=2.76$ to $13$ TeV.
In this work we will consider single-inclusive and dijets data at $\sqrt{s}=7$ and $8$ TeV,
whose specific features are summarized in Table~\ref{tab:input_datasets}, where for each dataset we reported
the centre of mass energy $\sqrt{s}$, the integrated luminosity $\mathcal{L}$, the jet radius $R$,
the measured differential distribution and the number of datapoints $n_{\text{dat}}$.
The relevant kinematic variables are defined as follows.
For single-inclusive jets denote as $p_T$ and $y$ the transverse momentum and rapidity,
while for dijets $m_{jj}$ is the invariant dijet mass, $y^*$ and $|y_{\text{mass}}|$ are the absolute rapidity difference
and the maximum absolute rapidity of the two leading jets of the event, defined as $y^*=|y_1-y_2|/2$
and $|y_{\text{mass}}|= \text{max}\left(|y_1|,|y_2|\right)$ respectively.
Finally, considering the dijets triple differential distribution,
$p_{T,\rm avg} = \left(p_{T_1}+p_{T_2}\right)/2$ is the average transverse momentum of the two leading jets and 
$y_b = |y_1+y_2|/2$ is the boost of the dijets system.


%-------------------------------------------------------------------------------
\begin{table}[!t]
    \centering
    \scriptsize
    \renewcommand{\arraystretch}{1.90}
    %-------------------------------------------------------------------------------
\begin{tabularx}{\textwidth}{XXcccccc}
\toprule
  Experiment 
& Measurement   
& $\sqrt{s}$ [TeV]
& $\mathcal{L}$ [fb$^{-1}$] 
& $R$
& Distribution  
& $n_{\rm dat}$ 
& Reference   \\
\midrule
  ATLAS  
& Inclusive jets  
& 7 
& 4.5
& 0.6
& $d^2\sigma/dp_Td|y|$  
& 140  
& \cite{Aad:2014vwa}  \\
  CMS  
& Inclusive jets  
& 7
& 4.5
& 0.7
& $d^2\sigma/dp_Td|y|$  
& 133 
& \cite{Chatrchyan:2012bja}  \\
  ATLAS  
& Inclusive jets  
& 8
& 20.2
& 0.6
& $d^2\sigma/dp_Td|y|$  
& 171  
& \cite{Aaboud:2017dvo}  \\
  CMS  
& Inclusive jets  
& 8 
& 19.7
& 0.7
& $d^2\sigma/dp_Td|y|$  
& 185  
& \cite{Khachatryan:2016mlc}  \\
\midrule
  ATLAS  
& Dijets  
& 7 
& 4.5
& 0.6
& $d^2\sigma/dm_{jj}d|y^{*}|$  
& 90 
& \cite{Aad:2013tea}  \\
  CMS  
& Dijets  
& 7 
& 4.5
& 0.7
& $d^2\sigma/dm_{jj}d|y_{\rm max}|$  
& 54  
& \cite{Chatrchyan:2012bja}  \\
  CMS  
& Dijets  
& 8 
& 19.7
& 0.7
& $d^3\sigma/dp_{T,\rm avg}dy_b dy^{*}$  
& 122  
& \cite{Sirunyan:2017skj}  \\
\bottomrule
\end{tabularx}
%------------------------------------------------------------------------------ 

    \vspace{0.3cm}
    \caption{\small The LHC single-inclusive jet and dijet cross-section data
       that will be used  in this study. For each dataset we indicate the experiment,
       the measurement, the center of mass energy $\sqrt{s}$, the luminosity 
       $\mathcal{L}$, the jet radius $R$, the measured distribution, the number of 
       datapoints $n_{\rm dat}$ and the reference.}
    \label{tab:input_datasets}
\end{table}
%-------------------------------------------------------------------------------
    