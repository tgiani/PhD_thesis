\addcontentsline{toc}{chapter}{Introduction}
\chapter*{Introduction}
The increasing demand for accuracy required nowadays to perform high-energy phenomenology
represents one of the main challenges for the particle theory community. The
overall precision of theoretical computations has to match the one of the corresponding
experimental measurements: more precise  
experimental data call for more precise computations, together with a better control
and understanding of the different sources of errors affecting theoretical predictions. 

The computations of high-energy processes involving nucleons are based on factorization, namely on the separation
of amplitudes or cross-sections in different contributions, each of which depends on a specific energy scale.
While short-distance (or high-energy) contributions can be computed within the framework of perturbation theory,
those related to long-distance phenomena and responsible for the internal
structure of the nucleons are not directly accessible by means of first principle computations, 
and are factorized into universal objects of non perturbative origin, denoted as Parton Distribution Functions (PDFs).
 
PDFs encode our knowledge about the structure of the nucleons in terms of quarks and
gluons, and represent an essential ingredient to perform theory computations in collider physics.
Using factorization theorems, PDFs can be extracted from global QCD fits
over a set of experimental data, and, thanks to their universality, the results can be subsequently used
to compute different processes. 
Unfortunately, they also represent the dominant source of uncertainty in many important computations,
including analyses at the Large Hadron Collider (LHC) and other experiments,
the determination of standard model parameters, Higgs boson characterisation and searches for New Physics.
It is therefore necessary to push the determination of PDFs to a new level of accuracy,
researching for new methodologies and physical ideas to improve such global analyses.

The problem of PDFs determination involves a number of different lines of research which can be investigated 
using diverse approaches.
%
For phenomenological applications, it is important to develop frameworks which allow to implement and test
different numerical and analytical techniques to deal with complex global fits involving a big number of data coming
from experimental measurements. Such frameworks need to be flexible enough in order to fit the available data 
while imposing known physical constraints which have to be satisfied by the final PDFs,
and should allow to easily include new data and computations in the analysis. 
%
When new experimental data for specific high-energy processes become available,
their impact on parton distributions needs to be studied and quantified, to see how the new experimental 
information changes our knowledge regarding the structure of the nucleons.
This involves, on one side, the understanding of the details of the new data (distributions to be included in the analysis,
statistical and systematic errors, kinematic cuts to apply in order to use factorization theorems in their domains of validity), 
on the other the implementation of the most up to date theoretical computation for the process considered.  
%
In order to quantify how reliable the theory predictions are, specific studies regarding the theoretical errors
and their propagations into PDFs need to be carried out, and a suitable prescription for the combination
of experimental and theoretical uncertainties have to be formulated.
%
Alongside this kind of studies, which are usually performed within the high-energy community,
other approaches to study parton distributions are possible.
Being non-perturbative objects, PDFs are also a natural subject for a lattice QCD investigation. 
Starting from the formal definition of parton distributions, given in terms of matrix elements between 
nucleons states of QCD bilocal operators, it is possible to study the detailed structure of the ultraviolet (UV)
and infrared (IR) divergences of such objects, and relate them to specific correlators which, in principle, 
can be obtained through lattice QCD simulations. Such ideas have been studied and developed in recent years,
and data from lattice QCD simulations have started appearing. 
Given the high interest shown by both the lattice and high-energy community, new data for different lattice observables
are likely to appear in the coming years. 
It is therefore important to understand how to extract information about PDFs from them, 
study their potential for high-energy phenomenology,
their interplay with experimental data and understand the different sources of uncertainties 
affecting the corresponding lattice simulation.

This thesis can be divided into three main parts. The first part, composed by the first two chapters, 
is devoted to review the basics of QCD and the main concepts underlying factorization theorems. 
We will introduce parton distributions first from a phenomenological point of view, following the parton model ideas and
discussing the general structure of the NLO QCD corrections, and subsequently following a more formal approach,
revising the theoretical definition of PDFs in terms of QCD bilocal operators.
These chapters are based on a number of standard QCD text books~\cite{Ellis:1991qj,Muta:2010xua,Collins:1984xc},
quantum field theories lectures and classical references regarding factorization in high-energy processes,
such as refs.~\cite{Collins:1980ui, Collins:1981uw, Collins:1989gx}, and will be used to set up the 
main notations adopted in the rest of the thesis.
%
In the second part, made by chapters~\ref{ch:nnpdf_methodology},~\ref{ch:jets},~\ref{ch:th_error} and ~\ref{ch:bottom}, 
we will present a number of phenomenological studies, as detailed in the following.
%
In chapter~\ref{ch:nnpdf_methodology} we will describe the fitting methodology which has been developed
within the NNPDF collaboration and its recent implementation within the new {\tt n3fit} framework, 
with particular emphasis on the numerical techniques adopted to impose physical constraints on PDFs.
We will also discuss the so called fit basis, showing how
the final results produced within this framework only depend on the experimental and physical inputs, 
and not on the specific details of our fitting methodology. Such studies will be part of NNPDF4.0, the next public release
of the NNPDF collaboration.
%
In chapter~\ref{ch:jets} we will discuss the impact of jets data in a global PDFs determination. 
This gives an example of the kind of analyses which need to be done
every time an important class of new experimental measurements is available, together with the corresponding
theoretical predictions. 
The results discussed in this chapter have been first presented in ref.~\cite{AbdulKhalek:2020jut}.
In chapter~\ref{ch:th_error} we will discuss the definition and implementation of a theoretical error 
accounting for missing higher orders in a PDFs global analysis. 
The chapter is based on refs.~\cite{AbdulKhalek:2019bux,AbdulKhalek:2019ihb} where this
study was first presented by the NNPDF collaboration.
In chapter~\ref{ch:bottom}, based on ref.~\cite{Forte:2019hjc}, we discuss an alternative treatment for the bottom PDFs,
analysing the specific case of Higgs production in bottom fusion and its potential impact on precise phenomenology. 
%
Finally, in the third part of the thesis, made by chapters~\ref{ch:scalar_model},~\ref{ch:qpdfNNPDF} and~\ref{ch:ppdf_NNPDF}, 
we will present a number of studies to understand the relation between PDFs and lattice computable
Euclidean correlators.
In particular in chapter~\ref{ch:scalar_model}, based on ref.~\cite{DelDebbio:2020cbz},
we will revise the main theoretical ideas behind the lattice approach
using a nongauge theory as a simple toy example. This chapter is used to introduce the theoretical framework necessary
to understand the studies presented in the two following chapters.
In chapter~\ref{ch:qpdfNNPDF} we analyse data for quasi-PDFs matrix elements, studying their potential
in constraining PDFs and presenting a detailed analysis of the systematic uncertainties affecting the data and of how
these affect the final results. The results reported in this chapter were first presented in ref.~\cite{Cichy:2019ebf}.
In chapter~\ref{ch:ppdf_NNPDF} we present results for PDFs extracted from a different kind of lattice 
observables, denoted as reduced Ioffe-time pseudodistribution, assessing differences and similarities with
the analysis of chapter~\ref{ch:qpdfNNPDF}. The chapter is based on ref.~\cite{DelDebbio:2020rgv}.
Finally in chapter~\ref{ch:conclusions} we briefly summarize the main results and conclusions of this thesis. 

The thesis is supplemented with a number of appendices, where some analyses are further expanded and 
the details and results of the more technical computations are reported for reference. 




