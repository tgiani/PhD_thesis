\addcontentsline{toc}{chapter}{Introduction}
\chapter*{Introduction}
The increasing demand for precision required nowadays to perform high-energy phenomenology
represents one of the main challenge for the particle theory community. The
overall precision of a theoretical computation has to match the one of the corresponding
experimental measure: more precise  
experimental results call for more precise theoretical computations, together with a better control
and understanding of the different sources of errors affecting them. % and new data analysis techniques.

The computations of high-energy processes involving nucleons are base on factorization, namely on the separation
of amplitudes or cross-sections in different contributions, each of which depends on a specific energy scales.
While short-distance (or high-energy) contributions can be computed within the framework of perturbation theory,
contribution relates to long-distance phenomena and responsible for the internal
structure of the nucleons are factorized in universal objects of non perturbative origin,
denotes as Parton Distribution Functions (PDFs).
 
%PDFs are therefore a crucial ingredient in any analysis 
%involving initial states hadrons, but unfortunately they also represent the dominant source 
%of uncertainty in many analyses, such as the determination of standard model parameters, 
%Higgs boson characterisation and searches for New Physics.

PDFs encode our knowledge about the structure of the nucleons in terms of quarks and
gluons, and represent an essential ingredients to perform theory computations in collider physics.
Using factorization theorems PDFs can be extracted from global QCD fits
over a set of experimental data and thanks to their universality the results can be subsequently used
to perform computation for different processes. 
Unfortunately they also represent the dominant source of uncertainty in many important computations
including analyses at the Large Hadron Collider (LHC) and other experiments,
the determination of standard model parameters, Higgs boson characterisation and searches for New Physics.
It is therefore necessary to push the determination of PDFs to a new level of accuracy,
researching for new technologies and physical ideas to be used in such global analyses.

Detailed introduction plus overview of the thesis, chapter by chapter and saying which papers are involved.