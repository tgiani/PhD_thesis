\usepackage{
  pgf,
  tikz}
\usetikzlibrary{
  arrows,
  trees,
  scopes,
  decorations.text,
  decorations.pathreplacing,
  decorations.pathmorphing,
  decorations.markings,
  positioning,
  calc,
  patterns,
  intersections,
  shapes,
  fadings
}

\usetikzlibrary{trees}
\usetikzlibrary{decorations.pathmorphing}
\usetikzlibrary{decorations.markings}

\tikzset{
  photon/.style=
  {
    decorate,
    decoration={snake},
    draw=black
  },
  particle/.style=
  {
    very thick,
    draw=black,
    postaction={decorate},
    decoration=
    {
      markings,
      mark=at position .5 with {\arrow[draw=black]{>}}
    }
  },
  antiparticle/.style=
  {
    very thick,
    draw=black,
    postaction={decorate},
    decoration=
    {
      markings,
      mark=at position .5 with {\arrow[draw=black]{<}}
    }
  },
  gluon/.style=
  {
    decorate,
    draw=black,
    decoration=
    {coil,
      amplitude=4pt,
      segment length=5pt
    }
  },
  scalar/.style=
  {
    densely dashed,
    draw=black
  }
 }

% \tikzset{
%   ch_scalar/.style=
%   {
%     densely dashed,
%     thick,
%     draw=mLightBrown,
%     postaction={decorate},
%     decoration=
%     {
%       markings,
%       mark=at position .6 with {\arrow[thick,draw=black]{>}}
%     }
%   },
%   majorana/.style=
%   {
%     draw=mLightBrown
%   },
%   posit/.style=
%   {
%     rectangle,
%     outer sep=0,
%     inner sep=0,
%   },
%   dirac/.style=
%   {
%     draw=mLightBrown,
%     postaction={decorate},
%     decoration=
%     % {
%     %   markings,
%     %   mark=at position .6 with {\arrow[thick,draw=black]{>}}
%     % }
%   },
%   vector/.style=
%   {
%     draw=mLightBrown,
%     decorate,
%     decoration={snake,amplitude=1.5pt,segment length=3pt}
%   },
%   rev_dirac/.style=
%   {
%     draw=mLightBrown,
%     postaction={decorate},
%     decoration=
%     {
%       markings,
%       mark=at position .6 with {\arrow[thick,draw=black]{<}}
%     }
%   },
%   me_quark/.style=
%   {
%     draw=mLightBrown,
%     line width=0.6
%   },
%   ps_quark/.style=
%   {
%     draw=black
%   },
%   scalar/.style=
%   {
%     densely dashed,
%     thick,
%     draw=black,
%   },
%   gluon/.style=
%   {
%     decorate,
%     draw=black,
%     line width=0.6,
%     decoration={coil,amplitude=1.8,segment length=2}
%   },
%   me_gluon/.style=
%   {
%     decorate,
%     draw=mLightBrown,
%     line width=0.6,
%     decoration={coil,amplitude=1.8,segment length=2}
%   },
%   ps_gluon/.style=
%   {
%     decorate,
%     draw=black,
%     decoration={coil,amplitude=1.8,segment length=2}
%   },
%   mi_gluon/.style=
%   {
%     decorate,
%     draw=black,
%     decoration={coil,amplitude=1.8,segment length=2}
%   },
%   mi_quark/.style=
%   {
%     fill=black,
%   },
%   pdf_line/.style=
%   {
%     draw=\pdfcol,
%     double,
%   },
%   ps_blob/.style=
%   {
%     circle,
%     draw=black,
%     fill=black,
%     inner sep=0,
%     outer sep=0.5,
%     minimum size=1
%   },
%   pdf_blob/.style=
%   {
%     circle,
%     outer color=transparent!70,
%     inner color=transparent!20,
%     inner sep=1,
%     outer sep=0,
%     text=mLightBrown,
%   },
%   vertex/.style=
%   {
%     anchor=center,
%     circle,
%     draw=black,
%     fill=black,
%     inner sep=0,
%     outer sep=0.5,
%     minimum size=1
%   }
% }

% \makeatletter
% \def\parsecomma#1,#2\endparsecomma{\def\page@x{#1}\def\page@y{#2}}
% \tikzdeclarecoordinatesystem{page}{
%     \parsecomma#1\endparsecomma
%     \pgfpointanchor{current page}{north east}
%     % Save the upper right corner
%     \pgf@xc=\pgf@x%
%     \pgf@yc=\pgf@y%
%     % save the lower left corner
%     \pgfpointanchor{current page}{south west}
%     \pgf@xb=\pgf@x%
%     \pgf@yb=\pgf@y%
%     % Transform to the correct placement
%     \pgfmathparse{(\pgf@xc-\pgf@xb)/2.*\page@x+(\pgf@xc+\pgf@xb)/2.}
%     \expandafter\pgf@x\expandafter=\pgfmathresult pt
%     \pgfmathparse{(\pgf@yc-\pgf@yb)/2.*\page@y+(\pgf@yc+\pgf@yb)/2.}
%     \expandafter\pgf@y\expandafter=\pgfmathresult pt
% }
% \makeatother
% % For every picture that defines or uses external nodes, you'll have
% % to apply the 'remember picture' style. To avoid some typing, we'll
% % apply the style to all pictures.
% \tikzstyle{every picture}+=[remember picture, overlay]

% % By default all math in TikZ nodes are set in inline mode. Change this to
% % displaystyle so that we don't get small fractions.
% %\everymath{\displaystyle}

% %\tikzstyle{na} = [baseline=-.5ex]

% \definecolor{mLightBrown}{HTML}{EB811B}

%%% Local Variables: 
%%% mode: latex
%%% TeX-master: "bbh-intrinsic"
%%% End: 
