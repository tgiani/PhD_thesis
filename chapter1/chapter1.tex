\chapter{Basics of QCD}
In the early '60s it was generally believed that a theory for the strong interaction could not be formulated
within the framework of Quantum Field Theory (QFT) \cite{tHooft:1998qmr}.
Despite the remarkable success of QED in describing phenomena such as the anomalous magnetic moment of the electron,
the renormalization process was not completely understood yet and 
renormalizable quantum field theories were still looked at with suspicion.
%, and considered not suitable to describe the dynamics responsible for the composition of hadrons and hadronic reactions.  

%
The experimental observation of Bjorken scaling \cite{PhysRev.179.1547} in DIS experiment (SLAC 1960)
suggested that the constituents of nucleons may be described as almost-free and point-like objects 
when observed with high spatial resolution, leading to the formulation of the parton model \cite{PhysRevLett.23.1415}.
Accordingly, the dynamic of  partonic systems should have the property of becoming weaker at shorter distances. 
In 1973 asymptotic freedom of non-Abelian gauge field theories was discovered \cite{PhysRevLett.30.1346, PhysRevLett.30.1343},
making possible to embed the partonic model ideas within the framework of a renormalizable QFT.
%
Soon after it was shown how non-Abelian gauge field theories are actually the only ones exhibiting such property
in four dimensional space-time \cite{PhysRevLett.31.851} and Quantum Chromodynamics (QCD) emerged as a mathematically 
consistent theory for the strong interaction.  

%
After its formulation, QCD has been successfully used to describe strong interactions processes observed at colliders, and
nowadays it represents one of the cornerstone of the Standard Model. In this chapter we present
a brief overview of QCD, recalling some basic features of the theory and introducing our notation. 
For a more detailed treatment of the basics of QCD we refer to standard QFT and QCD textbook, such
as Refs.~\cite{Ellis:1991qj,Muta:2010xua,Collins:1984xc}.


%%%%%%%%%%%%%%%%%%%%%%%%%%%%%%%%%%%%%%%%%%%%%%%%%%%%%%%%%%%%%%%%%%%%%%%%%%%%%%%%%%%%%%%%%%%%%%%
\section{Lagrangian and its symmetries}
%
Quantum Chromodynamics is a non-Abelian gauge theory based on the gauge group SU(3)$_\text{color}$.
The classical Lagrangian of QCD, describing the interaction of $N_f$ massive spin-$\frac{1}{2}$ quarks
and massless spin-$1$ gluons, is given by
\begin{align}
    \label{eq:QCD_lagrangian}
    \mathcal{L}_{classical} = -\frac{1}{4}F^{A}_{\mu\nu}F^{A\mu\nu} + 
    \sum_{k=1}^{N_f}\,{\overline{\psi}}^{\,k}_a\left(i\gamma^{\mu}D_{\mu} + m_k\right)_{ab}\psi^k_b\,,
\end{align}
with the field strength tensor and the covariant derivative defined as
\begin{align}
    \label{eq:field_strength_thensor}
    &F^{A}_{\mu\nu} = \partial_{\mu} \mathcal{A}^A_{\nu} - \partial_{\nu} \mathcal{A}^A_{\mu} 
    + g \, f^{ABC} \mathcal{A}^B_{\mu}\mathcal{A}^C_{\nu}\,, \\
    \label{eq:covariant_derivateive}
    &D_{\mu} = \partial_{\mu} - i\,g\,T^A \mathcal{A}^{A}_{\mu}\,.
\end{align}
The summation over $k$ runs over all the quark flavors, with
each quark field $\psi^k_a$ belonging to the fundamental representation of the gauge group SU(3)$_\text{color}$ ($a=1,2,3$),
while the gauge field $\mathcal{A}^A_{\mu}$, called gluon, belong to the adjoint representation
($A=1,2,..,8$). The quantities $g$ and $f^{ABC}$ are the gauge coupling and the SU(3)$_\text{color}$ structure constants
respectively, and $T^A$ are the eight gauge group generators, satisfying
\begin{align}
    \label{eq:algebra}
    &\left[T^A,T^B\right] = i f^{ABC} T^C\,, \\
    \label{eq:normalization_SU3_generators}
    &\text{Tr}\left[T^A T^B\right] = T_R\, \delta^{AB}\,,
\end{align}
with the normalization of the generators conventionally chosen to be $T_R = 1/2$. 
An explicit expression for the generators $T^A$ in the fundamental representation is given by
$(T^A)_{ab} = 1/2\,\left(\lambda^A\right)_{ab}$ with $\lambda^A$ representing the eight 3-dimensional Gell-Mann matrices. 
Given the equations above, the colour matrices obey
\begin{align}
    \label{eq:SU3_generators_relations}
    &\sum_A T^A_{ab} T^A_{bc} = C_F\, \delta_{ac}\,,\\
    &\sum_{C,D} T^A_{CD} T^B_{DC} = \sum_{C,D} f^{ACD} f^{BCD}  C_A\, \delta^{CD}\,,
\end{align}
with $C_F= 4/3$ and $C_A= 3$.

%
The classical Lagrangian of Eq.~\ref{eq:QCD_lagrangian} does not allow to formulate quantum
perturbation theory in a consistent way: the problem cannot be avoided as long as
we rely on a gauge invariant lagrangian, where the gauge field $\mathcal{A}^A_{\mu}$
has the freedom to change according to gauge transformations.
We can get rid of such freedom by putting constraints on the field $\mathcal{A}^A_{\mu}$, 
known as \textit{gauge fixing conditions}, which in general can be expressed as
\begin{align}
    \label{eq:gauge_fixing}
    G^{\mu}\mathcal{A}_{\mu}^A = B^A\,,
\end{align}
with $G^{\mu}$ and $B^A$ chosen in some convenient way. 
Upon functional integration over the arbitrary quantity $B^A$, such condition is implemented 
in the theory by adding to the classical Lagrangian the so-called gauge fixing term
\begin{align}
    \label{eq:Gauge_fixing_lorents}
    \mathcal{L}_{gauge-fixing} = -\frac{1}{2\xi}\left(G^{\mu}\mathcal{A}_{\mu}^A\right)\,,
\end{align}
with $\xi$ representing an arbitrary parameters whose specific values defines the gauge.
Different choices for the gauge fixing term can be done. Taking $G^{\mu}=\partial^{\mu}$ we obtain
a class of covariant gauges. In this case the gauge fixing term must be supplemented by an additional term,
known as \textit{ghost Lagrangian} \cite{Faddeev:1967fc},
describing a complex scalar field $\eta^a$ (the Faddeev-Popov ghosts) obeying the Grassmann algebra and belonging to the adjoint
representation of the gauge group
\begin{align}
    \label{eq:ghosts_lagrangian}
    \mathcal{L}_{ghost} = \left(\partial_{\alpha}\eta^A\right)^* D^{\alpha}_{AB}\, \eta^B\, .
\end{align}
Perturbation theory can be formulated starting from the Lagrangian density
\begin{align}
    \label{eq:QCD_lagrangian_gauge_fixing}
    \mathcal{L} = \mathcal{L}_{classical} + \mathcal{L}_{gauge-fixing} + \mathcal{L}_{ghost}\, .
\end{align}
Another possible gauge fixing term is the one giving the so called axial gauges, fixed
in terms of a chosen vector $n$ such that $G^{\mu}=n^{\mu}$. In this case ghost fields decouples
and can thus be ignored, but the explicit form of the gluon propagator turns out to be more complicated than 
the one in the covariant gauges.

The QCD Lagrangian has a number of important symmetries, both exact and approximate,
which is worth recalling here.
The classical Lagrangian given in Eq.~\ref{eq:QCD_lagrangian} is invariant under SU(3)$_{\text{color}}$
gauge transformations. After gauge fixing gauge invariance is broken, but the resulting Lagrangian of 
Eq.~\ref{eq:QCD_lagrangian_gauge_fixing} satisfied the BRST symmetry \cite{Becchi:1975nq, Tyutin:1975qk},
which in turn guarantees the renormalizability of the theory.
Other exact symmetries are the so called flavour symmetries, 
which act through a global phase transformation of each quark field separately, giving
the baryon number conservation. 
%the conservation of the number of each of the different quark flavours. 
Other flavour symmetries include the discrete global symmetries of parity and time reversal invariance.
Finally charge-conjugation is also an exact symmetry of Eq.~\ref{eq:QCD_lagrangian_gauge_fixing}. 

Assuming mass degeneracy for the up and down quarks, the U(1) global symmetry associated with quark number 
can be extended to a global U(2) $=$ U(1) $\otimes$ SU(2). The new symmetry SU(2) is known as isospin symmetry.
We can further enhanced the symmetry group to U(1) $\otimes$ SU(3) assuming the strange quark to be also 
degenerate in mass with the up and the down\footnote{Such approximate flavour symmetry SU(3) is the basis of the Gell-Mann
quark model \cite{GellMann:1964nj} which was proposed well before the birth of QCD.}.
In the case of massless quarks, a chiral symmetry
U(2)$_{L}$ $\otimes$ U(2)$_R$ $=$ SU(2)$_V$ $\otimes$ U(1)$_V$ $\otimes$  SU(2)$_A$ $\otimes$ U(1)$_A$ 
holds, which however is spontaneously broken to SU(2)$_V$ $\otimes$ U(1)$_V$ $\otimes$ U(1)$_A$,
with the subscripts $V$ and $A$ denoting the vector and axial combinations.
The three pseudo-scalar Goldstone bosons resulting from chiral SU(2) breaking 
to SU(2)$_V$ are identified with the three pions $\pi^+$, $\pi^-$ and $\pi^0$
in the massless quark limit. 
While the survived SU(2)$_V$ $\otimes$ U(1)$_V$ symmetry is identified with the isospin and baryon number 
conservation mentioned previously, the remaining U(1)$_A$, despite not being spontaneously broken,
appears to be lost in QCD. The study of what happens to this symmetry is known as the U(1)-problem.
The axial symmetry U(1)$_A$ is actually broken at the quantum level, through the Adler-Bell-Jackiw anomaly,
which induces a new term in the QCD Lagrangian proportional to 
$\epsilon_{\alpha\beta\gamma\delta}\, \text{Tr}\, F^{\gamma\delta} F^{\alpha\beta}$ .
This term would be responsible for a violation of CP in the strong sector and 
its magnitude is given by the size of the parameter $\theta$, whose values is fixed to be in the range
$\theta<10^{-9}$ by experimental measures. The unexplained smallness of such parameter is known as the strong CP-problem.
Among the proposed solutions, the Peccei-Quinn mechanism was developed \cite{Peccei:1977ur}, proposing a dynamical explanation
for the $\theta$ values through the introduction of additional particles called axions. 

\comment{Add references}


%%%%%%%%%%%%%%%%%%%%%%%%%%%%%%%%%%%%%%%%%%%%%%%%%%%%%%%%%%%%%%%%%%%%%%%%%%%%%%%%%%%%%%%%%%%%%%%
\section{The running coupling and asymptotic freedom}
%
In analogy with the QED fine structure constant,
the QCD coupling constant is defined in terms of the gauge coupling as $\alpha_s = g^2/4\pi$.
As a consequence of the renormalization process, the coupling acquires a dependence on the renormalization scale $\mu$,
namely the arbitrary scale at which the subtraction of the UV poles is performed. 
The resulting renormalization group equations read
\begin{align}
    \label{eq:renormalization_group_coupling}
    Q^2\frac{\partial\alpha_s}{\partial Q^2} = \beta\left(\alpha_s\right)\,.
\end{align}
The $\beta$ function can be computed in perturbation theory as a power expansion in $\alpha_s$. 
Nowadays results up to five loops have been computed \cite{Herzog:2017ohr}.
At next-to-leading order it is given by
\begin{align}
    \label{eq:beta_function_expansion}
    \beta\left(\alpha_s\right) = -\beta_0\,\alpha_s^2\left(1+ \beta_1 \alpha_s + \mathcal{O}\left(\alpha_s^2\right)\right)\,,
\end{align}
with
\begin{align}
    \label{eq:beta_function_coefficients}
    \beta_0 = \frac{33 - 2 N_f}{12\pi}\,,\,\,\,\,\,
    \beta_1 = \frac{153 -19 N_f}{2\pi\left(33-2 N_f\right)}\,.
\end{align}
Using the leading order expression for the $\beta$ in Eq.~\ref{eq:renormalization_group_coupling} we find the leading-log
solution for the running coupling, relating its value at the generic scale $Q^2$ to the one at a reference scale
$\mu^2$ 
\begin{align}
    \label{eq:ll_coupling}
    \alpha_s\left(Q^2\right) = \frac{\alpha_s\left(\mu^2\right)}{1+\alpha_s\left(\mu^2\right)\beta_0\log\frac{Q^2}{\mu^2}}\,.
\end{align}
Given the positive sign of $\beta_0$, from Eq.~\ref{eq:ll_coupling} it is evident how, as the scale $Q^2$ becomes very large, 
the coupling $\alpha_s\left(Q^2\right)$ decreases to zero. This property, which characterizes non-Abelian gauge theories 
like QCD, is known as asymptotic freedom and the theory is then said to be asymptotically free.
%and it opens the way to compute high energy scattering processes as an expansion in the coupling.
It is customary to introduce a dimensionful parameter, usually denoted as $\Lambda_{\text{QCD}}$, representing the energy 
scale at which the perturbative series of the coupling would diverge.
\begin{align}
    \label{eq:lambda_QCD}
    \frac{1}{\alpha_s\left(\Lambda_{\text{QCD}^2}\right)} = 0\,.
\end{align}
Its specific value will depend on the choice of the renormalization scheme, on the order
of the $\beta$ function power expansion and on the number of active flavour entering the theory.
Qualitatively it represents the scale at which the interaction becomes strong and the perturbative description breaks down.

\comment{Add something about quarks mass and active flavour?}



%%%%%%%%%%%%%%%%%%%%%%%%%%%%%%%%%%%%%%%%%%%%%%%%%%%%%%%%%%%%%%%%%%%%%%%%%%%%%%%%%%%%%%%%%%%%%%%
\section{Perturbative and non-Perturbative approaches}
%The property of asymptotic freedom allows to compute high energy scattering processes as
%an expansion in the coupling, allowing a perturbative treatment of QCD.
The property of asymptotic freedom allows a perturbative treatment of QCD: 
perturbation theory can be seen as a systematic way of approximating the solution 
of a quantum field theory keeping the error under control.
Starting from the Lagrangian given in Eq.~\ref{eq:QCD_lagrangian_gauge_fixing},
it can be written as the sum between the free Lagrangian $\mathcal{L}_0$, describing free fermion and gauge fields, 
plus an interaction term $\mathcal{L}_I$, containing all the terms proportional to the gauge coupling $g$.
When considering high enough energy scales, all the terms proportional to the gauge coupling $g$
appearing in the lagrangian can be treated as perturbative interactions, so that the corresponding
contribution in the action can be expanded in a power series of the coupling.
   
%
Although its success in describing high energy processes, there are situations
in which perturbation theory cannot be applied: in the case of QCD, at low energy the coupling
becomes large, and a power expansion in the coupling is no longer possible. 
It is therefore important to have a non-perturbative formulation of QCD, based on the classical Lagrangian 
of Eq.~\ref{eq:QCD_lagrangian}.
The framework of lattice QFT represents one of the most studied and developed systematic approaches to study quantum
field theories in non-perturbative regimes.

%
In lattice QFT the path integral of the theory is defined on a discrete and finite Euclidean space-time,
characterized by a finite volume and lattice spacing $a$, and directly evaluated through Monte-Carlo simulations.
The full Lorents invariance of the continuum Minkowski space-time is reduced to the hypercubic group, 
but the lattice version of theory is built in such a way to preserve gauge invariance.

%
In order to write a discrete version of QCD, we need to construct an action for the gauge fields and for quark fields.
In the case of the gauge filed, the action can be written in terms of gauge links $U_{\mu}$, obtaining the so called Wilson action
\begin{align}
    \label{eq:Wilson_action}
    &S_G\left(U_{\mu}\right) = 
    \frac{\beta}{N}\sum_{x\in \Lambda^4}\sum_{\mu>\nu}\text{Re}\left(1-P_{\mu\nu}\left(x\right)\right)\,, \\
    &P_{\mu\nu}\left(x\right) = U_{\mu}\left(x\right)U_{\nu}\left(x+a\hat{\mu}\right)
    U_{\mu}\left(x+a\hat{\nu}\right)^{\dagger}U_{\nu}\left(x\right)^{\dagger}\,,
\end{align}
where $\beta$ is a constant of the theory. It can be shown that in the continuum limit Eq.~\ref{eq:Wilson_action}
recovers the Euclidean Yang-Mills action.
Considering the fermionic fields, it can be shown how a naive discretization of the Dirac action
\begin{align}
    \label{eq:naive_Dirac}
    S_{\text{Dirac}}\left[\psi,\bar{\psi}\right] = 
    a^4\sum_{x\in\Lambda^4}\bar{\psi}\left(x\right)\left(\bar{\slashed{\delta}} + m\right) \psi\left(x\right)
\end{align}
would lead to a theory
with the wrong continuum limit, describing 16 independent fermion states all having the same energy.
This is a consequence of a general property of fermionic actions known as the Nielsen-Ninomiya no-go theorem.
A possible solution was proposed by Wilson, through the addition of another term to the fermionic action
to remove the unwanted states
\begin{align}
    \label{eq:fermions_wilson_action}
    S_W\left[\psi,\bar{\psi}\right] = S_{\text{Dirac}}\left[\psi,\bar{\psi}\right] 
    - \frac{a^5}{2}\sum_{x\in\Lambda^4}\bar{\psi}\left(x\right)\tilde{\delta}^2 \psi\left(x\right)\,.
\end{align}
The addition of the Wilson term in the action solves the doubling problem, however it explicitly breaks
chiral symmetry, leading to a more complex UV structure of the theory.

%
The expectation value of a generic observable $O\left[\psi,\bar{\psi},A_{\mu}\right]$ can then be evaluated
through the path integral
\begin{align} 
    \label{eq:lattice_path_integral}
    \langle O\rangle = \frac{1}{Z}\,\int D\psi\, D\bar{\psi}\,DU_{\mu}\,
    O\left[\psi,\bar{\psi},U_{\mu}\right]e^{-S[\psi,\bar{\psi},U_{\mu}]}\,,
\end{align}
$S$ being the total action of the theory.
The fermion fields appear in the action as a bilinear term $\bar{\psi} M\left(U_{\mu}\right) \psi$,
so that the functional integrals over $\psi$ and $\bar{\psi}$ bar can be performed analytically getting
\begin{align}
    \label{eq:integrated_path_integral}
    \langle O\rangle = \frac{1}{Z}\int DU_{\mu}\,\text{det}M\left(U_{\mu}\right)
    O_{\text{Wick}}\left[U_{\mu}\right]e^{-S_G[U_{\mu}]}\,,
\end{align}
where $O_{\text{Wick}}$ denotes the functional obtained from $O$ performing the Wick contractions 
of the quark fields.
The quantity
\begin{align}
    \label{eq:prob_distribution}
    \frac{1}{Z}\,\text{det}M\left(U_{\mu}\right)e^{-S_G[U_{\mu}]}\,,
\end{align}
can be interpreted as a probability distribution\footnote{This is only true when $\text{det}M >0$. In a continuum theory it 
can be shown that this is true thanks to chiral symmetry, as long as we consider non-zero masses.
On the lattice the situation can be more complicated: in the case of Wilson fermions for example chiral symmetry is lost,
and the determinant is positive only when the bare masses are bigger than the critical mass.}, and an estimation
for $\langle O \rangle$ up to a $\mathcal{O}\left(1/\sqrt{N}\right)$ statistical error can be obtained by drawing
N samples $U_{\mu}^{(i)}$ from it and computing
\begin{align}
    \label{eq:average_O_lattice}
    \langle O \rangle = \frac{1}{N}\sum_{i=0}^N \, O\left[U_{\mu}^{(i)}\right]\,.
\end{align}    