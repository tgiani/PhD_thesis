\chapter{Basics of QCD}
In the early '60s it was generally believed that a theory for the strong interaction could not be formulated
within the framework of Quantum Field Theory (QFT) \cite{tHooft:1998qmr}.
Despite the remarkable success of Quantum Electrodynamics (QED) in describing phenomena such as the anomalous magnetic moment of the electron,
the renormalization process was not completely understood yet and 
renormalizable quantum field theories were still looked at with suspicion.
 
%
The experimental observation of Bjorken scaling \cite{PhysRev.179.1547} in Deep Inelastic Scattering (DIS) experiment (SLAC 1960)
suggested that the constituents of nucleons may be described as almost-free and point-like objects 
when observed with high spatial resolution, leading to the formulation of the parton model \cite{PhysRevLett.23.1415}.
Accordingly, the dynamic of  partonic systems should have the property of becoming weaker at shorter distances. 
In 1973 asymptotic freedom of non-Abelian gauge field theories was discovered \cite{PhysRevLett.30.1346, PhysRevLett.30.1343},
making possible to embed the partonic model ideas within the framework of a renormalizable QFT.
%
Soon after it was shown how non-Abelian gauge field theories are actually the only ones exhibiting such property
in four dimensional space-time \cite{PhysRevLett.31.851} and Quantum Chromodynamics (QCD) emerged as a mathematically 
consistent theory for the strong interaction.  

%
After its formulation, QCD has been successfully used to describe strong interaction processes observed at colliders, and
nowadays it represents one of the cornerstone of the Standard Model. In this chapter we present
a brief overview of QCD, recalling some basic features of the theory and introducing our notation. 
For a more detailed treatment of the basics of QCD we refer to standard QFT and QCD textbook, such
as refs.~\cite{Ellis:1991qj,Muta:2010xua,Collins:1984xc}.


%%%%%%%%%%%%%%%%%%%%%%%%%%%%%%%%%%%%%%%%%%%%%%%%%%%%%%%%%%%%%%%%%%%%%%%%%%%%%%%%%%%%%%%%%%%%%%%
\section{Lagrangian and its symmetries}
%
Quantum Chromodynamics is a non-Abelian gauge theory based on the gauge group SU(3)$_\text{color}$.
The classical Lagrangian of QCD, describing the interaction of $N_f$ massive spin-$\frac{1}{2}$ quarks
and massless spin-$1$ gluons, is given by
\begin{align}
    \label{eq:QCD_lagrangian}
    \mathcal{L}_{\mathrm{classical}} = -\frac{1}{4}F^{A}_{\mu\nu}F^{A\mu\nu} + 
    \sum_{k=1}^{N_f}\,{\overline{\psi}}^{\,k}_a\left(i\gamma^{\mu}D_{\mu} + m_k\right)_{ab}\psi^k_b\,,
\end{align}
with the field strength tensor and the covariant derivative defined as
\begin{align}
    \label{eq:field_strength_thensor}
    &F^{A}_{\mu\nu} = \partial_{\mu} \mathcal{A}^A_{\nu} - \partial_{\nu} \mathcal{A}^A_{\mu} 
    + g \, f^{ABC} \mathcal{A}^B_{\mu}\mathcal{A}^C_{\nu}\,, \\
    \label{eq:covariant_derivateive}
    &D_{\mu} = \partial_{\mu} - i\,g\,T^A \mathcal{A}^{A}_{\mu}\,.
\end{align}
The summation over $k$ runs over all the quark flavours, with
each quark field $\psi^k_a$ belonging to the fundamental representation of the gauge group SU(3)$_\text{color}$ ($a=1,2,3$),
while the gauge field $\mathcal{A}^A_{\mu}$, called gluon, belongs to the adjoint representation
($A=1,2,...,8$). The quantities $g$ and $f^{ABC}$ are the gauge coupling and the SU(3)$_\text{color}$ structure constants
respectively, and $T^A$ are the eight gauge group generators, satisfying
\begin{align}
    \label{eq:algebra}
    &\left[T^A,T^B\right] = i f^{ABC} T^C\,, \\
    \label{eq:normalization_SU3_generators}
    &\text{Tr}\left[T^A T^B\right] = T_R\, \delta^{AB}\,.
\end{align} 
An explicit expression for the generators $T^A$ in the fundamental representation is given by
$(T^A)_{ab} = 1/2\,\left(\lambda^A\right)_{ab}$ with $\lambda^A$ representing the eight 3-dimensional Gell-Mann matrices
and with the normalization of the generators conventionally chosen to be $T_R = 1/2$.
Given the equations above, the colour matrices obey
\begin{align}
    \label{eq:SU3_generators_relations}
    &\sum_A\sum_b T^A_{ab} T^A_{bc} = C_F\, \delta_{ac}\,,\\
    &\sum_{A}\sum_C T^A_{BC} T^A_{CD} = \sum_{A,C} f^{ACB} f^{ACD} =  C_A\, \delta_{BD}\,,
\end{align}
where, considering the generic case of $SU(N)$, $C_F = \frac{N^2-1}{2N}$ and $C_A = N$.

%
The classical Lagrangian of eq.~\eqref{eq:QCD_lagrangian} does not allow to formulate quantum
perturbation theory in a consistent way. The problem cannot be avoided as long as
we rely on a gauge invariant Lagrangian, where the gauge field $\mathcal{A}^A_{\mu}$
has the freedom to change according to gauge transformations.
We can get rid of such freedom by putting constraints on the field $\mathcal{A}^A_{\mu}$, 
known as \textit{gauge fixing conditions}, which in general can be expressed as
\begin{align}
    \label{eq:gauge_fixing}
    G^{\mu}\mathcal{A}_{\mu}^A = B^A\,,
\end{align}
with $G^{\mu}$ and $B^A$ chosen in some convenient way \footnote{Eq.~\eqref{eq:gauge_fixing} in general admits various solutions, representing different possible
gauge configurations known as Gribov copies. Their occurrence is known as the \textit{Gribov copies problem},
or \textit{Gribov ambiguity} \cite{Gribov:1977wm}.}. 
Upon functional integration over the arbitrary quantity $B^A$, such condition is implemented 
in the theory by adding to the classical Lagrangian the so-called gauge fixing term
\begin{align}
    \label{eq:Gauge_fixing_lorents}
    \mathcal{L}_{gauge-fixing} = -\frac{1}{2\xi}\left(G^{\mu}\mathcal{A}_{\mu}^A\right)\,,
\end{align}
with $\xi$ representing an arbitrary parameter whose specific value defines the gauge.
Different choices for the gauge fixing term can be done. Taking $G^{\mu}=\partial^{\mu}$ we obtain
a class of covariant gauges. In this case the gauge fixing term must be supplemented by an additional term,
known as \textit{ghost Lagrangian} \cite{Faddeev:1967fc},
describing a complex scalar field $\eta^a$ (the Faddeev-Popov ghosts) obeying the Grassmann algebra and belonging to the adjoint
representation of the gauge group
\begin{align}
    \label{eq:ghosts_lagrangian}
    \mathcal{L}_{ghost} = \left(\partial_{\alpha}\eta^A\right)^* D^{\alpha}_{AB}\, \eta^B\, .
\end{align}
Perturbation theory can be formulated starting from the Lagrangian density
\begin{align}
    \label{eq:QCD_lagrangian_gauge_fixing}
    \mathcal{L} = \mathcal{L}_{classical} + \mathcal{L}_{gauge-fixing} + \mathcal{L}_{ghost}\, .
\end{align}
Another possible gauge fixing term is the one giving the so-called axial gauges, fixed
in terms of a chosen vector $n$ such that $G^{\mu}=n^{\mu}$. In this case ghost fields decouples
and can thus be ignored, but the explicit form of the gluon propagator turns out to be more complicated than 
the one in the covariant gauges.


The QCD Lagrangian has a number of important symmetries, both exact and approximate,
which is worth recalling here.
The classical Lagrangian given in eq.~\eqref{eq:QCD_lagrangian} is invariant under SU(3)$_{\text{color}}$
gauge transformations. After gauge fixing gauge invariance is broken, but the resulting Lagrangian of 
eq.~\eqref{eq:QCD_lagrangian_gauge_fixing} satisfied the BRST symmetry \cite{Becchi:1975nq, Tyutin:1975qk},
which in turn helps with the renormalizability of the theory.
The so-called flavour symmetries are also exact symmetries of QCD, 
acting through a global phase transformation of each quark field separately and giving
the baryon number conservation. 
Other symmetries include the discrete global symmetries of parity and time reversal invariance.
Finally charge-conjugation is also an exact symmetry of eq.~\eqref{eq:QCD_lagrangian_gauge_fixing}. 

Assuming mass degeneracy for the up and down quarks, the U(1) global symmetry associated with quark number 
can be extended to a global U(2) $=$ U(1) $\otimes$ SU(2). The new symmetry SU(2) is known as isospin symmetry.
We can further enhance the symmetry group to U(1) $\otimes$ SU(3) assuming the strange quark to be also 
degenerate in mass with the up and the down\footnote{Such approximate flavour symmetry SU(3) is the basis of the Gell-Mann
quark model \cite{GellMann:1964nj} which was proposed well before the birth of QCD.}.
In the case of massless quarks, a chiral symmetry
U(2)$_{L}$ $\otimes$ U(2)$_R$ $=$ SU(2)$_V$ $\otimes$ U(1)$_V$ $\otimes$  SU(2)$_A$ $\otimes$ U(1)$_A$ 
holds, which however is spontaneously broken to SU(2)$_V$ $\otimes$ U(1)$_V$ $\otimes$ U(1)$_A$,
with the subscripts $V$ and $A$ denoting the vector and axial combinations.
The three pseudo-scalar Goldstone bosons resulting from chiral SU(2) breaking 
to SU(2)$_V$ are identified with the three pions $\pi^+$, $\pi^-$ and $\pi^0$
in the massless quark limit. 
While the survived SU(2)$_V$ $\otimes$ U(1)$_V$ symmetry is identified with the isospin and baryon number 
conservation mentioned previously, the remaining U(1)$_A$, despite not being spontaneously broken,
appears to be lost in QCD. The study of what happens to this symmetry is known as the U(1)-problem~\cite{Weinberg:1975ui}.
The axial symmetry U(1)$_A$ is actually broken at the quantum level, through the Adler-Bell-Jackiw anomaly,
which induces a new term in the QCD Lagrangian proportional to 
$\epsilon_{\alpha\beta\gamma\delta}\, \text{Tr}\, F^{\gamma\delta} F^{\alpha\beta}$ .
This term would be responsible for a violation of CP in the strong sector and 
its magnitude is given by the size of the parameter $\theta$, representing an angular variable
whose values is fixed to be in the range $\theta<10^{-9}$ by experimental measures.
The unexplained smallness of such parameter is known as the strong CP-problem.
Among the proposed solutions, the Peccei-Quinn mechanism was developed \cite{Peccei:1977ur} which,
together with the introduction of additional particles called axions \cite{PhysRevLett.40.223,PhysRevLett.40.279},
proposes a dynamical explanation for the $\theta$ values. 

%%%%%%%%%%%%%%%%%%%%%%%%%%%%%%%%%%%%%%%%%%%%%%%%%%%%%%%%%%%%%%%%%%%%%%%%%%%%%%%%%%%%%%%%%%%%%%%
\section{The running coupling and asymptotic freedom}
%
In analogy with the QED fine structure constant,
the QCD coupling constant is defined in terms of the gauge coupling as $\alpha_s = g^2/4\pi$.
As a consequence of the renormalization process, the coupling acquires a dependence on the renormalization scale $\mu$,
namely the arbitrary scale at which the subtraction of the ultraviolet (UV) poles is performed. 
The resulting renormalization group equations read
\begin{align}
    \label{eq:renormalization_group_coupling}
    \mu^2\frac{\partial\alpha_s}{\partial \mu^2} = \beta\left(\alpha_s\right)\,.
\end{align}
The $\beta$ function can be computed in perturbation theory as a power expansion in $\alpha_s$. 
Nowadays results up to five loops have been computed \cite{Herzog:2017ohr}.
At next-to-leading order it is given by
\begin{align}
    \label{eq:beta_function_expansion}
    \beta\left(\alpha_s\right) = -\beta_0\,\alpha_s^2\left(1+ \beta_1 \alpha_s + \mathcal{O}\left(\alpha_s^2\right)\right)\,,
\end{align}
with
\begin{align}
    \label{eq:beta_function_coefficients}
    \beta_0 = \frac{33 - 2 N_f}{12\pi}\,,\,\,\,\,\,
    \beta_1 = \frac{153 -19 N_f}{2\pi\left(33-2 N_f\right)}\,.
\end{align}
Using the leading order expression for the $\beta$ in eq.~\eqref{eq:renormalization_group_coupling} we find the leading-log
solution for the running coupling, relating its value at the generic scale $Q^2$ to the one at a reference scale
$\mu^2$ 
\begin{align}
    \label{eq:ll_coupling}
    \alpha_s\left(Q^2\right) = \frac{\alpha_s\left(\mu^2\right)}{1+\alpha_s\left(\mu^2\right)\beta_0\log\frac{Q^2}{\mu^2}}\,.
\end{align}
Given the positive sign of $\beta_0$, from eq.~\eqref{eq:ll_coupling} it is evident how, as the scale $Q^2$ becomes very large, 
the coupling $\alpha_s\left(Q^2\right)$ decreases to zero. This property, which characterizes non-Abelian gauge theories 
like QCD, is known as asymptotic freedom and the theory is then said to be asymptotically free.
It is customary to introduce a dimensionful parameter directly in the definition of $\alpha_s$,
usually denoted as $\Lambda$.
It can be defined as
\begin{align}
    \log\frac{\mu^2}{\Lambda^2} = -\int_{\alpha_s\left(\mu^2\right)}^{\infty} \frac{dx}{\beta\left(x\right)}\,,
\end{align}
with its specific value depending on the choice of the renormalization scheme, on the order
of the $\beta$ function power expansion and on the number of active flavours\footnote{The notion of active flavours will
be discussed in sec.~\ref{sec:quark_masses}} entering the theory.
Any dimensionful quantity in QCD can be expressed
in units of $\Lambda$.
It can be thought as an intrinsic scale of QCD:
at scales much larger than $\Lambda$ the coupling $\alpha_s$ is small and
quarks behave as almost free particles.

\section{Quark masses}
\label{sec:quark_masses}
The quark masses represent another parameter of the Lagrangian eq.~\eqref{eq:QCD_lagrangian}.
Just like the coupling constant because of the renormalization process they also acquire a
dependence on a renormalization scale $\mu$ given by 
\begin{align}
    \label{eq:renormalization_mass}
    \mu^2\frac{\partial m }{\partial \mu^2} = - \gamma_m\left(\alpha_s\right)m\left(\mu^2\right)\,.
\end{align}
The quantity $\gamma_m$ is the mass anomalous dimension which can be computed in perturbation theory as
a power expansion in $\alpha_s$
\begin{align}
    \gamma_m\left(\alpha_s\right) = c\,\alpha_s\left(1+c'\alpha_s + ...\right)\,,
\end{align}
with the explicit expression of the coefficients depending on the choice of the renormalization scheme.

%
In order to study the impact of the quarks mass on a generic physical observable $R$, consider
the situation in which we have only one quark with mass $m$.
Writing $R = R\left(Q^2/\mu^2, \alpha_s\left(\mu^2\right),m\left(\mu^2\right)/Q \right)$ 
and setting the renormalization scale equal to the physical scale $\mu=Q$,
if we assume the first $N$ derivatives of $R$ to be finite in $m=0$ 
we can write the expansion
\begin{align}
    \label{eq:R_mass_dependence}
    R&\left(1, \alpha_s\left(Q^2\right),m\left(Q^2\right)/Q \right) \sim
    R\left(1, \alpha_s\left(Q^2\right),0 \right)\nonumber \\
    & + \sum_{n=1}^{N}\frac{1}{n!}\left[\frac{m\left(Q^2\right)}{Q^2}\right]^n R^{(n)}\left(1,\alpha_s\left(Q^2\right),0\right)\,.
\end{align}
where $R^{(n)}$ denotes the $n$-th derivative of $R$ with respect to its third argument. 
Given the fact that eq.~\eqref{eq:renormalization_mass} leads to a change of the renormalized mass with $Q$ 
which is at most logarithmic, from eq.~\eqref{eq:R_mass_dependence} it is clear how, when considering 
high energy scales $Q \gg m$, the mass dependence can be dropped, and the quark can be considered massless.
%
On the other hand, when the mass of the quark is much greater than the relevant scale $Q$ it can be shown 
that the heavy quark mass correction to $R$ are suppressed by inverse powers of $m$, and therefore they can be ignored when
$Q \ll m$. 
%
The $n_l$ active flavour introduced in the previous section are the light quarks whose mass is much smaller than
the physical scale $Q$.

%
Considering the situation in which we have $n_l$ light quarks (i.e. quarks whose mass is much smaller than $Q$)
and a single heavy quark with mass $m$, the two values $\alpha_s^{(n_l+1)}$ and $\alpha_s^{(n_l)}$ 
measured in the two domains $Q \gg m$ and $Q \ll m$ respectively,
are usually matched through matching equation of the form
\begin{align}
    \alpha_s^{(n_l+1)}\left(Q^2\right) = 
    \alpha_s^{(n_l)}\left(Q^2\right) 
    + \sum_{k=2}^{\infty} c_k\left(L\right) \left(\alpha_s^{(n_l)}\left(m^2\right)\right)^k\,,
\end{align}
where the coefficients $c_k\left(L\right)$ are polynomials in $L=\log Q^2/m^2$ and at the scale $m^2=Q^2$ 
the $\mathcal{O}\left(\alpha_s^2\right)$ coefficient vanishes, $c_2\left(0\right)=0$.
Depending on the specific energy scale of interest,
one can perform the computation of a generic physical observable considering either $n_l$ or $n_l + 1$ active flavours,
each choice being more convenient in a given kinematic region. 
In sec.~\ref{sec:fonll} we will discuss a way in which such computations can be combined in a unique result which
is accurate at every energy scale.

%%%%%%%%%%%%%%%%%%%%%%%%%%%%%%%%%%%%%%%%%%%%%%%%%%%%%%%%%%%%%%%%%%%%%%%%%%%%%%%%%%%%%%%%%%%%%%%
\section{Perturbative and non-Perturbative approaches}
The property of asymptotic freedom allows to compute high energy scattering processes as
an expansion in the coupling, paving the way to a perturbative treatment of QCD.
The Lagrangian given in eq.~\eqref{eq:QCD_lagrangian_gauge_fixing}
can be written as the sum between the free Lagrangian $\mathcal{L}_0$, describing free fermions and gauge fields, 
plus an interaction term $\mathcal{L}_I$, containing all the terms proportional to the gauge coupling $g$:
at high energy $g$ becomes small, and all these terms 
can be treated as perturbative interactions, so that the corresponding
contribution in the action can be expanded in a power series of the coupling.

%
In general, perturbation theory can be seen as a systematic way of approximating the solution 
of a quantum field theory keeping the error under control.
Although its success in describing high energy processes, it is not a full solution of the theory, 
and there are situations in which it cannot be applied: in the case of QCD, at low energy the coupling
becomes large, and a power expansion in $\alpha_s$ is no longer possible. 
It is therefore important to have a non-perturbative formulation of QCD, based on the classical Lagrangian 
of eq.~\eqref{eq:QCD_lagrangian}.
The framework of lattice QFT represents one of the most studied and developed systematic approaches to study quantum
field theories in non-perturbative regimes. In the following we recall the basic ideas underlying the formulation of 
QCD on an Euclidean lattice, referring to standard textbooks as ref.~\cite{smit_2002} for a complete discussion.

%
In lattice QFT the path integral of the theory is defined on a discrete and finite Euclidean space-time,
characterized by a finite volume and lattice spacing $a$, and directly evaluated through Monte-Carlo simulations.
The lattice can be defined as a cartesian product
\begin{align}
    \label{eq:lattice_def}
    \Lambda^4\left(N\right) = a\Big([\![0, N_0-1 ]\!]\times[\![0, N_1-1 ]\!]\times[\![0, N_2-1 ]\!]
    \times[\![0, N_3-1 ]\!] \Big)\,,
\end{align}
where $N$ is a four vector with integer components and $[\![0, n ]\!]$ is the set of all the integers $j$ such that
$0 \le j \le n$. The integer $N_{\mu}$ represents the number of lattice sites in the $\mu$ direction, corresponding
to a space-time extent equal to $a N_{\mu}$. 
From this definition it is clear how for each point $x_{\mu}$ belonging to the lattice there exists a four vector $j_{\mu}$ 
with integer coordinates such that $x_{\mu} = a j_{\mu}$. 
The zero-component $N_0$ is usually identified with the temporal extent
$T = a N_0$, while the three remaining ones, assumed to be equal, represent the spatial extent in the three spatial directions
$L = a N_1 = a N_2 = a N_3$. On such lattice the full Lorentz invariance of the continuum 
Minkowski space-time is reduced to the hypercubic group, 
however when considering gauge theories their lattice version is built in such a way to preserve gauge invariance. 

Considering for simplicity a theory containing a single scalar field $\phi$,
taking a generic correlation function in Minkowski space 
\begin{align}
    C_n^{(M)}\left(x_1,..., x_n\right) = 
    \frac{1}{Z} \int \mathcal{D}\phi\, \phi\left(x_1\right)...\phi\left(x_n\right)\exp\left(i S^{(M)}\left[\phi\right]\right)\,,
\end{align}
one can define the associated Euclidean correlation function
by performing a Wick rotation $x_i = \left(x^0_i, \vec{x}_i\right) \rightarrow \bar{x}_i = \left(-i x^0_i, \vec{x}_i\right)$
\begin{align}
    \label{eq:euclidean_correlator}
    C_n^{(E)}\left(x_1,..., x_n\right) &\equiv C_n^{(M)}\left(\bar{x}_1,..., \bar{x}_n\right) = \nonumber \\
    &\frac{1}{Z} \int \mathcal{D}\phi\, \phi\left(x_1\right)...\phi\left(x_n\right)\exp\left(- S^{(E)}\left[\phi\right]\right)\,,
\end{align}
where $S^{(E)}$ is the Euclidean action of the theory.
Such Wick rotation effectively transform the Minkowski metric $\eta_{\mu\nu} = \text{diag}\left(1,-1,-1,-1\right)$
into the positive Euclidean metric $\delta_{\mu\nu} = \text{diag}\left(1,1,1,1\right)$
\footnote{The precise conditions under which the Wick rotation works are stated by the 
Osterwalder-Schrader axioms \cite{cmp/1103858969}. These are a list of properties
that correlation functions in Euclidean space have to satisfy to be the analytic continuation of
the correlation functions of the original QFT in Minkowski space.}.

The functional integral of eq.~\eqref{eq:euclidean_correlator} can be estimated
by averaging the fields product $\phi\left(x_1\right)...\phi\left(x_n\right)$ on the probability density 
$D\phi \, \text{exp}\left(-S^{(E)}\left[\phi\right]\right)$.


%
In order to write a discrete version of QCD, we need to construct an action for the gauge field and one for the quark fields.
In the case of the gauge field, the action can be written in terms of gauge links $U_{\mu}\left(x\right)$, obtaining the so-called Wilson action
\begin{align}
    \label{eq:Wilson_action}
    &S_G\left(U_{\mu}\right) = 
    \frac{\beta}{N}\sum_{x\in \Lambda^4}\sum_{\mu>\nu}\text{Re}\left(1-P_{\mu\nu}\left(x\right)\right)\,, \\
    &P_{\mu\nu}\left(x\right) = U_{\mu}\left(x\right)U_{\nu}\left(x+a\hat{\mu}\right)
    U_{\mu}\left(x+a\hat{\nu}\right)^{\dagger}U_{\nu}\left(x\right)^{\dagger}\,,
\end{align}
where $\beta$ is a constant of the theory. It can be shown that in the continuum limit,
using $\beta=2N/g^2$, eq.~\eqref{eq:Wilson_action}
recovers the Euclidean Yang-Mills action.

%
Considering the fermionic fields, defining the translation operator in the $\hat{\mu}$ direction as 
$\tau_{\mu}f\left(x\right) = f\left(x+a\hat{\mu}\right)$, 
it turns out that a naive discretization of the Dirac action
\begin{align}
    \label{eq:naive_Dirac}
    S_{\text{Dirac}}\left[\psi,\bar{\psi}\right] = 
    a^4\sum_{x\in\Lambda^4}\bar{\psi}\left(x\right)\left(\frac{\gamma^{\mu}}{2a}\left(\tau_{\mu}-\tau_{-\mu}\right) + m\right) \psi\left(x\right)
\end{align}
would lead to a theory with the wrong continuum limit, describing 16 independent fermion states all having the same energy.
This is the so-called \textit{doubling problem}, and it is a consequence of a more general property of 
fermionic actions known as the Nielsen-Ninomiya no-go theorem~\cite{Nielsen:1981hk}.
A possible solution to the doubling problem was proposed by Wilson~\cite{PhysRevD.10.2445}, 
through the addition of another term to the fermionic action
to remove the unwanted states
\begin{align}
    \label{eq:fermions_wilson_action}
    S_W\left[\psi,\bar{\psi}\right] = S_{\text{Dirac}}\left[\psi,\bar{\psi}\right] 
    - \frac{a^5}{2}\sum_{x\in\Lambda^4}\bar{\psi}\left(x\right)\tilde{\delta}^2 \psi\left(x\right)\,,
\end{align}
where $\tilde{\delta}^2 = a^{-2}\sum_{\mu}\left(1-\tau_{-\mu}\right)\left(\tau_{\mu}-1\right)$.
In the presence of a gauge field, eq.~\eqref{eq:fermions_wilson_action} becomes 
\begin{align}
    S_W\left[\psi,\bar{\psi}, U_{\mu}\right] = a^4\sum_{x\in\Lambda^4}\bar{\psi}\left(x\right)\left(K\left[U_{\mu}\right] + m\right)\psi\left(x\right)\,,
\end{align}
where $K\left[U_{\mu}\right]$ is a suitable discretization for the covariant derivative which implements
the Wilson prescription. 
The addition of the Wilson term in the action solves the doubling problem, however it explicitly breaks
chiral symmetry, leading to a more complex UV structure of the theory. 
An example of this appears when looking at the renormalization of the fermion mass:
when a Wilson fermion is coupled to some gauge interaction the mass shift, which would be zero in the continuum and in the case 
of naive lattice fermions, is not zero anymore. This means that a Wilson fermion with zero bare mass is actually not massless.
Its mass is denoted as \textit{critical mass}.


%
Once the Euclidean action has been defined, the expectation value of a generic observable 
$O\left[\psi,\bar{\psi},U_{\mu}\right]$ can be evaluated
through the path integral
\begin{align} 
    \label{eq:lattice_path_integral}
    \langle O\rangle = \frac{1}{Z}\,\int D\psi\, D\bar{\psi}\,DU_{\mu}\,
    O\left[\psi,\bar{\psi},U_{\mu}\right]e^{-S[\psi,\bar{\psi},U_{\mu}]}\,.
\end{align}
with 
\begin{align}
    S[\psi,\bar{\psi},U_{\mu}] = S_G\left(U_{\mu}\right) + S_W\left[\psi,\bar{\psi}, U_{\mu}\right]\,.
\end{align}
Since the fermionic action is quadratic,
the functional integrals over $\psi$ and $\bar{\psi}$ can be performed analytically getting
\begin{align}
    \label{eq:integrated_path_integral}
    \langle O\rangle = \frac{1}{Z}\int DU_{\mu}\,\text{det}\left(K+M\right)
    O_{\text{Wick}}\left[U_{\mu}\right]e^{-S_G[U_{\mu}]}\,,
\end{align}
where $O_{\text{Wick}}$ denotes the functional obtained from $O$ performing the Wick contractions 
of the quark fields. Note that this observable depends on the quark propagator $\left(K+M\right)^{-1}$,
appearing for every $\psi\bar{\psi}$ contraction. Inverting the term $K+M$ in order to obtain the fermionic propagator
represents the elementary computation in lattice QCD simulations.
Assuming $\text{det}\left(K+M\right)>0$ \footnote{In a continuum theory it 
can be shown that this is true thanks to chiral symmetry, as long as we consider non-zero masses.
On the lattice the situation can be more complicated: in the case of Wilson fermions for example chiral symmetry is lost,
and the determinant is positive only when the bare masses are bigger than the critical mass.}, the quantity
\begin{align}
    \label{eq:prob_distribution}
    \frac{1}{Z}\,\text{det}\left(K+M\right)e^{-S_G[U_{\mu}]}\,,
\end{align}
can be interpreted as a probability distribution and an estimation
for $\langle O \rangle$ up to a $\mathcal{O}\left(1/\sqrt{N}\right)$ statistical error can be obtained by drawing
N samples $U_{\mu}^{(i)}$ from it and computing
\begin{align}
    \label{eq:average_O_lattice}
    \langle O \rangle = \frac{1}{N}\sum_{i=0}^N \, O\left[U_{\mu}^{(i)}\right]\,.
\end{align} 
