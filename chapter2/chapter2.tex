\chapter{Factorization theorems in QCD}
In the previous chapter we have seen how, thanks to asymptotic freedom, the structure of QCD simplifies when considering problems
involving short-distance or high energy scales: the coupling becomes small and the theory can be solved perturbatively.
However, cross sections for high energy processes are usually a combination of short- and long-distance effects, and cannot
be fully computed within the framework of perturbation theory. Factorization theorems allows to separate (factorize) a cross section
or amplitude in separate contributions, each factor containing the dependence of the process on a specific distance scale:
while short-distance effects can be computed in perturbation theory, contributions coming from long-distance phenomena have
to be extracted from experimental data. 
In this chapter, starting from the Parton Model ideas we will discuss factorization theorems in QCD,
using the case of Deep Inelastic Scattering as a basic example. We will then introduce the main subject of the
present work, namely Parton Distribution Functions (PDFs).

%%%%%%%%%%%%%%%%%%%%%%%%%%%%%%%%%%%%%%%%%%%%%%%
\section{The Parton Model}
The basic ideas underlying factorization theorems for high energy processes can 
be described appealing to Feynman's parton model \cite{PhysRevLett.23.1415,Feynman:1973xc, Bjorken:1969ja}.
In this picture fundamental particles called \textit{partons} are bounded together to form hadrons.
Since the details of the partonic system are unknown, the scattering between a test particle and the hadron 
as a whole cannot be computed. However we assume that we do know how to describe the scattering with a free parton.
%
As example, let us consider the case of the scattering of a high-energy charged lepton off a
hadron target 
$$e^{-}(k)\,H\,(P)\,\rightarrow e^{-}(k')\,X \,. $$ 
Such process is known as Deep Inelastic lepton-hadron Scattering (DIS).
Looking at this scattering in the centre of mass frame, the hadron will be Lorentz contracted 
in the direction of the collision and the lifetime of the internal partonic states will be lengthened.
As a consequence, the time the electron takes to cross the hadron will be much shorter than the average lifetime of
each partonic states. During the time of the interaction the hadron can then be thought as "frozen" in 
a well defined partonic states, with each parton carrying a definite fraction $\xi$ of the hadron momentum
and not interacting with the other ones. 
If the energy of the collision is high enough, the virtual photon mediating the electron-hadron 
interaction will interact with a single parton having a given momentum fraction. Likewise, interactions
occurring in the final states are assumed to occur on time scales too long to interfere with the hard scattering.
%
Given this picture, it is natural to think about the scattering cross section as classical and incoherent,
namely as a sum of probabilities rather than of amplitudes. The parton model ideas can be summarized in the simple formula
\begin{align}
    \label{eq:parton_model}
    \sigma&\left(e^{-}\left(k\right)H\left(P\right)\rightarrow e^{-}\left(k'\right)+X\right) = \nonumber\\
     &\,\,\,\,\,\,\,\,\sum_i\int_0^1 d\xi\, q_{i/H}\left(\xi\right)
     \hat{\sigma}\left(e^{-}\left(k\right)q_i\left(\xi P\right)\rightarrow e^{-}\left(k'\right)+q_f\right)
\end{align}
where $\hat{\sigma}$ represents the partonic cross section describing the interaction between 
a single free parton and the virtual photon, and the set of functions $q_{i/H}\left(\xi\right)$ the probability densities
of having a parton of kind $i$ inside the hadron $H$, carrying a fraction $\xi \in \left(0,1\right)$ of the total hadron momentum. . 

%
In the following we recall in more details how parton model ideas apply to the case of 
DIS, setting the stage for a more complete 
discussion of factorization in the next section, where the inclusion of QCD corrections will be discussed.  
DIS experiments are traditionally the main testing ground of perturbative QCD,
having been the first processes were pointlike particles were seen inside the hadron, 
thus motivating the formulation of the parton model. 
They play a central role in any discussion regarding factorization and provide a simple experimental and theoretical framework to
study the strong interaction.
As we are going to recall in the following, measurements of DIS structure functions directly probe the structure of hadrons, 
giving the bulk of experimental measurements at the basis of every phenomenological determination of Parton Distribution Functions.

\begin{figure}[h!]
    \center
    \includegraphics[scale=0.7]{dis.png}
    \caption{DIS kinematic in the parton model.}
    \label{fig:DIS_kin}
\end{figure}

The kinematic for DIS is reported in Fig.~\ref{fig:DIS_kin}.
The space-like momentum of the photon (if the initial lepton is an electron or a muon then the scattering is 
mediated by the exchange of a photon) is given by $q = k - k'$, the centre of mass energy square is $s=\left(P+k\right)^2$ and
we denote the invariant mass square of the finale states as $W = \left(P+q\right)^2$.
It is customary to introduce the kinematic variables
\begin{align}
	\label{eq:DIS_kinematic}
    & Q^2 = -q^2 > 0\,, \,\,\,\,\,\, x = \frac{Q^2}{2P\cdot q}\,, \,\,\,\,\,\, y = \frac{Q^2}{x s}\,.
    %y = \frac{P\cdot q}{P\cdot k }\,.
\end{align}
In the regime 
$$ Q^2,\, W^2 >> m^2_{\text{hadron}} \sim \Lambda^2_{\text{QCD}}\,, $$ 
leptons and quarks masses can be neglected.
It is easy to see that the variable $x$, known as \textit{Bjorken variable} can take values between 0 and 1, 
with $x\rightarrow 1$ representing the elastic limit in which $W=m^2_{\text{hadron}}$. The Deep Inelastic regime
is then defined as $Q^2 >> \Lambda^2_{\text{QCD}} $ with $x$ fixed and different from 1.

%
The corresponding amplitude reads
\begin{align}
    \label{eq:DIS_amplitude}
    \mathcal{M} = 
    \frac{e}{q^2}\bar{u}\left(k'\right)\gamma^{\alpha}u\left(k\right)\langle X | j_{\alpha}\left(0\right)|P\rangle,
\end{align}
where $|X\rangle $ represents the generic final state with $n$ on-shell particles and $j_{\alpha}$ 
is the electromagnetic current through which the photon interacts in the proton.
The cross section, which is proportional to the amplitude square, is then found to be proportional to the product between 
a leptonic and an hadronic part
\begin{align}
    \label{eq:DIS_xsec}
    \frac{d\sigma}{dx\,dQ^2} \propto \int \frac{d^3k'}{2E_{k'}\left(2\pi\right)^3}\, W^{\mu\nu}L_{\mu\nu}\,.
\end{align}
The leptonic tensor $L_{\mu\nu}$ can be easily computed within QED, while the hadronic one, containing 
the inforation about the interaction between the virtual photon and the hadron, can be parameterized as 
\begin{align}
    \label{eq:hadronic_tensor}
    W^{\mu\nu}\left(P,q\right) = 
    -&\left(g^{\mu\nu} -\frac{q^{\mu}q^{\nu}}{q^2}\right)F_1\left(x,Q^2\right) + \nonumber \\
    &\frac{1}{P\cdot q}\left(P^{\mu}-q^{\mu}\frac{P\cdot q}{q^2}\right)\left(P^{\nu}-q^{\nu}\frac{P\cdot q}{q^2}\right)
    F_2\left(x,Q^2\right),
\end{align}
$F_1 $ and $F_2 $ being scalar functions, called \textit{structure functions}, 
depending on the invariant quantities of the problem, namely $x$ and $Q^2$. If, more generally,
we allow $j_{\alpha}$ to be any electroweak current, there will be more than two scalar structure functions $F_i$.

%
Computing explicitly the leptonic tensor and plugging Eq.~\eqref{eq:hadronic_tensor} in Eq.~\eqref{eq:DIS_xsec} 
we can derive a general expression for the double differential cross section of DIS in the Center of Mass frame
%\begin{align}
%    \label{general cross xsec DIS}
%    \frac{d\sigma}{dx\, d} = \frac{4\pi \alpha^2}{x y Q^2}\left[xy^2F_1\left(x,Q^2\right) +\left(1-y\right)F_2\left(x,Q^2\right)\right].
%\end{align}
%or\footnote{$F_2\left(x,Q^2\right)-2 x F_1\left(x,Q^2\right) \equiv F_L\left(x,Q^2\right)$}
%\begin{align}
%    \label{general cross xsec DIS}
%    \frac{d\sigma}{dx\, dQ^2} = \frac{4\pi \alpha^2}{Q^4}\left[\left[1+\left(1-y\right)^2\right]F_1\left(x,Q^2\right) 
%    +\frac{\left(1-y\right)}{x}\left(F_2\left(x,Q^2\right)-2 x F_1\left(x,Q^2\right)\right)\right].
%\end{align} 

\begin{align}
    \label{general cross xsec DIS}
    \frac{d\sigma}{dx\, dQ^2} = \frac{2\pi \alpha^2}{Q^4}\left[\left[1+\left(1-y\right)^2\right]F_T\left(x,Q^2\right) 
    +\frac{2\left(1-y\right)}{x}F_L\left(x,Q^2\right)\right]\,,
\end{align}
where $\alpha = e^2/\left(4\pi\right)$ is the fine structure constant and the transverse and longitudinal structure functions 
are defined as
\begin{align}
    F_L = F_2 -2F_1\,, \,\,\,\, F_T = 2F_1\,.
\end{align}
%Following the same steps, considering a single parton of momentu $\xi P$ in the initial state,
%we can introduce partonic structure functions $\hat{F}_1$, $\hat{F}_2$ to parameterized the 
%parton level cross section $\hat{d\sigma}$ for the lepton-parton scattering.

%
The partonic cross section $d\hat{\sigma}$ for the scattering of the lepton off a single parton with momentum $\xi P$
can be computed through a simple leading order QED computation ($e^- + q \rightarrow e^- + q $) getting
\begin{align}
    \frac{d\hat{\sigma}}{dx\, dQ^2} = 
    \frac{4\pi \alpha^2}{Q^4}\left[1+\left(1-y\right)^2\right]\frac{1}{2} e^2\delta\left(x-\xi\right)\,,
\end{align}
from which we can read the expressions for the partonic structure functions 
$\hat{F}_1$ and $\hat{F}_2$
\begin{align}
    \hat{F}_2 = x e^2 \delta\left(x-\xi\right) = 2x \hat{F}_1\,.
\end{align}
%
Finally, using the parton model assumption of Eq.~\eqref{eq:parton_model}, we can write
an explicit expression for the full structure functions
%\begin{align}
%    \label{structure functions}
%    F_i\left(x,Q^2\right)= \int_x^1\frac{d\xi}{\xi}\sum_j f_j\left(\xi\right)\hat{F}_{ij}\left(\frac{x}{\xi},Q^2\right)
%\end{align}
%where $F_i = F_1, \, F_2/x $, so that

\begin{align}
    \label{eq:LO DIS xsex}
    &F_L\left(x,Q^2\right) = 0\,,\,\,\,\,\,\, 
    F_2\left(x,Q^2\right) = x\sum_a e_a^2\, q_{a/H}\left(x\right)\,. 
    %&\frac{d\sigma}{dx\,dQ^2} = \frac{2\pi\alpha^2 s}{Q^4}\left(\sum_f x f_f\left(x\right)Q_f^2\right)\left[1+\left(1-y\right)^2\right]\,.
\end{align} 
Eq.~\ref{eq:LO DIS xsex} makes manifest how DIS experiments probe the structure of the incoming hadron $H$,
giving direct access to the functions $q_{a/H}\left(x\right) $ encoding its internal distribution of quarks and gluons.
Also, it shows explicitly how in the parton model the structure functions do not depend on the energy scale.
Such property is known as Bjorken scaling, and its experimental observation was taken as a confirmation
of the composite nature of hadrons, confirming the picture of the parton model.

\comment{some final remark about parton model, maybe smt about sum rules and the evidence for the 
presence of the gluon, references.}

%%%%%%%%%%%%%%%%%%%%%%%%%%%%%%%%%%%%%%%%%%%%%%%%%%%%
\section{Improved Parton Model and factorization of collinear singularities}
In this section, starting from the parton model ideas, we briefly recall how to include Next-to-Leading-Order
(NLO) QCD corrections.
As we are going to see, when considering QCD corrections two important things happen.
First Bjorken scaling is broken, namely the structure functions acquire a non trivial scale dependence.
Second, when considering gluons emissions from the initial state particles, infrared collinear singularities,
arise. The universal factorization of such collinear poles and the subsequent renormalization of parton distribution functions
represent the main conceptual point of factorization in high energy processes, and 
will be discussed in the following.  

%
\subsection{Next-to-Leading-Order QCD corrections}
Considering a generic structure function $F$, following the parton model ideas we can write it as
\begin{align}
    \label{eq:strcuture_function}
    F\left(x,Q^2\right) = \sum_a \int_x^1 \frac{d\xi}{\xi}\, q^{\bare}_{a/H}\left(\xi\right)\, \hat{F}_a\left(\frac{x}{\xi},Q^2\right)\,,
\end{align}
with $\hat{F}_a$ representing the partonic level structure function for the scattering of a
quark off the virtual photon, and $q^{\bare}_{a/H}$ denoting the parton model PDFs
\footnote{Although here we are introducing this formula starting from the ideas of the parton model, 
Eq.~\eqref{eq:strcuture_function} can be proved in the Bjorken limit, and the bare PDF $q^{\bare}_{a/H}$ can be
defined in terms of operator matrix element. We will get back to this point in the next sections}. 
Such formula is valid at leading-twist, namely further corrections of non-pertubative origin are suppressed by powers
of $\Lambda_{QCD}/Q$.

%
From the previous section we know that at Born level $\hat{F}_a$ is proportional
to a delta function $e^2\delta\left(1-x\right)$. Considering QCD correction of order $\alpha_s$,
the additional diagram reported in Fig.~\ref{fig:NLO_QCD_DIS} have to be computed, 
accounting for virtual and real corrections due to the emission of a single gluon.
The full result has the following form
\begin{align}
    \label{eq:1_loop_dis}
    \hat{F}\left(x,Q^2\right) = e^2\left[\delta\left(1-x\right) 
    + \frac{\alpha_s}{2\pi}\left(P\left(x\right)\log\frac{Q^2}{Q_0^2} + R\left(x\right) \right)  \right]\,,
\end{align}
with
\begin{align}
    \label{eq:splitting_function}
    P\left(x\right) = C_F\left[\frac{1+x^2}{\left(1-x\right)_+} + \frac{3}{2}\delta\left(1-x\right)\right]\,.
\end{align}
Two observations can be done. Firstly, as anticipated above, beyond leading order
Bjorken scaling is broken by logarithms of $Q^2$, and the structure function acquires a $Q^2$ dependence.
Secondly, Eq.~\ref{eq:1_loop_dis} contains a logarithmic dependence on an infrared cutoff $Q_0^2$, pointing out 
the presence of an infrared divergence.
\begin{figure}[h]
    \includegraphics[scale=0.2]{LO_dis.pdf}
    \includegraphics[scale=0.2]{NLO_collinear.pdf}
    \caption{NLO QCD corrections. Virtual and real corrections.}
    \label{fig:NLO_QCD_DIS}
\end{figure}
%
Working in a light-cone gauge, such logarithmic divergence can be traced back to the square of the amplitude associated
to a gluon emission from the initial state quark.
It can be shown, that denoting as $k_{\perp}$ the longitudinal momentum of the emitted gluon,
we end up with a contribution of the form
\begin{align}
    \label{eq:collinear_div}
    \hat{F}_{q\, \gamma \rightarrow q\,g}\left(x,Q^2\right) =
    \int^{\,Q^2}\frac{dk_{\perp}^2}{k_{\perp}^2}\, \frac{\alpha_s}{2\pi}P\left(x\right) + ...\,,
    %\sim \alpha_s\left(Q^2\right)\,\log\frac{Q^2}{Q_0^2}\,,
\end{align}
where the ellipses stand for finite regular terms.
It is clear from Eq.~\ref{eq:collinear_div} that such term diverges in the region of small-$k_{\perp}$.
In order to regularize such pole we can introduce the infrared cutoff $Q_0^2$, getting the logarithmic 
contribution observed in Eq.~\ref{eq:1_loop_dis}.
Similarly, when considering multiple gluons emissions from initial state particles, terms of the kind 
$\left(\alpha_s\left(Q^2\right)\,\log\frac{Q^2}{Q_0^2}\right)^n$ show up.
Since all these terms are of order 1, if we accounted for only some of them we would spoil perturbation theory.
In order to get a proper perturbative expansion such terms have to be resummed at all orders.

\subsection{Factorization of collinear singularities}
The singularities described in the previous sections arises from the kinematic region where $k_{\perp}\rightarrow 0$,
namely when a gluon is emitted parallel to an initial state quark. For this reason they are often called collinear
singularities.
To understand how to deal with such terms, one needs to realize that the limit of small$-k_{\perp}$
corresponds to the long-range (low energy) regime of the strong interaction and therefore cannot be treated
within perturbation theory.
We can then consider the parton distributions introduced through the parton model as bare, unmeasurable
quantities, and use them to reabsorb the collinear singularities. In this way, all the dependence on
low energy phenomena can be factorized in the parton distribution functions, leaving the hard
cross sections free from collinear singularities.

%
Starting from the log divergent contribution appearing in Eq.~\eqref{eq:1_loop_dis}, we can introduce an
additional unphysical scale $\mu_F$ and write
$\log\frac{Q^2}{Q_0^2} = \log\frac{Q^2}{\mu_F^2} + \log\frac{\mu_F^2}{Q_0^2} $.
Looking back at Eq.~\eqref{eq:strcuture_function},
the infrared divergent partonic structure function can then be written as
\begin{align}
  \label{eq::IRsubtraction}
  \hat{F}\left(\xi,Q\right) = 
  \int_{\xi}^1 \frac{d\eta}{\eta} \,\Gamma\left(\frac{\xi}{\eta},\mu_F\right)
  \hat{F}_{\text{reg}}\left(\eta,\frac{Q}{\mu_F}\right) ,
\end{align}
with
\begin{align}
    &\Gamma\left(y,\mu_F\right) = \delta\left(1-y\right) 
    + \frac{\alpha_s}{2\pi}\left[P\left(y\right)\log\frac{\mu_F^2}{Q_0^2} + \Gamma_{finite}\left(y\right)\right]\,, \\
    &\hat{F}_{\text{reg}}\left(\eta,\frac{Q}{\mu_F}\right) = \delta\left(1-\eta\right)  
    + \frac{\alpha_s}{2\pi}
    \left[P\left(\eta\right)\log\frac{Q^2}{\mu_F^2}+ R\left(\eta\right) - \Gamma_{finite}\left(\eta\right) \right]\,.  
\end{align}
The new scale $\mu_F$ introduced above, often referred to as factorization scale, separates long and short distance 
contributions: everything which is below $\mu_F$ is considered to be in non perturbative regime 
and is factorized in the kernel $\Gamma$, which therefore contains the infrared poles.
The term $\Gamma_{finite}$ represents finite contributions that can be
kept into the subtraction kernel rather than in the hard structure function. Its specific choice is what defines
the renormalization scheme.  
Substituting Eq.~\eqref{eq::IRsubtraction} in Eq.~\eqref{eq:strcuture_function} it is easy to see that
we can write
\begin{align}
  F\left(x,Q\right) = \sum_a\int_x^1 \frac{d\eta}{\eta}\, 
  q_{a/H}\left(\eta,\mu_F\right)\hat{F}_{\text{reg}}\left(\frac{x}{\eta},\frac{Q}{\mu_F}\right),
\end{align}
where the renormalized quark PDFs $q_{a/H}$ is defined as
\begin{align}
    \label{eq:renormalized_pdf}
    q_{a/H}\left(x,\mu_F\right) = \int_x^1 \frac{d\eta}{\eta} \,
    q^{\bare}_{a/H}\left(\frac{x}{\eta}\right)\Gamma\left(\eta, \mu_F\right)\,.
\end{align}
Collinear poles are therefore factorized from the hard scattering structure function and reabsorbed into
the PDFs, following a procedure symilar to the one used for UV renormalization.
As a consequence PDFs acquire a non trivial dependence on an unphysical scale $\mu_F$,
which will be further described in in the next section.

%
To sum up, considering higher orders QCD corrections, the DIS structure functions can be written
as
\begin{align}
    \label{eq:dis_qcd}
    F\left(x,Q^2\right) = 
    \sum_a \int_x^1\frac{d\xi}{\xi}\,C_a\left(\frac{x}{\xi},\frac{Q^2}{\mu_F^2}, \alpha_s\right)q_{a/H}\left(\xi,\mu_F^2\right)
    +\mathcal{O}\left(\frac{\Lambda_{QCD}}{Q}\right)\,.
\end{align}
The coefficients functions $C_a$ appearing in Eq.~\eqref{eq:dis_qcd} correspond to the finite partonic structure 
functions $\hat{F}_{\text{reg}}$ after renormalization and subtraction of collinear singularities.
Their explicit expression will depend on the specific structure function under consideration
and on the choice for the renormalization and factorization schemes, defined when removing UV and collinear
singularities respectively. 
Once properly defined they can be computed order by order in perturbation theory as an expansion in the strong coupling
\begin{align}
    \label{eq:coeff_functions_expansion}
    C_a\left(x, \alpha_s\right) = C^{(0)}\left(x\right) + \alpha_s\, C^{(1)}\left(x\right) 
    + \alpha_s^2\, C^{(2)}\left(x\right) + ...\,,
\end{align}
where the first contribution (LO) recover the parton model predictions, the second one (NLO) corresponds to the QCD
corrections discussed above and the coming ones (N$^{\text{n}}$NLO) will refer to higher order corrections.
Differently from the initial formula of Eq.~\eqref{eq:strcuture_function}, which was written in analogy to
the parton model formulation, the parton distributions
have now acquired a scale dependence, which cancel against an analogue dependence in the coefficient functions,
leaving the physical structure function independent from any unphysical scales. Also, even if in our 
discussion we have only considered the quark channel, the sum over the flavour types $a$ now includes
also gluon initiated contributions, which formally start at NNLO.

%
So far we have discussed factorization for processes with only one hadron in the initial state, but
the same ideas and logic apply to inclusive enough high-energy hadron-hadron collisions
$$
H_1\left(p_1\right) + H_2\left(p_2\right) \rightarrow W\left(Q\right) + X\,,
$$
where $H_1$ and $H_2$ are the incoming hadrons, having momenta $p_1$ and $p_2$, $H$ represents
the particle produced in the hard scattering (Higgs or vector bosons, heavy quarks) and $W$
denotes any other particle appearing in the final state. In this case the factorization formula takes the form
\begin{align}
    \label{eq:hadron_hadron}
    \sigma&\left(p_1,p_2,Q\right) = \sum_{a,b}\int_{\tau}^1\, 
    dx_1 dx_2 \, \nonumber \\ 
    &q_{a/H_1}\left(x_1,\mu_F^2\right)q_{b/H_2}\left(x_2,\mu_F^2\right)
    \hat{\sigma}_{ab}\left(x_1p_1,x_2p_2,\frac{Q^2}{\mu_F^2},\alpha_s\right) 
    + \mathcal{O}\left(\frac{\Lambda_{QCD}}{Q}\right)\,,
\end{align}
where $\tau = \frac{Q^2}{s}$ and $s=\left(p_1+p_2\right)^2$.

%
The two factorized expressions given in Eqs.~\eqref{eq:dis_qcd},~\eqref{eq:hadron_hadron}
allow to connect cross sections for high-energy processes having hadrons in the initial states to hard scattering processes.
The former can be measured in collider experiments, while the latter 
can be computed in perturbation theory. The bit connecting perturbation theory with physical observables is given
by parton distribution functions.
The content of the factorization theorem is that all the dependence
on low mass phenomena is entirely contained in the PDFs, which are therefore nonpertubative and universal objects:
since they describe the internal structure of a given kind of hadron and have been decoupled from the short-distance details of
the specific process we consider, the PDFs appearing in the case of DIS must be the same considered for 
any other high-energy collision.

%%%%%%%%%%%%%%%%%%%%%%%%%%%%%%%%%%%%%%%%%%%%%%%%%%%%
\section{Parton Distribution Functions}
In the previous section we have introduced PDFs as some bare objects,
which are then used to reabsorb the infrared collinear poles coming from the fixed order computation of partonic
hard cross sections. Following this approach PDFs are introduced in the discussion through 
the parton model ideas, and defined as objects containing all the dependence 
of the physical observable on low energy physics. 
%
It is possible to give a rigorous operator definition of parton distributions,
which can be applied beyond perturbation theory and makes manifest the universal nature of PDFs.
In this section we briefly revise the formal definition of PDFs, their UV renormalization
and renormalization group equation, recovering the same results as in the previous section following
a different approach.

\subsection{PDFs operator definition}
\iffalse
Going back to the hadronic tensor defined in Eq.~\ref{eq:hadronic_tensor}, it can be shown that,
when working in the Bjorken limit, i.e.~large $Q$ and fixed $x$, $W^{\mu\nu}$ can be written as
\begin{align}
    \label{eq:factorization_hadronic_tensor}
    W^{\mu\nu}\left(P,q\right) = \sum_a \int_x^1 \frac{d\xi}{\xi}f_{a/A}\left(\xi,\mu\right) 
    H^{\mu\mu}_a \left(q,\xi P, \mu,\alpha_s\left(\mu\right)\right)\,\,+ \text{remainder}\,.
\end{align} 
The hard scattering coefficient  $H_a$ has two important properties. First it depends only on the parton type $a$
and not on the specific hadron $A$, so that it can be computed with the simplest choice of external parton.
Second it receives contributions only from momenta of order $Q$, so that it can be expressed as a power series in
$\alpha_s\left(Q\right)$ with finite coefficients. The content of the factorization theorem is that,
on the other hand, all the dependence
on low mass phenomena is entirely contained in the functions $f_{a/A}$.
\fi
%
Working in the Bjorken limit, it can be proved~\cite{Collins:1980ui,Collins:1981uw} that the bare unpolarized 
quark PDF appearing in Eq.~\eqref{eq:strcuture_function} is given by 
\begin{align}
	\label{eq::barepdf}                                                  
	f_\mathrm{q}^\bare\lp x \rp = \int \frac{d\xi^-}{4\pi} e^{-ixP^+\xi^-} 
	\langle \text{P}|\bar{\psi}_\mathrm{q}^\bare\lp\xi^-\rp\gamma^+ \,   
	U\lp\xi^-,0\rp \psi_\mathrm{q}^\bare\lp 0\rp  |\text{P}\rangle\, ,   
\end{align}
where $|\text{P}\rangle$ denotes a hadronic state with momentum $P^\mu = \lp
P^0,0,0,P^z\rp$, and $P^{\pm}=\frac {\lp P^0 \pm P^z \rp}{ \sqrt{2}}$ are
light-cone coordinates. The index $\mathrm{q}$ identifies the parton under
investigation. For instance, in a theory where we only consider the four
lightest quarks, we have $q=u,d,s,c$. The momentum carried by the parton is
$k^\mu = x P^\mu$, $\psi_\mathrm{q}^\bare$ is the bare quark field operator and the
Wilson line $U$ is given by 
\begin{align}
	\label{eq::wilsonline}                                                      U\lp\xi^-,0\rp = \text{P}\exp 
    \lp -ig\int_0^{\xi^-}d\eta^- A^{\bare\,+}\lp \eta^- \rp \rp\, .         
\end{align}
An analogous definition can be given for the gluon bare PDFs, denoted as
$f_g^\bare\lp x \rp$. The superscripts $\bare$ in the above expressions identify
bare fields: the matrix elements that enter in the definition of
$f_\mathrm{q}^\bare$ are ultraviolet divergent, and therefore need to be
renormalized.
Renormalized parton distributions are usually defined by minimal
subtraction, and the relation between the bare and the renormalized quantities
is given by
\begin{align}
	\label{eq:RenormPDF}                                   
	f_a^\bare\lp x \rp = \sum_{b}\int_x^1\frac{dy}{y}\,\text{Z}_{ab}\lp\frac{x}{y},\mu \rp f_b\lp y,\mu^2 \rp\, , 
\end{align}
where the indices $a$ and $b$ run over all the parton types (gluon and flavors
of quarks) and $\mu$ denotes the renormalization scale introduced by the minimal
subtraction scheme. 

Focusing on the quark PDFs for now, the renormalized distributions introduced
above have a compact support given by the interval $[-1,1]$. 
To recover the conventions of the previous sections, used for
phenomenological applications, it is customary to define the PDFs on the
interval $[0,1]$, and to introduce independent functions for the quarks and the
antiquarks, which we have previously denoted as $q(x,\mu^2)$ and $\bar{q}(x,\mu^2)$ respectively.
The relation between $f_q$, $q$ and $\bar{q}$ is 
\begin{equation}
    \label{eq:DefFQQbar}
    f_\mathrm{q}\lp x,\mu^2\rp = 
    \begin{cases}
        \phantom{-}q(x,\mu^2)\, , &\quad \mathrm{if}\ x>0\, , \\
        -\bar{q}(-x,\mu^2)\, , &\quad \mathrm{if}\ x<0 \, .
    \end{cases}
\end{equation}
It is useful to consider the symmetrised and antisymmetrised combinations of
$f_\mathrm{q}$ in the interval $x\in[0,1]$:
\begin{eqnarray}
	\label{eq:fsym}
	f^\mathrm{sym}_\mathrm{q}(x,\mu^2)  &= f_\mathrm{q}(x,\mu^2) + f_\mathrm{q}(-x,\mu^2) 
	\, , \\
	\label{eq:fasym}
	f^\mathrm{asy}_\mathrm{q}(x,\mu^2)  &= f_\mathrm{q}(x,\mu^2) - f_\mathrm{q}(-x,\mu^2) \, .
\end{eqnarray}
It can be readily shown that
\begin{align}
    f^\sym_\mathrm{q}(x,\mu^2) &= 
    q(x,\mu^2) - \bar{q}(x,\mu^2) = q^-(x,\mu^2) \, , \\
    f^\asy_\mathrm{q}(x,\mu^2) &= 
    q(x,\mu^2) + \bar{q}(x,\mu^2) = q^+(x,\mu^2) \, .
\end{align}
where $q^+$ and $q^-$ are defined by the equations above. The flavor
decomposition can be rewritten by collecting the quark fields in a vector, \eg\
$\psi = \lp \psi_u,\psi_d,\psi_s,\psi_c\rp$, and defining the following nonsinglet bare
PDFs:
% \begin{eqnarray}
%     \label{eq:f3Def}
%     f_3^\bare(x) &= \int \frac{d\xi^-}{4\pi} e^{-ixP^+\xi^-} 
% 	\langle \text{P}|\bar{\psi}^\bare\lp\xi^-\rp \lambda_3 \gamma^+ \,   
% 	U\lp\xi^-,0\rp \psi^\bare\lp 0\rp  |\text{P}\rangle\, ,  \\
%     \label{eq:f8Def}
%     f_8^\bare(x) &= \int \frac{d\xi^-}{4\pi} e^{-ixP^+\xi^-} 
% 	\langle \text{P}|\bar{\psi}^\bare\lp\xi^-\rp \lambda_8 \gamma^+ \,   
% 	U\lp\xi^-,0\rp \psi^\bare\lp 0\rp  |\text{P}\rangle\, ,  \\
%     \label{eq:f15Def}
%     f_{15}^\bare(x) &= \int \frac{d\xi^-}{4\pi} e^{-ixP^+\xi^-} 
% 	\langle \text{P}|\bar{\psi}^\bare\lp\xi^-\rp \lambda_{15} \gamma^+ \,   
% 	U\lp\xi^-,0\rp \psi^\bare\lp 0\rp  |\text{P}\rangle\, ,  
% \end{eqnarray}
\begin{eqnarray}
    \label{eq:fADef}
    f_A^\bare(x) &= \int \frac{d\xi^-}{4\pi} e^{-ixP^+\xi^-} 
	\langle \text{P}|\bar{\psi}^\bare\lp\xi^-\rp \lambda_A \gamma^+ \,   
	U\lp\xi^-,0\rp \psi^\bare\lp 0\rp  |\text{P}\rangle\, , 
\end{eqnarray}
where $A=3,8,15$, and we have used the Gell-Mann matrices
\begin{eqnarray}
    \lambda_3=
    \begin{pmatrix}
        1 & 0 & 0 & 0\\
        0 & -1& 0 & 0\\
        0 & 0 & 0 & 0\\
        0 & 0 & 0 & 0
    \end{pmatrix}\, , \quad
    \lambda_8=
    \begin{pmatrix}
        1 & 0 & 0 & 0\\
        0 & 1& 0 & 0\\
        0 & 0 & -2 & 0\\
        0 & 0 & 0 & 0
    \end{pmatrix}\, , \quad
    \lambda_{15}=
    \begin{pmatrix}
        1 & 0 & 0 & 0\\
        0 & 1& 0 & 0\\
        0 & 0 & 1 & 0\\
        0 & 0 & 0 & -3
    \end{pmatrix}\, . 
\end{eqnarray}
In this notation $f_3$ corresponds to $f^{u-d}=f_u-f_d$, $f_8=f^{u+d-2s}$, and
so on. The symmetrised and antisymmetrised combinations map directly into the
so-called {\em evolution basis} for the PDFs that is normally used in
phenomenological studies, see \eg\ Ref.~\cite{Vogt:2004ns} for a detailed
definition of the flavor decomposition. More specifically, we have:
\begin{align}
    f^\asy_{3}  &= u^+ - d^+ = T_3 \, , \\
    f^\sym_{3}  &= u^- - d^- = V_3 \, , \\
    f^\asy_{8}  &= u^+ + d^+ - 2 s^+ = T_8 \, , \\
    f^\sym_{8}  &= u^- + d^- - 2 s^- = V_8 \, , \\
    f^\asy_{15} &= u^+ + d^+ + s^+ - 3 c^+ = T_{15} \, , \\
    f^\sym_{15} &= u^- + d^- + s^- - 3 c^- = V_{15} \, .
\end{align}

%
As mentioned above the bilocal operator products of Eq.~\ref{eq::barepdf} requires renormalization.
The corresponding renormalization group equations are the Altarelli-Parisi equations fo PDFs.
Considering on-shell incoming partons, a straightforward 1-loop computation gives
\begin{align}
    \label{eq:1-loop_partonic_pdf}
    f^{\bare}_{a/b}\left(x,\epsilon\right) = \delta_{ab}\,\delta\left(1-x\right)
    + \left[\frac{1}{\epsilon_{UV}}-\frac{1}{\epsilon_{IR}}\right]\frac{\alpha_s}{\pi}P_{a/b}^{(1)}\left(x\right)
    + \mathcal{O}\left(\alpha_s^2\right)\,.
\end{align}
Working in $\overline{MS}$ the UV pole $1/\epsilon_{UV}$ is removed through renormalization,
and we are left with the renormalized partonic PDFs. Such object, despite being UV finite, does contain IR poles.

%
We can now see how the formal approach followed here recovers the picture given in the previous section.
Taking as example the case of DIS structure function, we can apply Eq.~ to the case of incoming partons
\begin{align}
    \label{eq:partonic_structure_function}
    \hat{F}_{a} = \sum_b\int_x^1\,\frac{d\xi}{\xi}\,f_{b/a}\left(\xi,\epsilon\right)
    H_b\left(\frac{x}{\xi},\frac{Q}{\mu}, \alpha_s, \epsilon\right).
\end{align}
Consider the power expansions for $\hat{F}_{a}$ and $H_b$ and using the 1-loop expression for the
renormalized partonic PDF \comment{add them} we get
\begin{align}
    \label{eq:IR_subtraction_from_formal_definition}
    &H^{(0)}_a = \hat{F}^{(0)}\,,\\
    &H^{(1)}_a = \hat{F}^{(1)} + \frac{1}{2\epsilon}\sum_b\int_x^1 \frac{d\xi}{\xi}P_{a/b}^{(1)}\left(\xi\right)
    G_b\left(\frac{x}{\xi},\frac{Q}{\mu}, \alpha_s\right)\,.
\end{align}
Therefore we find back the prescription introduced in the previous section: in order to compute
the hard scattering cross sections, one should calculate the structure function at the parton level and subtract
from it certain IR terms divergent terms proportional to the splitting kernel and the Born cross section.
Such terms, identified as collinear emissions in the previous sections, here are computed in a process independent
way starting directly from the formal definition of PDFs.

\subsection{DGLAP evolution equations}