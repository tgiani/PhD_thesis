\chapter{PDFs from quasi-PDFs matrix elements}
Data for equal-time correlators coming from first lattice QCD simulations
have started appearing and have gotten into already a relatively advanced stage over the last few years
\cite{Lin:2014zya,Alexandrou:2015rja,Chen:2016utp,Alexandrou:2016jqi,Zhang:2017bzy,Alexandrou:2017huk,Lin:2017ani,Chen:2017gck,Alexandrou:2018pbm,Chen:2018xof,Chen:2018fwa,Alexandrou:2018eet,Liu:2018uuj,Lin:2018qky,Fan:2018dxu,Liu:2018hxv,Alexandrou:2019lfo,Izubuchi:2019lyk}.
This gives an idea of what PDFs from the lattice might look like, not only for
nonsinglet quark PDFs of the nucleon, but also for the pion PDF and distribution
amplitude, as well as for the gluon PDF of the nucleon. Given the general
interest shown by the community, a quick improvement in the technologies
involved in such lattice simulations is to be expected in the next few years. A
great quantity of increasingly precise lattice data is then likely to be
available in the near future, requiring detailed studies about the possible
impact they might have on the overall precision of PDFs determination.

Despite the increasing number of numerical results becoming available, an
optimal strategy for reconstructing the PDFs from these data has not been
entirely addressed yet. The approach which has been initially used within the lattice community
(and that is still employed in many analyses)
consists in approximating the quasi- or pseudo-PDFs by mean of a discrete Fourier transform,
starting from the limited number of points for the corresponding equal-time correlator available from numerical simulations. 
The continuous function resulting from this procedure
is subsequently convoluted with the perturbative matching coefficients relating euclidean and light-cone distributions,
in order to obtain the final PDF.
The numerical error introduced by this procedure is rather large and difficult to control,
so that this procedure generally provides unstable and inaccurate results.
This problem was first addressed within the lattice community in Ref.~\cite{Karpie2019}
where a series of possible approaches to tackle the problem of incomplete and discretized Fourier
transform has been presented.

In this Chapter, we exploit the momentum space factorization of quasi-PDFs in PDFs and perturbatively
computable coefficients to extract nonsinglet distributions from the data of
Refs.~\cite{Alexandrou:2018pbm,Alexandrou:2019lfo}, using the {\tt NNPDF} framework described
in Chapter~\ref{ch:nnpdf_methodology} and treating the lattice data on the same footing as experimental ones.

%The discussion is focused on the
%quasi-PDFs case, but it can be extended to include data coming from different
%lattice approaches, like for example pseudo-PDFs
%\cite{Radyushkin:2017cyf,Orginos:2017kos,Karpie:2018zaz}, fictitious heavy/light
%quark~\cite{Detmold:2005gg,Braun:2007wv} or current-current correlators
%\cite{Ma:2014jla,Ma:2017pxb,Sufian:2019bol}, paving the way to a first global
%lattice QCD analysis~\cite{Ma:2017pxb}.

\section{quasi-PDFs in QCD}
%
Denoting by $\Gamma$ a generic Dirac structure and by the suffix $A$ the specific nonsinglet distribution
we want to consider, we can introduce the {\em Ioffe time distributions} \cite{Radyushkin:2017cyf,Braun:1994jq}, 
defined as the matrix element between nucleon states with momentum $P$
\begin{align}
	\label{eq:Ioffe}
	M^\bare_{\Gamma,A}\left(z,P\right) &= \langle P |\mathcal{M}^\bare_{\Gamma,A}\left(z\right) |P\rangle \, ,
\end{align}
with
\begin{align}
	\label{eq:bilocal}
	\mathcal{M}^\bare_{\Gamma,A}\left(z\right)= \bar{\psi}^\bare\lp z\rp \lambda_A \Gamma \,   
	U\lp z,0\rp \psi^\bare\lp 0\rp.
\end{align}
Eqs.~\eqref{eq:Ioffe}, \eqref{eq:bilocal} represent the QCD generalization of the scalar quantity defined 
in Eq.~\eqref{eq::ME}.
The vector bilocal operator obtained for $\Gamma=\gamma^\mu$ can be decomposed
in terms of two form factors depending on Lorentz invariants:
\[
    \label{eq:IoffeDecomposition}
    M^\bare_{\gamma^\mu,A}\left(z, P\right)
    = 2 P^\mu\,\text{h}^{(0)}_{\gamma^\mu,A}\lp z\cdot P,z^2 \rp
    + z^\mu\, \tilde{\text{h}}^{(0)}_{\gamma^{\mu},A}\lp z\cdot P,z^2 \rp \, . 
\]
The light-cone PDF in Eq.~\eqref{eq:fADef} is obtained by taking the Fourier
transform with respect to $z^-$ of a Ioffe time distribution with $\Gamma =
\gamma^+$, $z=\left(0,z^-,0_{\perp}\right)$ and $P=(P^+,0,0_{\perp})$, given
by
\begin{align}
	\label{eq:Ioffelightcone}
    M^\bare_{\gamma^+,A}\left(z,P\right) =  
    \langle \text{P}|\bar{\psi}^\bare\lp z^-\rp \lambda_A \gamma^+ \,   
	U\lp z^-,0\rp \psi^\bare\lp 0\rp  |\text{P}\rangle
	= 2 P^+\,\text{h}^{(0)}_{\gamma^+,A}\lp z^- P^+,0 \rp\, .
\end{align} 
Choosing instead a pure spatial direction $z=\left(0,0,0,z\right)$, and taking
the Fourier transform with respect to $z$, we obtain the definition of the
quasi-PDF  
\begin{align}
	\label{eq::bareqpdf}                                                 
	\tilde{f}_A^{(0)}\lp x, P_z\rp = 
	\int_{-\infty}^{\infty}\frac{dz}{4\pi}\,e^{i x \text{P}_z z}\,M^\bare_{\Gamma,A}\left(z,P\right)\, .
\end{align}
Choosing for example $\Gamma = \gamma^0$ we get
\begin{align}
	\label{eq::bareqpdf}                                                 
	\tilde{f}_A^{(0)}\lp x, P_z\rp = 
	2 P_0 \, \int_{-\infty}^{\infty}\frac{dz}{4\pi}\,e^{i x \text{P}_z z}\,\text{h}^{(0)}_{\Gamma,A}\lp z P_z,z^2 \rp, 
\end{align}
which will be the case considered in this work. As in the case of standard PDFs
(which from now on we will call {\em light-cone}\ PDFs), the matrix elements
defining the quasi-PDFs contain UV divergences, and need to be renormalized. The
main features of the perturbative renormalization of a bilocal operator, 
as the one appearing in
Eq.~\eqref{eq::bareqpdf}, have been described in Chapter~\ref{ch:scalar_model}
in the context of the scalar theory.
Despite the details of the full QCD computation are of course more complicate,
see Ref.~\cite{Ishikawa:2017faj}, as mentioned before 
its main features and results are well described by the simple scalar case.
As we found out looking at the scalar QFT,
the position space operator appearing in Eq.~\eqref{eq::bareqpdf} can be multiplicatively
renormalized, according to
\begin{align}
    \text{h}_{\Gamma,A}\lp z P_z,z^2,\mu \rp = Z_{A}(z)\,
    e^{\delta m |z|} \text{h}^\bare_{\Gamma,A}\lp z P_z,z^2 \rp\,.
\end{align}
The only difference with respect to the scalar case is given by
the exponential factor $e^{\delta m |z|}$, which reabsorbs the power divergence
from the Wilson line which appears in the QCD case. The position-dependent factor $Z_{A}(z)$ takes care of
the remaining UV logarithmic divergences.  
%
Importantly, we recall that the quasi-PDFs retain a dynamical dependence on the hadron momentum
$P$, unlike PDFs, which are defined to be invariant under Lorentz boosts. Also,
their support is defined to be the full real axis.

The interest in quasi-PDFs comes from the potential to relate them to light-cone
PDFs, as detailed in Sec.~\ref{sec::momentumspace};
factorization allows us to rewrite the quasi-PDFs as a convolution of the
light-cone PDFs with a coefficient function that can be computed in perturbation
theory, up to corrections that are suppressed by inverse powers of $P_z$.
It follows that they can be written as
\begin{align}
	\label{eq::pdftoqpdf}                                                                             
	\tilde{f}_A\lp x , {\mu}^2 \rp =                                                               
	\int_{-1}^{1} \frac{dy}{|y|}\, C_A\lp\frac{x}{y},\frac{\mu}{|y|P_z},\frac{\mu}{\mu'} \rp  f_A\lp y, {\mu'}^2\rp 
	+ \mathcal{O}\lp \frac{M^2}{P_z^2},\frac{\Lambda^2_{\text{QCD}}}{P_z^2} \rp\, ,                   
\end{align}
where the terms $\mathcal{O}\lp
\frac{M^2}{P_z^2},\frac{\Lambda^2_{\text{QCD}}}{P_z^2} \rp $ include
the power corrections suppressed by the hadron momentum. The functions $C_A$,
usually called matching coefficients, depend on the choice of the
renormalization scheme, and on the kind of quasi-PDF under consideration. The
first matching expressions, for all Dirac structures, were derived in
Ref.~\cite{Xiong:2013bka}, using a simple transverse momentum cutoff scheme. In
later works, matching coefficients were derived that relate the quasi-PDFs in
different renormalization schemes to light-cone PDFs in the $\MSb$ scheme. The
matching from $\MSb$ quasi-PDFs was first considered in
Ref.~\cite{Wang:2017qyg}, both for non-singlet and singlet quark PDFs, as well
as for gluons. Even though one can choose operators for the latter that do not
mix with singlet quark quasi-PDFs under
renormalization~\cite{Zhang:2018diq,Li:2018tpe}, mixing under matching is
inevitable. 
No mixing of the flavour nonsinglet sector with flavour singlet or
gluon sectors occurs, as stated in Eq.~\eqref{eq::pdftoqpdf}. Ref.~\cite{Wang:2017qyg} did not, however, address the
known issue of self-energy corrections, exhibiting a logarithmic UV divergence.
This was resolved in Ref.~\cite{Izubuchi:2018srq} by adding terms outside of the
plus prescription in the matching coefficient. As noticed in
Ref.~\cite{Alexandrou:2018pbm}, such prescription for renormalizing this
divergence violates vector current conservation, \ie\ the integral of the
matched PDF is different from the integral of the input quasi-PDF, and not
necessarily equal to 1 over the whole integration range. As a remedy, a modified
matching expression, which is given explicitly in Eq.~\eqref{eq::matching} of
App.~\ref{app:coefficients}, was proposed in Ref.~\cite{Alexandrou:2018pbm}. It
consists in resorting to pure plus functions when subtracting the logarithmic
divergence in self-energy corrections. However, this is an additional
subtraction with respect to the minimal subtraction of the $\MSb$ scheme and
thus, defines a modified $\MSb$ scheme, the so-called $\MMSb$ scheme. As such,
it requires the quasi-PDF to be expressed in this modified scheme. The
expression for the conversion of $\MSb$-renormalized matrix elements to the
$\MMSb$ scheme was worked out in Ref.~\cite{Alexandrou:2019lfo} and we refer to
it for the details of the procedure. Nevertheless, this modification is
numerically very small, as also shown in Ref.~\cite{Alexandrou:2019lfo}. An
alternative modification of the $\MSb$ scheme that guarantees vector current
conservation was derived in an updated version of Ref.~\cite{Izubuchi:2018srq}.
This defines the so-called ``ratio'' scheme. In this scheme, only pure plus
functions are used, like in the $\MMSb$ scheme, but the modification is done
also for the ``physical'' region of $0<z<1$ (in the notation of
Eq.~\eqref{eq::matching}). Thus, the expected numerical effect of this
modification is larger, as shown explicitly in Ref.~\cite{Alexandrou:2019lfo}.
For this reason, we choose to use the $\MMSb$ procedure, with details of the
lattice computation of the bare matrix elements and the renormalization in the
$\MMSb$ scheme outlined in the next section. Yet another possibility of
performing the matching consists in directly relating the quasi-PDFs in the
intermediate RI-type scheme to $\MSb$ light-cone PDFs. This was proposed in
Ref.~\cite{Stewart:2017tvs} for the unpolarized case. Obviously, such
procedure is equivalent to the one adopted here, with possibly different
systematic effects. All of the discussed papers considered the matching to only
first order in perturbation theory, but NNLO results for the matching coefficients have recently become available
in both position and momentum space \cite{Li:2020xml, Chen:2020ody, Braun:2020ymy, Chen:2020arf, Chen:2020iqi}.

