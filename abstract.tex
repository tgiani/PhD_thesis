\chapter*{Abstract}
A precise understanding of the proton structure, encoded in Parton Distribution Functions (PDFs), is required 
to make reliable predictions and analyses at the Large Hadron Collider (LHC), 
the main source of experimental data probing subnuclear interactions we have today.
PDFs have played a central role in the recent 
discovery of the Higgs boson and, since it is increasingly clear that any effect due to new physics 
will manifest itself as small deviations from the current theory, 
a precise determination of PDFs is likely to be a key ingredient 
for new physics studies.
%
The PDFs are formally defined as matrix elements of renormalized operators in Quantum Chromodynamics (QCD) involving hadronic states.
They are inherently non-perturbative quantities, and they are extracted from global QCD analysis over
experimental data using the so-called factorization theorems. 
Producing a new generation of PDFs, satisfying the precision and reliability requirements demanded by the current research,
is a challenging task which involves, together with the experimental data input, the development of robust
fitting methodologies, along with new physical ideas. 
%
In this thesis I present a number of progresses which have been developed in the last few years in the context of 
PDFs determination, some of which will lead to the next PDFs release by the NNPDF collaboration. 
In particular I will discuss a new framework for global PDFs determinations, the impact of new experimental data,
with particular emphasis on jets data, the inclusion of theory uncertainty in a PDFs fit 
and the treatment of heavy quarks distributions.
%
I will then discuss a set of recent ideas which would allow to extract PDFs from equal time correlators computable 
within the framework of lattice QCD, and I will present results regarding data coming from different 
lattice approaches and collaborations.

