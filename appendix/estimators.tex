\chapter{Statistical estimators}
\label{app:estimators}
\section{PDF distance}
Considering a set of $N_{\text{rep}}$ replicas $q_i$ of a given parton distribution $q$, the estimator for 
the expected true value of $q$ is given by
\begin{align}
    \langle q \rangle = \frac{1}{N_{\text{rep}}}\sum_{i=1}^{N_{\text{rep}}}q_i\,.
\end{align}
The square distance between two estimates for the expected true value obtained from two different fits
is given by \cite{Ball:2010de}
\begin{align}
    d^2\left(\langle q^{(1)} \rangle, \langle q^{(2)} \rangle\right) = 
    \frac{\left(\langle q^{(1)} \rangle - \langle q^{(2)} \rangle\right)^2}{\sigma^2\left[\langle q^{(1)} \rangle\right]
    + \sigma^2\left[\langle q^{(2)} \rangle\right]}\,,
\end{align}
with the variance of the mean given by 
\begin{align}
    \sigma^2\left[\langle q^{(k)} \rangle\right] = \frac{1}{N_{\text{rep}}}\sigma^2\left[q^{(k)}\right]\,,
\end{align}
with $\sigma^2\left[q^{(k)}\right]$ the variance of the variable $q^{(k)}_i$
\begin{align}
    \label{eq:}
    \sigma^2\left[q^{(k)}\right] = \frac{1}{N_{\text{rep}}-1}\sum_{i=1}^{N_{\text{rep}}}
    \left(q^{(k)}_i - \langle q^{(k)} \rangle\right)^2\,.
\end{align}
According to this definitions, $ d \simeq 1 $ corresponds to statistically equivalent fits, while, considering 
a fit with $N_{\text{rep}} = 100$ replicas , 
$d \simeq 10 $ corresponds to a difference of one-sigma in units of the corresponding variance.

\section{$\phi$ estimator}
\label{app:phi}
In Chapter~\ref{ch:th_error} we introduced the estimator $\phi$, defined as 
\begin{align}
    \label{eq:phi_app}
    \phi = \sqrt{\langle \chi^2_{\text{exp}}\left[T^{(k)}\right] \rangle - \chi^2_{\text{exp}}\left[\langle T^{(k)} \rangle\right]}\,.
\end{align}
Here, following Ref.~\cite{Ball:2014uwa}, we show how such quantity measures the standard
deviation over the replica sample in units of the data uncertainty.
%
Using the $\chi^2$ definition the first term of Eq.~\ref{eq:phi_app} can be written as
\begin{align}
    N_D \langle \chi^2_{\text{exp}}\left[T^{(k)}\right] \rangle = 
     \sum_{IJ}& \langle T^{(k)}_I  C_{IJ}^{-1} T^{(k)}_J \rangle  -  \sum_{IJ} D_I C_{IJ}^{-1} \langle T^{(k)}_J  \rangle
     \nonumber \\ &-  \sum_{IJ} \langle T^{(k)}_I \rangle C_{IJ}^{-1} D_J  
     + \sum_{IJ} D_I C_{IJ}^{-1} D_J\,,
\end{align}
so that subtracting $\chi^2_{\text{exp}}\left[\langle T^{(k)} \rangle\right]$ we get
\begin{align}
    N_D \left(\langle \chi^2_{\text{exp}}\left[T^{(k)}\right] \rangle 
    - \chi^2_{\text{exp}}\left[ \langle T^{(k)} \rangle \right]\right) =
    \sum_{IJ}& \langle T^{(k)}_I  C_{IJ}^{-1} T^{(k)}_J \rangle 
    - \sum_{IJ}& \langle T^{(k)}_I\rangle  C_{IJ}^{-1} \langle T^{(k)}_J \rangle
\end{align}
from which 
\begin{align}
    \phi^2 = \frac{1}{N_D}\sum_{IJ} C^{-1}_{IJ}T_{JI}\,,
\end{align}
i.e. the average over all the data points of the uncertainties and correlations of the theoretical
predictions, $T_{IJ}$, normalized according to the corresponding uncertainties and correlations of the
data as expressed through the covariance matrix $C_{I}$.