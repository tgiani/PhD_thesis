\chapter{Massive-b scheme}

\section{Matching coefficients}
\label{sec:app-splitting}
We collect for ease of reference the well-known matching coefficients
which relate
the four and five scheme PDFs. Up to $O(\alpha_s)$
\begin{equation}
  K_{ij}(z,Q^2) = \delta_{ij}\delta(1-z) +
  \alpha_s(Q^2)\,K_{ij}^{(1)}(z,Q^2) +{\cal O}(\alpha_s^2)\,.
\end{equation}
so that
\begin{equation}
  K_{ij}^{-1}(z,Q^2) = \delta_{ij}\delta(1-z) -
  \alpha_s(Q^2)\,K_{ij}^{(1)}(z,Q^2) +{\cal O}(\alpha_s^2)\,.
\end{equation}
The only non-zero contributions at order ${\cal O}(\alpha_s)$
are the heavy quark-heavy quark and the heavy quark-gluon
matching functions, which are respectively given by
\begin{equation}
  \label{eq:ks}
  \begin{split}
    K_{bb}^{(1)}\left(x,\frac{Q^2}{\mu_b^2}\right) & = \frac{C_F}{2\,\pi}{\left\{P_{qq}(x)\left[
          \ln{\frac{Q^2}{\mu_b^2}} -2\ln(1-x)-1 \right] \right\}}_{+} \\
    K_{bg}^{(1)}\left(x,\frac{Q^2}{\mu_b^2}\right) & = \frac{T_R}{2\,\pi}
    P_{qg}(x)\,\ln{\frac{Q^2}{\mu_b^2}}
  \end{split}
\end{equation}
where 
\begin{equation}
  P_{qg}(x) = \left(1\,-\,2\,x\,+\,2\,x^2\right)\quad {\rm and }\quad
  P_{qq}(x) = \frac{2}{1-x} -\left( 1+x \right)\,.
\end{equation}

\section{Massive coefficient functions}
\label{sec:app-coeff}
In this Appendix we summarize  the computation of the
coefficient functions in the massive scheme and of their massless
limit up to $O(\alpha_s)$. The NLO corrections are computed using
the extension of
Catani-Seymour subtraction for massive initial states developed
in Ref.~\cite{Dittmaier:1999mb} and extended to QCD
in Ref.~\cite{Krauss:2017wmx}. This way of preforming the computation 
has the main advantage of following closely that of the
five-flavor massive scheme, so that a direct comparison is much easier to
at the analytic level. Indeed, strictly speaking because of
Eq.~(\ref{eq:genfonll}) the massless limit is not needed. However, we
have computed it explicitly in order to check that it matches the
massless-scheme result (thereby verifying Eq.~(\ref{eq:canc})
explicitly), and also in order to produce Fig.~\ref{fig:mub-var},
which provides a further consistency check. Another advantage of this
way of performing the computation (though we do not use it here) is
that it allows for the computation of
the fully differential cross section in this scheme. 

\subsection{Leading order}
The leading order partonic cross section for the production of a Higgs
boson, accounting for the mass of the initial state $b$ and $\bar{b}$,
is given by
\begin{equation}
  \label{eq:bbh-lo}
  \hat{\sigma}_{0}(xs) = \left( \frac{g_{b\bar{b}H}^2\,\beta_0\,\pi}{6}
  \right)\delta(xs -m_H^2) = \sigma_0 \,x\,\delta\left( x-\frac{m_H^2}{s} \right)
\end{equation}
where 
\begin{equation}
  \label{eq:s0}
  \sigma_0 = \frac{g_{b\bar{b}H}^2 \,\beta_0 \, \pi}{6\,m_H^2}\,,
  \quad {\rm and}\quad  \beta_0 = \sqrt{1-\frac{4\,m_b^2}{m_H^2}}\,.
\end{equation}
where $g_{b\bar{b}H}$ is the coupling of the $b$ quark to the Higgs
boson, obtained as the mass of the quark divided by vacuum expectation
value of the Higgs sector:
\begin{equation}
  g_{b\bar{b}H} = \frac{m_b}{v}\,.
\end{equation}
In the following we will also use the notation
\begin{equation}
  {\cal B}(x) \equiv \hat{\sigma}_{0}(xs)\,,\quad {\rm and} \quad
  {\cal B} \equiv \hat{\sigma}_{0}(s)\,.
\end{equation}

\subsection{Next-to-leading order: $b\bar{b}$-channel}
The next to leading order corrections to the Higgs production in
bottom quark fusion consist in virtual corrections (${\cal V}$) to the
left diagram of Fig.~\ref{fig:massive-4fs}, as well as of real
emission corrections (${\cal R}$) , represented by the central diagram of
Fig.~\ref{fig:massive-4fs}.
Both this contributions are separately divergent when the additional
gluon, real or virtual, becomes soft, though the final result remains
finite. In order to handle these soft divergences we employ the
subtraction scheme defined in~\cite{Krauss:2017wmx}. This implies that
we need two more ingredients: a subtraction term, ${\cal S}$, and its
integral over the gluon phase space, ${\cal I} = \int{\rm d}\Phi_g{\cal S}$.
Our final result is then given by:
\begin{equation}
  \hat{\sigma}_{\rm NLO} = \int{\rm d}\Phi_1 {\cal B} +{\cal V}+{\cal
    I}+  \int{\rm d}\Phi_2 {\cal R} -{\cal S}\,.
\end{equation}

\subsubsection{Real corrections, and subtraction term}
The real emission partonic differential cross section, is given by
\begin{equation}
 \int {\rm d} \Phi_{2}  {\cal R} =\int {\rm d} \Phi_{2}  \left| \overline{{\cal M}}_{b\bar{b}Hg} \right|(s,t,u)\,,
\end{equation}
where 
\begin{equation}
  {\rm d} \Phi_{2}=\frac{1}{32\,\pi\,\beta\,s} {\rm d}\cos\theta\,
  \Theta(1+\cos\theta )\,\Theta(1-\cos\theta)\, , \quad
  \beta=\sqrt{1-\frac{4\,m_b^2}{s}}\,,
\end{equation}
and
\begin{equation}
  \label{eq:real}
  \begin{split}
    \left| \overline{{\cal M}}_{b\bar{b}Hg} \right|&(s,t,u) = 
    \frac{4}{3}\pi g_{b\bar{b}H}^2C_F\alpha_s
    \left\{
      \left( s-m_H^2\right)\left[ \frac{1}{m_b^2-t}
        + \frac{1}{m_b^2-u}\right]
    \right.\\
    &
    \left.+ (m_H^2-4\,m_b^2)
      \left[
        \frac{2\left( s-2\,m_b^2 \right)}{(m_b^2-t)(m_b^2-u)} -
        \frac{2\,m_b^2}{(m_b^2-t)^2} -
        \frac{2\,m_b^2}{(m_b^2-u)^2}
      \right]
    \right\}\,.
  \end{split}
\end{equation}
The Mandelstam variables in terms of scalar
products and $\cos\theta$ are given by
\begin{equation}
  \left\{
    \begin{split}
      t &= m_b^2 - \frac{s-m_H^2}{2}\left( 1-\beta\,\cos\theta\right)\\
      u &= m_b^2 - \frac{s-m_H^2}{2}\left( 1+\beta\,\cos\theta\right) 
    \end{split}
  \right.\,.
\end{equation}

In order to remove the soft divergence which appears in the  $s\rightarrow
m_H^2$ limit we need to construct a suitable  subtraction term. Using
the relevant 
equations in Ref.~\cite{Krauss:2017wmx} we find
\begin{align}
  \label{eq:sub}
  {\cal S} =
  \frac{2}{3}\,\pi \,\alpha_s\,C_F\,g_{b\bar{b}H}^2\,\beta_0^2\,m_H^2
  \frac{1}{\tilde{x}}&
  \biggl[\frac{2}{m_b^2-t}\left( P_{qq}(\tilde{x})-\frac{2\,\tilde{x}\,m_b^2}{m_b^2-t} \right) \nonumber \\
    &+\frac{2}{m_b^2-u}\left( P_{qq}(\tilde{x})-\frac{2\,\tilde{x}\,m_b^2}{m_b^2-u} \right)\biggr]\,,
\end{align}
where
\begin{equation}
  \tilde{x} = \frac{m_H^2-2\,m_b^2}{s-2\,m_b^2}\,.
\end{equation}

Combining Eqs.~(\ref{eq:real}) and~(\ref{eq:sub}) and factoring
the trivial $\frac{\alpha_s\,C_F\,\sigma_0}{\pi}$ dependence we get
\begin{equation}
  \begin{split}
    \frac{\alpha_s\,C_F\,\sigma_0}{\pi}
    \int{\rm d}\Phi_2 \left[{\cal R} -{\cal S}\right] & =
    \frac{\alpha_s\,C_F\,\sigma_0}{\pi}\frac{m_b^2}{2}
    \int_{-1}^1{\rm d}\cos\theta
    \left[\frac{s\,(s-m_H^2)^2}{(m_H^2-2\,m_b^2)(m_b^2-t)(m_b^2-u)}
    \right]\\
    & = -\frac{\alpha_s\,C_F\,\sigma_0}{\pi}
    \frac{1}{\beta_0}\left(\frac{ 1-\beta^2 }{\beta^2}\right)
    \frac{x}{\left( 1 - 2\,x - \beta^2\right)}\ln d\,,
  \end{split}
\end{equation}
where we defined
\begin{equation}
  d \equiv \frac{1+\beta}{1-\beta}\,,
  \quad {\rm and}\quad x\equiv \frac{m_H^2}{s}\,.
\end{equation}

\subsubsection{Virtual corrections, and integrated subtraction term}
QCD virtual corrections to the Born process in this simple case
completely factorize in a vertex form factor:
\begin{equation}
  {\cal V} = \frac{\alpha_s\,C_F}{\pi}{\cal B}\,\delta_g\,,
\end{equation}
with
\begin{align}
  \delta_g &= -1 - L_{\lambda} +
  \frac{(1-\beta_0^2)}{\beta_0} \ln d_0 \nonumber\\
  &- \frac{1+\beta_0^2}{2\,\beta_0}\left[-\ln d_0\,L_{\lambda} + \ln^2d_0+{\rm Li}_2\left(
      1-\frac{1}{d_0} \right) -\frac{\pi^2}{2} \right]\,,
\end{align}
where
\begin{equation}
  L_{\lambda} \equiv \frac{1}{\epsilon} + \ln\frac{4\,\pi\,\mu_R^2}{m_b^2}
  +{\cal O}(\epsilon^2)\,.
\end{equation}

The integrated subtraction term ${\cal I}$ is obtained by
integrating  ${\cal S}$, Eq.~(\ref{eq:sub}),  over the phase space of
the emitted gluon. This term can be separated into two pieces: a term
proportional to 
$\delta(1-x)$, which contains the singularity,
and a plus distribution:
\begin{equation}
  {\cal I} = \delta(1-x)\,I + \left\{ {\cal G}(x) \right\}_+\,,
\end{equation}
where
\begin{equation}
  \begin{split}
    I  = &2 +L_{\lambda}- \ln{\frac{(1+\beta_0^2)^2}{1-\beta_0^2}} +
    \frac{1-3\,\beta_0^2}{4\beta_0}\ln d_0\\
     &+\frac{1+\beta_0^2}{2\,\beta_0}
    \left[
      \frac{1}{2}\ln^2 d_0
      -\ln d_0\,\ln {\frac{4\,\beta_0^2}{(1+\beta)^2}}
      - L_{\lambda}\,\ln d_0 -1
      +2\,{\rm Li}_2\left(\frac{1}{d_0} \right)-\frac{\pi^2}{3}
    \right]\,,
  \end{split}
\end{equation}
and
\begin{equation}
  \{{\cal G}(x)\}_+ =  {\left\{P_{qq}(x)\left[ \frac{1+\beta^2}{2\beta}
        \ln{d} -1 \right] +
      \left( 1-x \right)\right\}}_{+}\,.
\end{equation}

\subsubsection{Final formulae, mass and PDF renormalization}
We now combine the various partial results obtained in the previous
subsections  into the  full expression for the $b\bar{b}$-channel
coefficient functions. First, however, we need to adjust 
$b$-quark mass and the PDFs.
Renormalization of the  $b$ mass leads to the   replacement
\begin{equation}
  g_{hb\bar{b}}^2 = g_{hb\bar{b}}^2(\mu_R^2) \left(
    1-\frac{\alpha_s\,C_F}{\pi}\left(
        \frac{3}{2}\ln\frac{m_b^2}{\mu_R^2}-2  \right) \right)\,.
\end{equation}
in $\sigma_0$, Eq.~\eqref{eq:s0}.

The massive $b$ PDF is free of collinear singularities and thus it
does
not have to
undergo subtraction: indeed it is scale independent.
However, we must perform the change of
renormalization scheme Eq.~\eqref{eq:change-of-scheme} which relates
the massive and massless schemes.
Up to ${\cal O}(\alpha_s)$ we get
\begin{align}
  B_{b\bar{b}}\left(x,\mu_R^2,\mu_F^2 ,\mu_b^2\right) &= 
  \biggl[ \sigma_0(\mu_R^2) \delta(1-x) + \nonumber \\ &\alpha_s(\mu_R^2)
    B_{b\bar{b}}^{(1)}\left(x,\mu_R^2,\mu_F^2 ,\mu_b^2\right) \biggr]
  +{\cal O}(\alpha_s^2)
\end{align}
where
\begin{equation}
  \begin{split}
  B_{b\bar{b}}^{(1)}&\left(x,\mu_R^2,\mu_F^2 ,\mu_b^2\right) =
  \frac{\sigma_0(\mu_R^2)\,C_F}{\pi}
  \left\{
    \left[
      \frac{3}{2}\ln\frac{\mu_R^2}{\mu_b^2}+2 + I + \delta_g
    \right]\delta(1-x) \right.\\
  &\left.
   + \int_0^1{\rm d}z \{{\cal G}(z) - 2\,K_{b\bar{b}}^{(1)}(z)\}_+z\,\delta(z-x)
   +  \int{\rm d}\Phi_2 \left[{\cal R} -{\cal S}\right] \right\}\,.
  \end{split}
\end{equation}

Performing the $z$ integration gives the final result
\begin{equation}
  \label{eq:b1-massive}
  \begin{split}
    B_{b\bar{b}}^{(1)}&\left(x,\mu_R^2,\mu_F^2 ,\mu_b^2\right) = \\
    &\frac{\sigma_0(\mu_R^2)\,C_F}{\pi}
    \Biggl\{
    \delta(1-x)
    \left[
      \xi -2 + \frac{3}{2}\left(\gamma_0\ln
        \frac{(1+\beta)^2}{4} -\gamma_0\ln \frac{m_H^2}{m_b^2}+
        \ln\frac{\mu_R^2}{\mu_F^2} \right)
    \right]\\
  +&\,4\,{\cal D}_1(1-x) 
  +  2
  \left[
    \gamma\ln\frac{(1+\beta)^2}{4}+\gamma\ln\frac{m_H^2}{m_b^2}+
    \ln{\frac{\mu_b^2}{\mu_F^2}}
  \right]\,
  {\cal D}_0(1-x)\\
  -& (2 + x + x^2)\left[\gamma\ln\frac{(1+\beta)^2}{4}
    +\gamma\ln\frac{m_H^2}{m_b^2}-\gamma\ln x+
    \ln{\frac{\mu_b^2}{\mu_F^2}}
    +2\ln(1-x)\right]+ x\,(1-x) \\
  -&  \frac{2\,\gamma\,\ln x}{1-x} -
    \frac{1}{\beta_0}\left(\frac{ 1-\beta^2}{\beta^2}\right)
    \frac{x}{\left( 1 - 2\,x - \beta^2\right)}\ln d \Biggr\}\,,
  \end{split} 
\end{equation}
where
\begin{align}
  \xi &= 1+\ln\left(\frac{1-\beta_0^2}{(1+\beta_0)^2}\right)+
  \frac{\left(5-7\beta_0^2\right)}{4\,\beta_0}\ln d_0 \nonumber \\
  &+\frac{\left(\beta_0^2+1\right)}{\beta_0} \left( 2\,{\rm
      Li}_2\left(\frac{1}{d_0}\right)+\frac{\pi ^2}{6} -\ln d_0
    \ln\frac{4\beta_0^2} {(1+\beta_0^2)(1+\beta_0)}\right)\,,
\end{align}
and
\begin{equation}
  \gamma = \frac{1+\beta^2}{2\,\beta}\,,\quad \gamma_0 =
  \frac{1+\beta_0^2}{\,2\beta_0}\, \quad \text{ and }
  \quad {\cal D}_n(x) ={\left( \frac{\ln^n(1-x)}{1-x} \right) }_{+}\,.
\end{equation}

\subsubsection{Massless limit}
The massless limit of the $b\bar{b}$-channel can be computed directly
from Eq.~\eqref{eq:b1-massive},
by setting $\beta=1$ everywhere except in the logarithms, where
one can use the simple expansion
\begin{equation}
  \label{eq:beta}
  \beta \sim 1\,-\,\frac{2\,x\,m_b^2}{m_H^2} + {\cal O}\left( \frac{m_b^4}{m_H^4} \right)\,.
\end{equation}
We get
\begin{equation}
  \label{eq:b1-massless}
  \begin{split}
    B_{b\bar{b}}^{(1),(0)}&\left(x,\mu_R^2,\mu_F^2,\mu_b^2 \right) = 
    \frac{\alpha_s \, C_F\,\sigma_0(\mu_R^2)}{\pi}
    \Biggl\{
      \delta(1-x)\left[ -1+ \frac{\pi^2}{3} + \frac{3}{2}
          \ln\frac{\mu_R^2}{\mu_F^2} \right] \\+&\,4\,{\cal
        D}_1(1-x) 
      +  2\left( 
      \ln{\frac{m_H^2}{\mu_F^2}}+ \ln{\frac{\mu_b^2}{m_b^2}}\right)\,{\cal D}_0(1-x)
      -\frac{2\ln x}{1-x}  \\
      &- (2 + x + x^2)\left[
        \ln\frac{m_H^2}{\mu_F^2} +\ln{\frac{\mu_b^2}{m_b^2}} + \ln \frac{(1-x)^2}{x}\right]+ x\,(1-x)  \Biggr\}\,.
  \end{split} 
\end{equation}
As it can be easily verified, this exactly corresponds to its massless
scheme equivalent, which can be found in Eq.~(A6) of Ref.~\cite{Harlander:2003ai}.
\subsection{Next-to-leading order: $bg$-channel}
In the presence of initial-state massive quarks, the cross-section for
the $bg$-channel is free of  soft or 
collinear divergences, and no subtraction is accordingly
necessary. Also in this case, however, we must  perform the scheme
change Eq.~\eqref{eq:change-of-scheme}. We get
\begin{equation}
  \begin{split}
    B_{bg}^{(1)}(x,\mu_R^2,\mu_F^2,\mu_b^2) & = \hat{\sigma}_{bg}(x,\mu_R^2)
    - \alpha_s\,\int_0^1{\rm d}z\, K_{bg}^{(1)}(z,\mu_F^2)\sigma(zs)\,\\
    & = \left. \hat{\sigma}_{bg}(x,\mu_R^2) - \frac{\alpha_s\,
        T_R\,\sigma_0}{\pi}\left[ \frac{x}{2}\,P_{qg}(x)
        \ln{\frac{\mu_F^2}{\mu_b^2}}\right]\,\right|_{x=\tfrac{m_H^2}{s}},
  \end{split}
\end{equation}
where
\begin{equation}
  \hat{\sigma}_{bg}(x,\mu_R^2) = \int{\rm d}\Phi_2^{(b)}
  \left| \overline{{\cal M}}_{bgHb} \right|^2(s,t,u)\,,
\end{equation}
and the subscript $(b)$ in $\Phi_2^{(b)}$ denotes the fact that now
the phase-space has a massive $b$ instead of a massless gluon, in the
final state. The color- and helicity-averaged square matrix element,
can be obtained from Eq.~\eqref{eq:real} using crossing symmetry. In
addition, we have to take into account that the gluon can have 8
possible colors (as opposed to 3 for a quark),
\begin{equation}
  \left| \overline{{\cal M}}_{bgHb} \right|^2(s,t,u)=
  - \frac{3}{8} \left| \overline{{\cal M}}_{b\bar{b}Hg} \right|^2(t,s,u)\,,
\end{equation}
where now the Mandelstam invariants are given by
\begin{equation}
  \left\{
    \begin{split}
      t &= 2\,m_b^2 +
      \frac{s}{32}\left((5-\beta^2)(\beta^2+4\,x-5) -
        (3+\beta^2)\,\Lambda\,\cos\theta\right)\\
      u &= m_b^2 +\frac{s}{32}\left((5-\beta^2)(\beta^2+4\,x-5) +
        (3+\beta^2)\,\Lambda\,\cos\theta\right), 
    \end{split}
  \right.\,
\end{equation}
where
\begin{equation}
  \Lambda = \sqrt{\left( 3+\beta^2 \right)^2 + 16 x^2 -
    8\,x\,\left(5-\beta^2\right)}\,,
\end{equation}
while the phase-space ${\rm d}\Phi_2^{(b)}$ is given by
\begin{equation}
  {\rm d}\Phi_2^{(b)} = \frac{\Lambda\,x}{32\,\pi\,\left( 3+\beta^2
    \right)\,m_H^2}\,{\rm d}\cos\theta\,
  \Theta(1+\cos\theta )\,\Theta(1-\cos\theta)\,.
\end{equation}

Performing the $\cos\theta$ integration gives
\begin{equation}
  \begin{split}
    \hat{\sigma}_{bg}(x,\mu_R^2) =&
    \frac{\alpha_s\,T_R\,\sigma_0(\mu_R^2)}{\pi}\frac{  x}{16 \beta_0
      \left(\beta^2+3\right)^3}\\
    &\times\Biggl\{- 64 \left(9 \beta^4+(40
     x-42) \beta^2 +8 x (4 x-9)+49\right){\rm arctanh}\left(\frac{\Lambda }{\beta^2+4
       x-5}\right)   \\
    &\frac{4096\,\Lambda\, \left(1-\beta^2 \right)
      \left(\beta^2+x-1\right)} {\left(-\Lambda +\beta^2+4 x-5\right)
      \left(\Lambda +\beta^2+4 x-5\right)} \\  
    &+\Lambda\,\left(5-\beta^2\right) \left(\beta^4+(4 x+22) \beta^2+44 x-71\right)
  \Biggr\}\,.
  \end{split}
\end{equation}

\subsubsection{Massless limit}
As in the case of the $b\bar{b}$ channel, taking the massless limit
requires setting $\beta=1$ everywhere except in the logarithms where
one can use Eq.~\eqref{eq:beta}, which gives
\begin{align}
  \label{eq:mlimbg}
  B_{bg}^{(1),(0)}(x,\mu_R^2,\mu_F^2,\mu_b^2) &=
  \frac{T_R}{\pi}\biggl\{  
    \frac{x}{2}\,P_{qg}(x)\left[
      \ln\left(\frac{(1-x)^2}{x}\right)+\ln\frac{m_H^2}{\mu_F^2}+
      \ln{\frac{\mu_b^2}{m_b^2}} \right]\nonumber \\ &-\frac{x}{4}(1-x)(3-7\,x) \biggr\}\,.
\end{align}
Once again, one can explicitly check that this exactly corresponds to
it massless limit counterpart, which can be found in Eq.~(A9) of Ref.~\cite{Harlander:2003ai}.
