\chapter{Lattice observables}
\label{app:scalar_lattice}
%%%%%%%%%%%%%%%%%%%%%%%%%%%%%%%%%%%%%%%%%%%%%%%%%%%%%%%%%%%%%%%
\section{Momentum space factorization}
\label{app:plus}
%%%%%%%%%%%%%%%%%%%%%%%%%%%%%%%%%%%%%%%%%%%%%%%%%%%%%%%%%%%%%%%%

In this appendix we report in detail some of the computations performed in
Sec.~\ref{sec::momentumspace}, to obtain the coefficient $C$ of
Eq.~\eqref{eq::C0} and its high momentum limit of Eq.~\eqref{eq::matching_scalar}. In
order to compute the Fourier transform of the coefficient $\tilde{C}$ entering
Eq.~\eqref{eq::fact0}, we perform a change variable, $\theta = \xi P_3 z_3 $,
and define $\eta = \frac{y}{\xi}$, so that
\begin{align}
    \frac{P_3}{2\pi} 
    & \int_{-\infty}^{\infty} dz_3\,
        e^{-i y P_3 z_3}\,
        \tilde{C}\left(x P_3 z_3, m z_3, \frac{\mu^2}{m^2} \right)
        = \frac{1}{x} \int_{-\infty}^{\infty} \frac{d\theta}{2\pi}\,
        e^{-i \eta \theta}\,
        \tilde{C}\left(\theta, \frac{m \theta}{x P_3}, \frac{\mu^2}{m^2} \right)
        = \nonumber\\
    \label{eq:CFourierTransform}
    &= \frac{1}{x} \biggl[
        \delta\left(\eta-1\right) - \alpha \int_0^1 d\xi\, \left(1-\xi\right)\,
        \int_{-\infty}^{\infty} \frac{d\theta}{2\pi}\, 
        e^{-i \left(\eta-\xi\right) \theta} \,
        \biggl(2 K_0\left(\frac{M\theta}{x P_3}\right)
        -\log\frac{\mu^2}{M^2}\biggr)\biggr]\, .
\end{align}
The Fourier transform of the Bessel function, obtained also in Ref.~\cite{Radyushkin:2016hsy}, can be computed using the integral
representation in Eq.~\eqref{eq::Bessel}, computing the gaussian integral over
$\theta$ first:
\begin{align}
    \int_{-\infty}^{\infty} \frac{d\theta}{2\pi}\,
    & e^{-i \left(\eta-\xi\right)\theta}\,
    \int_0^{\infty} \frac{dT}{T}\, e^{-T} 
    e^{-\left(\frac{M\theta}{x P_3}\right)^2 \frac{1}{4T}} 
    =  \frac{1}{\sqrt{\left(\eta - \xi\right)^2 + \frac{M^2}{x^2 P_3^2}}}\, ,
\end{align}
so that the $\mathcal{O}(\alpha)$ contribution to \eqref{eq:CFourierTransform}
can be written as 
\begin{align}
    \label{eq::appC0}
    \int_0^1 d\xi\,
    \left(1-\xi\right) 
    \left[ \frac{1}{\sqrt{\left(\eta-\xi\right)^2 + \frac{M^2}{x^2P_3^2}}} 
    - \delta\left(\xi-\eta\right) \log\frac{\mu^2}{M^2}\right]\, .
\end{align}
As mentioned in Sec.~\ref{sec::momentumspace}, the computation of the
large-$P_3$ limit when $\eta \in \left(0,1\right)$ requires additional care,
since the integrand develops a non-integrable divergence for $\xi=\eta$ when
$M^2/(x^2P_3^2) \to 0$. This issue was first addressed and solved in Ref.~\cite{Radyushkin:2017lvu} in the context of QCD.
Since in the scalar theory the same kind of Bessel function appears, its Fourier transform leads to an analogous singularity. 
In order to elucidate this problem, given a generic test
function $\phi\left(\xi\right)$, we consider the integral
\begin{align}
    \label{eq::limitApp}
%    \lim_{a\rightarrow 0}\,
        \int_0^1 d\xi\,
        \frac{\phi\left(\xi\right)}{\sqrt{\left(\eta-\xi\right)^2 + \kappa^2}}
\end{align}
in the limit where $\kappa\to 0$. Defining 
\begin{equation}
    \label{eq:GfunDef}
    G\left(\eta, \kappa^2\right) = 
    \int_0^1 \frac{d\xi}{\sqrt{\left(\xi - \eta\right)^2 + \kappa^2}}
\end{equation}
allows us to rewrite Eq.~\eqref{eq::limitApp} above as 
\begin{align}
    \label{eq:IntPlusDecomp}
    \int_0^1 d\xi\,
    \frac{\phi\left(\xi\right)}{\sqrt{\left(\eta-\xi\right)^2 + \kappa^2}}
    = \phi(\eta) G\left(\eta, \kappa^2\right) + 
    \int_0^1 d\xi\,
    \frac{1}{\sqrt{\left(\eta-\xi\right)^2 + \kappa^2}}
    \left(\phi\left(\xi\right)-\phi\left(\eta\right)\right)\, .
\end{align}
The divergence of the original integral is encoded in the function
$G\left(\eta,\kappa^2\right)$, which can be readily evaluated:
\begin{equation}
    \label{eq:GfunIntegral}
    G\left(\eta, \kappa^2\right) =
    \log \left(4\eta\left(1-\eta\right)\frac{1}{\kappa^2}\right) +
    \mathcal O\left(\kappa^2\right)\, .
\end{equation}
The integral on the RHS of Eq.~\eqref{eq:IntPlusDecomp} is convergent for $\kappa\to
0$, and we have
\begin{align}
    \int_0^1 & d\xi\, 
    \frac{1}{\sqrt{\left(\eta-\xi\right)^2 + \kappa^2}} 
    \left(\phi\left(\xi\right)-\phi\left(\eta\right)\right) = 
    \nonumber \\
    &= \int_0^1 d\xi\, 
    \frac{1}{\left| \xi - \eta\right|}
    \left(\phi\left(\xi\right)-\phi\left(\eta\right)\right) 
    + \mathcal{O}(\kappa^2) \nonumber \\
    \label{eq:PlusDistrLimit}
    &= \int_0^1 d\xi\,
    \frac{1}{\left| \xi - \eta\right|_+}\,
    \phi\left(\xi\right) 
    + \mathcal{O}(\kappa^2) \, .
\end{align}
Therefore, collecting both contributions, 
\begin{align}
    \frac{1}{\sqrt{(\eta - \xi)^2 + \kappa^2}} = 
    \delta(\eta - \xi) \log\left(4\eta (1-\eta) \frac{1}{\kappa^2}\right)
    + \frac{1}{\left|\eta - \xi\right|_+} + \mathcal{O}\left(\kappa^2\right)\, .
\end{align}

\section{Equivalence between pseudo- and quasi-PDF approaches}
\label{app:check}

As discussed at the end of Sec.~\ref{sec:factorization}, taking the
small-$z_3^2$ limit in position space is equivalent to taking the large-$P_3$
limit in momentum space. This can be verified at 1-loop by showing that the
coefficent functions of Eqs.~\eqref{eq::Cpseudo0} and~\eqref{eq::matching_scalar} are
related through a Fourier transform, as stated in Eq.~\eqref{eq::check}. Here we
report the details of the computation. Taking the Fourier transform of the
small-$z_3^2$ coefficient of Eq.\eqref{eq::Cpseudo0} and defining $\eta= y/x$ we
have 
\begin{align}
    \label{eq::checkapp}
	\frac{P_3}{2\pi}&\int_{-\infty}^{\infty} dz_3\, e^{-i y P_3 z_3 }\,
	\tilde{C}\left(x\nu, \mu^2 z_3^2 \right) = 
    \frac{1}{x}\int_{-\infty}^{\infty}\frac{d\theta}{2\pi}\, e^{-i\theta\eta}\,
	\tilde{C}\left(\theta, \frac{\mu^2\theta^2}{x^2 P_3^2} \right) \nonumber \\
    & = \frac{1}{x}\biggl[\delta\left(\eta-1\right) 
    +\alpha\log\frac{4 \left(xP_3\right)^2}{\mu^2 e^{2\gamma_E}}\int_0^1 d\xi\,
    \delta\left(\xi-\eta\right)\left(1-\xi\right) \nonumber \\
    &\,\,\,\,\,\,\,\,\,\,\,\,\,\,\,\,\,\,\,\,\,\,\,\,\,\,\,\,\,\,
    -\alpha\int_0^1 d\xi \left(1-\xi\right)\int_{-\infty}^{\infty} \frac{d\theta}{2\pi} 
    e^{-i \left(\eta-\xi\right) \theta} \,\log\theta^2\biggr]\, .
\end{align}
Following Ref.~\cite{Izubuchi:2018srq}, the Fourier transform of $\log\theta^2$
can be defined as
\begin{align}
    \label{eq::FTlog}
    \int& \frac{d\theta}{2\pi} e^{-i t \theta} \log \theta^2
    = \left[\frac{d}{d\tau} \int \frac{d\theta}{2\pi} e^{-it \theta} 
    \left(\theta^2\right)^{\tau}\right]_{\tau=0} \nonumber\\
    &= -2 \gamma_E \, \delta\left(t\right) - \frac{\theta\left(1 - |t|\right)}{|t|_{\left(+0\right)}}
    - \frac{\theta\left(|t| -1\right)}{|t|_{\left(+\infty\right)}}
    + \frac{1}{\left(t\right)^2}\,\delta\left(\frac{1}{|t|}\right)\, ,
\end{align}
with
\begin{align}
    \label{eq::distribution1}
    &\frac{1}{|t|}_{\left(+0\right)} 
    = \lim_{a\rightarrow 0}
    \left[\frac{\theta\left(|t|-a\right)}{|t|} +  \delta\left(|t|-a\right)\log a\right], \\
    &\frac{1}{|t|}_{\left(+\infty\right)} 
    = \frac{1}{\left(t\right)^2}\lim_{a\rightarrow 0}
    \left[\theta\left(\frac{1}{|t|}-a\right)|t| + \delta\left(\frac{1}{|t|}-a\right)\log a\right], \\
    &\delta\left(\frac{1}{|t|}\right) = \lim_{a\rightarrow 0}\delta\left(\frac{1}{|t|} - a\right)\, .
\end{align}
The proof of Eq.~\eqref{eq::FTlog} can be found, for example, in the Appendix A
and C of Ref.~\cite{Izubuchi:2018srq}, to which we refer for more details.
Setting $t = \eta-\xi $ and plugging everything in Eq.~\eqref{eq::checkapp},
remembering that $\xi \in \left[0,1\right]$, we get different answers depending
on the value of $\eta$. For $\eta \in \left[0,1\right]$, just the first two
terms in Eq.~\eqref{eq::FTlog} contribute, giving
\begin{align}
    \label{eq::cont1}
    \int_0^1 &d\xi \left[2 \gamma_E \, \delta\left(\eta-\xi\right) -
    \lim_{a\rightarrow 0}
    \left(\frac{\theta\left(|\eta-\xi|-\beta\right)}{|\eta-\xi|} +  
    \delta\left(|\eta-\xi|-a\right)\log a\right)\right]
    \left(1-\xi\right) \nonumber \\
    & = \log{e^{2\gamma_E}}\left(1-\eta\right) + 
    \left(1-\eta\right)\log{\eta\left(1-\eta\right)} +2\eta -1\, ,
\end{align}
while for $\eta > 1$ or $\eta < 0$ the third contribution in Eq.~\eqref{eq::FTlog} gives simply
\begin{align}
    \label{eq::cont2}
    -\int_0^1 d\xi\, \left(1-\xi\right)\frac{|\eta-\xi|}{\left(\eta-\xi\right)^2}\, .
\end{align}
Looking at the last term in Eq.~\eqref{eq::FTlog}, considering its contribution
to the convolution integral with the PDF and doing the integral over $x$ first
we find
\begin{align}
    \label{eq::cont3}
    \lim_{a\rightarrow 0}\int_0^1 \frac{dx}{x}\int_0^1 d\xi \,\left(1-\xi\right)
    \delta\left(\frac{1}{|\frac{y}{x}-\xi|} - a\right)f\left(x\right) 
    \propto \lim_{a\rightarrow 0} a^2 f\left(a\right) = 0.
\end{align}
Using Eqs.~\eqref{eq::cont1}, \eqref{eq::cont2}, \eqref{eq::cont3} in
Eq.~\eqref{eq::checkapp} we find back the expression for
$C\left(\eta,\frac{\mu^2}{x^2 P_3^2}\right)$ as in Eq.~\eqref{eq::matching},
which completes our check.

\section{quasi-PDFs and their moments}
\label{app:moments}

As mentioned in the introduction of this paper, the works where the concept of
quasi-PDF was first introduced have been criticized in
Refs.~\cite{Rossi:2017muf, Rossi:2018zkn}, where it was argued that such
approach does not give access to the full nonperturbative PDF. In support of
their argument, the Authors have shown that moments of quasi-PDFs are divergent:
since the moments of parton distributions should reproduce the (finite) matrix
elements of the renormalized local DIS operator, they conclude that the
quasi-PDF cannot be considered as an euclidean generalization of the light-cone
PDF. The problem has been addressed in several independent papers, see e.g.
Refs.~\cite{Ji:2017rah, Radyushkin:2018nbf, Karpie:2018zaz}. In this appendix we
revise these criticisms in the framework of the scalar model: first we show how
the points raised in Ref.~\cite{Rossi:2017muf, Rossi:2018zkn} can be easily seen
and understood within the toy model presented in this paper, showing explicitly
how all the moments of quasi-PDFs are indeed divergent; second we discuss how
such feature does not invalidate the programme presented in
Sec.~\ref{sec:conclusions_scalar}, based on the determination of a parametric form of
the light-cone PDF based on a discrete set of data for the euclidean matrix
element.

We start this section by computing the moments of the quasi-PDF. From
Eq.~\eqref{eq::1loopcont}, using the integral representation of the Bessel
function, the $\mathcal{O}\left(\alpha\right)$ contribution to the euclidean
matrix element reads
 \begin{align}
    \hat{\mathcal{M}}^{(1)}\left(\nu, -z_3^2\right) = 
    \alpha\int_0^{1} d\xi \, \left(1-\xi\right) \int_0^{\infty}\frac{dT}{T} e^{-T} e^{-\frac{z_3^2 M^2}{4T}} e^{-i\xi P_3 z_3}\, .
\end{align}
The corresponding contribution to the quasi-PDF is found by taking the Fourier
transform of the expression above:
\begin{align}
    \hat{q}^{(1)}\left(y\right) &= 
    \frac{P_3}{2\pi}\int_{-\infty}^{\infty}d z_3\, e^{-i y P_3 z_3 } \hat{\mathcal{M}}^{(1)}\left(\nu, -z_3^2\right) \\
    & = \alpha\,\frac{P_3}{\sqrt{\pi}}\int_0^1 d\xi \, \left(1-\xi\right)\frac{1}{M} 
	\int_0^{\infty} \frac{dT}{\sqrt{T}}e^{-T} e^{-T\left(y+\xi\right)^2 \frac{P_3^2}{M^2}}\, ,
\end{align}
where in the last line we have computed the gaussian integral over $z_3$.
Taking the $n$-th moment of $\hat{q}^{(1)}\left(y\right)$ yields
\begin{align}
    \int_{-\infty}^{\infty} dy\, y^n \hat{q}^{(1)}\left(y\right) =
	\alpha\,\frac{P_3}{\sqrt{\pi}}\int_0^1 d\xi \, \left(1-\xi\right)
	\frac{1}{M} \int_0^{\infty} \frac{dT}{\sqrt{T}}e^{-T}
	\int_{-\infty}^{\infty} dy\, 
	\left(y-\xi\right)^n e^{-T y^2 \frac{P_3^2}{M^2}}\, .
\end{align}
We can expand the polynomial term as
\begin{align}
	\left(y-\xi\right)^n = \sum_{k=0}^{n}\binom{k}{n}y^{n-k}\xi^k\
\end{align}
and evaluate each contribution in turn. The term with $k=n$, performing the
integral over $y$ first, yields
\begin{align}
	 \alpha\,\frac{P_3}{\sqrt{\pi}}&\int_0^1 d\xi\, 
	 \left(1-\xi\right) \xi^n \frac{1}{M}
	 \int_0^{\infty} \frac{dT}{\sqrt{T}}\, e^{-T } 
	 \int_{-\infty}^{\infty} dy\, e^{-T y^2 \frac{P_3^2}{M^2}}  \\
	 & = \alpha \int_0^1 d\xi\, 
	 \left(1-\xi\right) \xi^n \int_0^{\infty} \frac{dT}{T}e^{-T}\, . 
\end{align}
The integral over $T$ is divergent, with the divergence originating from the
lower end of the integration region, i.e. when $T\to 0$. Introducing a cutoff
$a^2$ for small values of $T$~\footnote{The cutoff $a$ has dimensions of length
and can be thought of as a lattice spacing if the theory were regulated on a
lattice.} and considering the limit $a^2 \rightarrow 0$, we get the logarithmic
divergent contribution
\begin{align}
\label{eq::logdiv}
	\alpha\int_0^1 d\xi\, 
	\left(1-\xi\right) \xi^n \int_{a^2}^{\infty} \frac{dT}{T}e^{-T}\,\,
	\stackrel{a^2\rightarrow 0}{\sim}\,\, 
	- \alpha\int_0^1 d\xi\, \left(1-\xi\right) \xi^n \log a^2\, .
\end{align}
Similarly we can consider contributions coming from even values of $n-k$. Using 
\begin{align}
	\int_{-\infty}^{\infty}
	&dy\, y^{2m} e^{-T y^2 \frac{P_3^2}{M^2}} = 
		\frac{M}{P_3} \left(-\frac{M^2}{P_3^2}\frac{d}{dT}\right)^m 
		\int_{-\infty}^{\infty} dy\,e^{-T y^2} \nonumber \\
	&= \frac{M\sqrt{\pi}}{P_3} 
	\left(-\frac{M^2}{P_3^2}\frac{d}{dT}\right)^m \frac{1}{\sqrt{T}}  
	\propto \frac{M \sqrt{\pi}}{P_3} \frac{1}{T^{m + \frac{1}{2}}}\, ,
\end{align}
and considering $n-k = 2m$, we get
\begin{align}
	\label{eq::powerdiv}
	\alpha\, \frac{P_3}{\sqrt{\pi}}
	&\int_0^1 d\xi\, \left(1-\xi\right) \xi^{n-2m} \frac{1}{M}
		\int_0^{\infty} \frac{dT}{\sqrt{T}}\, e^{-T } 
		\int_{-\infty}^{\infty} dy\, y^{2m} e^{-T y^2 \frac{P_3^2}{M^2}} \nonumber \\
	&\propto \alpha\, \int_0^1 d\xi\, \left(1-\xi\right) \xi^{n-2m}
		\int_{a^2}^{\infty} \frac{dT}{T^{m +1}}\, e^{-T} \,\, \nonumber \\
		&\stackrel{a^2\rightarrow 0}{\sim}\,\, 
		\alpha\, \int_0^1 d\xi\, \left(1-\xi\right) \xi^{n-2m}
		\frac{1}{m} \left(\frac{1}{a^2} \right)^{m}\, ,
\end{align}
where again we have introduced a cutoff $a^2$ for small values of $T$ and
considered the limit $a^2 \rightarrow 0$. Contributions from odd values of $n-k$
vanish. Looking at Eqs.~\eqref{eq::logdiv}, \eqref{eq::powerdiv} it is then
clear that all the moments of the quasi-PDFs will be at least logarithmically
divergent with the cutoff $a^2$, with higher moments affected by higher power
divergences.

This relatively simple calculation shows that we obtain divergent contributions
for the moments of the quasi-PDF and therefore quasi-PDFs cannot be considered
as the proper euclidean generalization of the light-cone parton distribution.
This, however, does not invalidate the approach described in
Sec.~\ref{sec:conclusions_scalar}: as mentioned, what really matters is the existence
of a factorization theorem connecting the collinear PDF with a renormalizable
quantity that can be computed on the lattice, which in our case will be the
euclidean matrix element of Eq.~\eqref{eq::pIoffe1loopren}, computed for fixed
values of $P_3$ and $z_3$. As long as $z_3$ is kept small and different from
$0$, the factorization formula \eqref{eq::fact2} holds, and can be used to fit
the light-cone PDF using the available lattice data. How well such data can
constrain the PDF is something which should be investigated, just as in the same
way the constraints from new experimental measurements are usually analyzed.