\chapter{PDFs from quasi-PDFs matrix elements: \\
matching coefficient and lattice convolution}
\label{app:coefficients}
%%%%%%%%%%%%%%%%%%%%%%%%%%%%%%%%%%%%%%%%%%%%%%%%%%%%%%%%%%%%%%%%
As detailed at the end of sec.~\ref{subsec:latticedata}, the matching coefficients to be
used to relate the data of refs.~\cite{Alexandrou:2018pbm,Alexandrou:2019lfo} to the light-cone PDFs
are those expressed in the $\MMSb$ scheme. Their explicit expression is given
by \cite{Alexandrou:2018pbm,Alexandrou:2019lfo}
\begin{equation}
	\label{eq::matching}
	\begin{split}
		&C_{3}\left(\xi,\eta\left(\xi\right) \right)= \delta\left(1-\xi\right)\, +  C_{3}^{\text{NLO}}\left(\xi,\eta\left(\xi\right) \right)_+\,,  \\ \\ 
		& C_{3}^{\text{NLO}}\left(\xi,\eta\left(\xi\right) \right)_+ =  \frac{\alpha_s}{2\pi}C_F \begin{cases} \left[\frac{1+\xi^2}{1-\xi}\log{\frac{\xi}{\xi-1}} + 1 + \frac{3}{2\xi}\right]^{\left[1,+\infty\right]}_{+(1)} \,\,\,\,\,\,\,\,\,\,\,\,\,\,\,\,\,\,\,\,\,\,\,\,\,\,\,\,\,\,\,\,\,\,\,\,\,\,\,\,\,\,\,\,\,\,\,\,\,\,\,\,\,\,\,\,\,\,\,\, \xi > 1\\ \left[\frac{1+\xi^2}{1-\xi}\log\left[{\frac{1}{\eta^2\left(\xi\right)}}\left(4\xi\left(1-\xi\right)\right)\right]  -\frac{\xi\left(1+\xi\right)}{1-\xi}\right]^{\left[0,1\right]}_{+(1)}\,\,\,\,\,\,\,\,\,\,\,\,\,\,\,\,\,\,\, 0<\xi < 1 \\ \left[-\frac{1+\xi^2}{1-\xi}\log{\frac{\xi}{\xi-1}} - 1 + \frac{3}{2\left(1-\xi\right)}\right]^{\left[-\infty,0\right]}_{+(1)} \,\,\,\,\,\,\,\,\,\,\,\,\,\,\,\,\,\,\,\,\,\,\,\,\,\,\,\,\,\,\,\,\,\,\,\,\,\,\,\,\,\,\,\,\,\, \xi<0\end{cases}\, .
	\end{split}
\end{equation}
where the superscripts indicate the domain over which the plus prescription acts.
The matching coefficients relate the light-cone PDF to the quasi-PDF up to power suppressed terms
according to
\begin{align}
	\label{eq::matchingfud}                                   
	\tilde{f}_{3}\lp x P_z, {\mu}^2 \rp =              
	\int_{-1}^{1} \frac{dy}{|y|}\, C_{3}\lp\frac{x}{y},\frac{\mu}{y P_z}\rp  
	f_{3}\lp y, {\mu}^2\rp. 
\end{align}
In the following, we work out the full expression of the coefficients appearing in eqs.~\eqref{eq::V3factorization}, \eqref{eq::T3factorization}.
Starting from eq.~\eqref{eq::matchingfud} we have
\begin{align}
	\label{eq::explicitmatching}
	\tilde{f}_3\left(x,\mu^2,P_z\right) = \int_{-1}^{1} \frac{dy}{|y|}\, \delta\lp 1- \frac{x}{y} \rp  
	f_{3}\lp y, {\mu}^2\rp + \int_{-1}^{1} \frac{dy}{|y|}\, C_{3}^{\text{NLO}}\lp\frac{x}{y},\frac{\mu}{y P_z}\rp_+  
	f_{3}\lp y, {\mu}^2\rp\, .
\end{align}
Let us focus on the next-to-leading order term, making the plus distribution
explicit. In order to do so, we find it useful to split the integral in the two
contributions for $y<0$ and $y>0$. A change of variables, $\frac{x}{y} =
\xi$, yields 
\begin{align}
\label{eq::intermediate}
	\tilde{f}_3^{\text{NLO}}\left(x,\mu^2,P_z\right) 
	  & \equiv \int_{-1}^{1} \frac{dy}{|y|}\, 
	C_{3}^{\text{NLO}}\lp\frac{x}{y},\frac{\mu}{y P_z}\rp_+ 
	f_{3}\lp y, \mu^2\rp \nonumber \\ 
	  & = \int_{|x|}^{\infty} d\xi\,          
	C_3^{\text{NLO}}\left(\xi,\frac{\mu\xi}{x P_z}\right)_+ 
	\frac{1}{|\xi|} f_3\left(\frac{x}{\xi},\mu^2\right) + \nonumber \\
	  & \quad + \int_{-\infty}^{-|x|} d\xi\,  
	C_3^{\text{NLO}}\left(\xi,\frac{\mu\xi}{xP_z}\right)_+ 
	\frac{1}{|\xi|} f_3\left(\frac{x}{\xi},\mu\right). 
\end{align}
%
The plus distribution appearing in the matching coefficients is defined in ref.~\cite{Izubuchi:2018srq} and implemented as
follows
\begin{align}
	\int_D d\xi\, 
	C\left(\xi,g\left(\xi\right)\right)_+ f\left(\xi\right) 
	= \int_D d\xi\,
	\left[C\left(\xi,g\left(\xi\right)\right)f\left(\xi\right) - 
	C\left(\xi,g\left(1\right)\right)f\left(1\right) \right]\, , 
\end{align}
with $g\lp\xi\rp =\frac{\mu\xi}{xP_z} $ and $D$ representing a generic integration domain, which in our case will be, according to eq.~\eqref{eq::intermediate},
either $\left(-\infty,-|x|\right)$ or $\left(|x|,+\infty\right)$. 
It follows
\begin{align}
	\tilde{f}_3^{\text{NLO}}&\left(x,\mu^2,P_z\right) 
	   =\int_{|x|}^{\infty} d\xi\, 
	  \left[ C_3^{\text{NLO}}\left(\xi,\frac{\mu\xi}{xP_z}\right)
	  \frac{f_3\left(\frac{x}{\xi},\mu^2\right)}{|\xi|} - 
	  C_3^{\text{NLO}}\left(\xi,\frac{\mu}{xP_z}\right) f_3\left(x,\mu^2\right)\right] \nonumber \\
	  &  
	  + \int_{-\infty}^{-|x|}d\xi
	  \left[C_3^{\text{NLO}}\left(\xi,\frac{\mu\xi}{xP_z}\right)
	  \frac{f_3\left(\frac{x}{\xi},\mu^2\right)}{|\xi|} - 
	  C_3^{\text{NLO}}\left(\xi,\frac{\mu}{xP_z}\right)f_3\left(x,\mu^2\right)\right]. 
\end{align}
It can be easily verified that the two contributions appearing in the above
equation are indeed well defined for every fixed $x$: the singularity in
$\xi=+1$ is cured by the plus prescription, while for $\xi\rightarrow \pm
\infty$ the matching coefficient behaves like $C\left(\xi\right)\sim
\frac{1}{\xi^2}$, which is enough to guarantee the convergence of all the
integrals above. For numerical stability we find it useful to avoid the
singularity in $\xi = +1$ introducing a suitable small parameter $\delta \sim
10^{-6}$, and rewriting the above equation as
\begin{equation}
	\label{eq::explicitplus}
	\begin{split}
		\tilde{f}_3^{\text{NLO}}&\left(x,\mu^2,P_z\right)
		=\int_{|x|}^{1-\delta} d\xi\,
		C_3^{\text{NLO}}\left(\xi,\frac{\mu\xi}{xP_z}\right)
		\frac{f_3\left(\frac{x}{\xi},\mu\right)}{\xi} - 
		f_3\left(x,\mu\right) \int_{|x|}^{1-\delta} d\xi\,
		C_3^{\text{NLO}}\left(\xi,\frac{\mu}{xP_z}\right) \\
		& + \int_{1+\delta}^{\infty}d\xi\,
		C_3^{\text{NLO}}\left(\xi,\frac{\mu\xi}{xP_z}\right)
		\frac{f_3\left(\frac{x}{\xi},\mu\right)}{\xi} - 
		f_3\left(x,\mu\right) \int_{1+\delta}^{\infty}d\xi\,
		C_3^{\text{NLO}}\left(\xi,\frac{\mu}{xP_z}\right) \\
		& - \int_{-\infty}^{-|x|} d\xi\, 
		C_3^{\text{NLO}}\left(\xi,\frac{\mu\xi}{xP_z}\right)
		\frac{f_3\left(\frac{x}{\xi},\mu\right)}{\xi} - 
		f_3\left(x,\mu\right) \int_{-\infty}^{-|x|}d\xi\, 
		C_3^{\text{NLO}}\left(\xi,\frac{\mu}{xP_z}\right).
	\end{split}
\end{equation}
%
In order to obtain the lattice ME, we need to compute the real and imaginary part
of the Fourier transform of eq.~\eqref{eq::explicitmatching} as shown in
eq.~\eqref{eq:factorisedME}. Starting from the leading-order contribution, we get
\begin{align}
	\label{eq::LO}
	\int_{-\infty}^{\infty}& dx\, \cos\left(x P_z z\right) \int_{-1}^{1} \frac{dy}{|y|}\, \delta\lp 1-\frac{x}{y} \rp  
	f_{3}\lp y, {\mu}^2\rp  \nonumber \\
	&= \int_0^1 dy \cos\lp y P_z z  \rp \lp f_{3}\lp y, {\mu}^2\rp + f_{3}\lp -y, {\mu}^2\rp  \rp = 
	\int_0^1 dx \cos\lp x P_z z  \rp V_3 \lp x, {\mu}^2 \rp \nonumber \\
	&= \int_0^1 dx\, \text{A}^{\text{Re}, \,\text{LO}} \lp x P_z z \rp V_3 \lp x, {\mu}^2 \rp
\end{align}
where we have integrated in $x$ first, re-expressed the integral $\int_{-1}^1 dy$ as $\int_0^1 dy$, used
\begin{align} 
	f_3\left(x\right)+f_3\left(-x\right) = f_3^{\text{sym}}\left(x\right) = u^-\left(x\right) - d^-\left(x\right) = V_3\left(x\right) 
\end{align}
and finally changed variables back to $x$.  
Moving now to the next-to-leading order part, we analyse each of the six contributions listed in
eq.~\eqref{eq::explicitplus}, defining for each lattice observable six integrals
to be computed, denoted as
$\text{I}^{\text{Re}}_i,\,\text{I}^{\text{Im}}_i\,\,\, i = 1,..,6$. Starting
from the first contribution to the real part we get
\begin{align}
	\text{I}^{\text{Re}}_1 
	  & =\int_{-\infty}^{\infty} dx\, \cos\left(x P_z z\right)
	  \int_{|x|}^{1-\delta}d\xi\, 
	  C_3^{\text{NLO}}\left(\xi,\frac{\mu\,\xi}{xP_z}\right)
	  \frac{f_3\left(\frac{x}{\xi},\mu\right)}{\xi} \nonumber \\
	  & = \int_{0}^{\infty} dx\, \cos\left(x P_z z\right)
	  \int_{x}^{1-\delta}\frac{d\xi}{\xi}\,
	  C_3^{\text{NLO}}\left(\xi,\frac{\mu\,\xi}{xP_z}\right)
	  \left(f_3\left(\frac{x}{\xi},\mu\right) + 
	  f_3\left(-\frac{x}{\xi},\mu\right) \right) \nonumber \\
	  & =\int_{0}^{1} dx\, \cos\left(x P_z z\right)
	  \int_{x/(1-\delta)}^{1}\frac{dy}{y}\,
	  C_3^{\text{NLO}}\left(\frac{x}{y},\frac{\mu}{y P_z}\right) V_3\left(y,\mu\right)\, ,
\end{align}
where in the last line we have changed variables back to $\frac{x}{\xi} = y$.
Also, the integration range for $x$ becomes $\lp 0,1 \rp$, since $x<y<1$.
Renaming variables, we have
\begin{align}
	\text{I}^{\text{Re}}_1 
	  & = \int_{0}^{1}dx\,
	  	\left[
			  \frac{1}{x}\int_{0}^{1} dy\, 
			  \Theta\lp x - \frac{y}{1-\delta} \rp
			  \cos\left(y P_z z\right) 
			  C_3^{\text{NLO}}\left(\frac{y}{x},\frac{\mu}{x P_z}\right)
		\right]
	V_3\left(x,\mu^2\right) \nonumber \\
	  & = \int_{0}^{1}dx\, 
		  \text{A}^{\text{Re},\,\text{NLO}}_1 \lp x, z, \frac{\mu}{P_z} \rp 
		  V_3\left(x,\mu^2\right)\, . 
\end{align}
Analogously, we find out that the other five contributions can be written as 
\begin{align}
	\text{I}^{\text{Re}}_{\,i} =
	\int_{0}^{1}dx\, \text{A}^{\text{Re},\,\text{NLO}}_{\,i} \lp x, z, \frac{\mu}{P_z} \rp V_3\left(x,\mu^2\right). 
\end{align}
with
\begin{align}
	& \text{A}^{\text{Re},\,\text{NLO}}_2 = \cos(x P_z z)\int_{x}^{1-\delta}d\xi \,
		C_3^{\text{NLO}}\left(\xi,\frac{\mu}{xP_z}\right), \\
	& \text{A}^{\text{Re},\,\text{NLO}}_3 = \frac{1}{x}\,\int_{0}^{\infty} dy\,
		\Theta\left(\frac{y}{1+\delta}-x\right) \cos\left(y P_z z\right)C_3^{\text{NLO}}\left(\frac{y}{x},\frac{\mu}{x P_z}\right), \\
	& \text{A}^{\text{Re},\,\text{NLO}}_4 = \cos(x P_z z)\int_{1+\delta}^{\infty}d\xi\,
		C_3^{\text{NLO}}\left(\xi,\frac{\mu}{xP_z}\right), \\
	& \text{A}^{\text{Re},\,\text{NLO}}_5 = -\frac{1}{x}\,\int_{0}^{\infty} dy\,
		\cos\left(y P_z z\right) 
		C_3^{\text{NLO}}\left(-\frac{y}{x},\frac{\mu}{x P_z}\right), \\
	& \text{A}^{\text{Re},\,\text{NLO}}_6 = \cos(x P_z z) \int_{-\infty}^{-x}d\xi \,
		C_3^{\text{NLO}}\left(\xi,\frac{\mu}{xP_z}\right)
\end{align}
Collecting all the terms yields eq.~\eqref{eq::V3factorization}
\begin{align}
	\mathcal{O}_{\gamma^0}^{\text{Re}}\lp z,\mu \rp = \int_{0}^{1} dx \, \mathcal{C}_3^{\text{Re}}\lp x, z, \frac{\mu}{P_z}\rp 
	\nsv\left(x,\mu^2\right)\, , 
\end{align}
where
\begin{align}
	\mathcal{C}_3^{\text{Re}}&\lp x, z, \frac{\mu}{P_z}  \rp = 
	\text{A}^{\text{Re},\,\text{LO}} + \text{A}^{\text{Re},\,\text{NLO}}  
\end{align}
with
\begin{align}
	\text{A}^{\text{Re},\,\text{NLO}}= 
	\text{A}^{\text{Re}, \,\text{NLO}}_1 - \text{A}^{\text{Re}, \,\text{NLO}}_2 + \text{A}^{\text{Re}, \,\text{NLO}}_3 
	- \text{A}^{\text{Re}, \,\text{NLO}}_4 - \text{A}^{\text{Re}, \,\text{NLO}}_5 - \text{A}^{\text{Re}, \,\text{NLO}}_6. 
\end{align}
%
We now turn to the imaginary part of the Fourier transform. The computation is
exactly the same as in the previous case, with the only difference that now we
have a $\sin$ instead of a $\cos$. Because of this, when re-expressing the
integral $\int_{-\infty}^{\infty} dx$ as $\int_{0	}^{\infty} dx $, we get an
additional minus sign, which gives the combination
\begin{align} 
	f\left(x\right)-f\left(-x\right) = f_3^{\text{asy}}\left(x\right) = u^+\left(x\right) - d^+\left(x\right) = T_3\left(x\right). 
\end{align} 
Therefore, the results for the imaginary part can be obtained from those for the real part simply by replacing $\cos$ with $\sin$ and $V_3$ with $T_3$.