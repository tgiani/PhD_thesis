\chapter{Fitting the b-quark PDF as a massive-b scheme}
\label{ch:bottom}

It has been recently shown~\cite{Ball:2017nwa} that for accurate
phenomenology at the LHC it is advantageous to treat the charm parton
distribution (PDF) on the same footing as light-quark PDFs, namely, to
parametrize it and extract it from data, rather than to take it
as radiatively generated from the gluon using  perturbative matching
conditions. This is likely to be
due to the fact that matching conditions are only known to
the lowest nontrivial order, which may well be subject to large higher
order
corrections, as revealed by the strong dependence of results on the
choice of matching scale. On top of this, of course,
the starting low-scale heavy quark
PDFs could in principle also have a
non-perturbative ``intrinsic''
component~\cite{Brodsky:1980pb,PhysRevD.23.2745}. It is important to note  that
whether or not the heavy quark PDF has a nonperturbative component,
and whether it is advantageous to parametrize the heavy quark PDF are
separate issues: in fact in Ref.~\cite{Ball:2017nwa} it was shown that
the main phenomenological
advantage in parametrizing and fitting the charm PDF comes from a region in
which any nonperturbative contribution to charm is likely to be
extremely small. 

The case of the bottom quark PDF is, in this respect, particularly
interesting. On the one hand, one may think that that the problem of
large higher order corrections to the matching conditions is alleviated
in this case by the larger value of the mass. However, on the other
hand, there is a more subtle consideration. Namely, there are $b$-initiated
hadron collider processes --- some of which
are especially relevant for new physics searches --- such as Higgs production in
bottom fusion, for which $b$ quark mass effects might be
non-negligible~\cite{Maltoni:2012pa,Lim:2016wjo,Bagnaschi:2018dnh}. This
suggests the use of a scheme in which the $b$ quark is
treated as a 
massive final-state parton --- hence not endowed with a
PDF. In such a scheme
the $b$-induced process necessarily
starts at a higher perturbative order than in  a scheme in which
there exists a $b$ PDF, because the $b$ production process is included in the
hard matrix element. As a consequence, the computation of the
$b$-induced process itself is more difficult and it can typically
only be performed with a lower perturbative accuracy than in a scheme
in which the $b$ quark is a massless parton.

The problem is somewhat
alleviated if the massive-scheme and massless-scheme
computations are combined, with the $b$-PDF in the massless scheme assumed to be
produced by perturbative matching conditions.  We  henceforth refer to
such a computational 
framework as ``matched-$b$''. However, in a matched-$b$ framework the
massive computation is still beset by the need to start at high
perturbative order.
As a possible way out, the use of a ``massive five flavor scheme'' has
been suggested recently~\cite{Krauss:2017wmx,Figueroa:2018chn}, in
which there is a $b$ PDF (hence five flavors), yet $b$ quark mass
effects are included (possibly, at least in part, also in parton showering).
The use of an independently parametrized $b$ quark PDF
within a framework in which massive and massless computations are
combined
offers a simpler way of dealing with the same
problem. We  refer to this as a ``parametrized $b$''
computational framework.
Such an approach has been developed for electroproduction in
Refs.~\cite{Ball:2015tna,Ball:2015dpa}, and it has been used in order
to produce PDF sets with parametrized
charm~\cite{Ball:2016neh,Ball:2017nwa}, including the recent NNPDF3.1
set. Because the only data currently used for PDF determination in which 
heavy quark mass effects have  a significant impact are deep-inelastic
scattering data close  to the charm production threshold, in these
references only electroproduction was considered and only the
parametrization of the charm was studied.

In these previous studies, an independently parametrized heavy quark PDF is
included in the so-called FONLL  method~\cite{Cacciari:1998it},
which allows for the
matching of a scheme in which the heavy quark mass is included but the
heavy quark decouples from QCD evolution equations, and a massless
scheme in which the heavy quark mass is neglected, but the heavy quark
PDF couples to perturbative evolution.
In this parametrized heavy quark version of the FONLL scheme, the heavy
quark PDF is present both in the massive and massless scheme, though
decoupled from evolution in the massive scheme; unlike in conventional
matched heavy quark computations
in which the number of PDFs is different, with one more
PDF in the massless scheme. The rationale for FONLL
with a parametrized heavy quark is to simultaneously include heavy quark
mass effects at lower scales and
the resummation of collinear mass logarithms in the heavy quark PDFs at
higher scales. This has the important byproduct that one
ends up with a computational framework in which there are heavy quarks in
the initial state even in the scheme in which mass effects are
retained.

Therefore, in a parametrized-$b$ FONLL framework, problems
related to the fact that the relevant processes in a massive scheme start
at higher order is thus completely
evaded, since the heavy quark 
PDF is always present. Mass effects are then included to finite
perturbative order, along with the resummation of mass logarithms, though
(unlike in some  ``massive five-flavor scheme'') mass corrections to
resummed perturbative evolution are not included. On the other hand,
any possible nonperturbative corrections to the $b$ PDF, including,
say, the effective value of the $b$ mass at which the matching should
happen, are then included in the PDF itself and thus extracted from
the data.

In this chapter, following Ref.~\cite{Forte:2019hjc}, we explicitly construct the parametrized-$b$ FONLL method, by
generalizing to hadronic processes the construction 
of Refs.~\cite{Ball:2015tna,Ball:2015dpa} of FONLL  with
parametrized heavy quark PDF. We specifically consider the
application to Higgs production in bottom fusion. This process has
been computed at the matched level both using the FONLL
method~\cite{Forte:2015hba,Forte:2016sja} and EFT-based
methods~\cite{Bonvini:2015pxa,Bonvini:2016fgf}, with the respective results
benchmarked in
Ref.~\cite{deFlorian:2016spz} and found to be in good agreement with
each other. All these computations were performed
in a matched-$b$ approach, in which the $b$ PDF  is absent in the
massive (four-flavor) scheme, and determined by matching condition in
the massless (five-flavor) scheme. Here we  take this process as
a prototype for the use of a parametrized-$b$  scheme
for hadron-collider processes.

First, we discuss how the counting of perturbative orders changes
in the presence of a parametrized-$b$ PDF, and redefine suitable
matched schemes based on this new counting. We  then work out the
generalization to hadronic processes of FONLL with
parametrized heavy quark PDF of Refs.~\cite{Ball:2015tna,Ball:2015dpa},
we discuss in which sense it effectively provides an alternative to the
massive five-flavor scheme, and then we work out explicit expressions for
Higgs production in bottom fusion to the matched next-to-leading order
- next-to leading log (NLO-NLL) level and NLO-NNLL level. We 
finally compare
predictions obtained within this approach with some plausible choices
of the $b$-quark PDF to those obtained in the approach of
Refs.~\cite{Forte:2015hba,Forte:2016sja}, and argue that results with
similar or better phenomenological accuracy can be obtained in a much
simpler way within this new approach.

\section{The FONLL scheme with parametrized heavy quark PDF
  in hadronic collisions}
\label{sec:FONLL-HI}

Even though we have the general goal of constructing a parametrized-$b$
FONLL scheme for hadronic processes, we  always specifically refer to Higgs
production in gluon fusion, in order to have  a concrete reference
case, and  test scenario. We recall that the FONLL
method matches two calculations of the same process
performed in two different renormalization schemes: a massive scheme
in which the heavy quark mass is retained, but the heavy quark
decouples from the running of $\alpha_s$ and from QCD evolution
equations, and a massless scheme in which the heavy quark contributes
to the running of $\alpha_s$ and QCD evolution
equations, but the heavy quark mass is neglected.
In the computation of a hard process at scale $Q^2$, in the former scheme
mass effects $O\left(\frac {m_b^2}{Q^2}\right)$
are retained, but mass logarithms $\ln\frac{Q^2}{m_b^2}$
 are only
kept to finite order in $\alpha_s$ (where $m_b$ denotes generically
the mass of the heavy quark). In the latter scheme, mass effects
are neglected but mass logarithms are resummed to all orders in
$\alpha_s$. Hence by matching the two calculations one retains
accuracy both at low scales where quark mass effects are important, and
at high scales where mass logarithms are large.

The general idea of the FONLL method is to realize that these are just
two different renormalization schemes: the massive scheme is a
decoupling scheme, and the massless scheme is a minimal subtraction
scheme. So the two calculations can be simply matched by re-expressing
both in the same renormalization scheme, and then subtracting common
contributions. In practice, this is done by expressing the massive
scheme computation in terms of the PDFs and $\alpha_s$ of the
massless scheme, and then adding to it the difference $\sigma ^{\rm
  d}$
between the massless calculation and the massless limit of the
massive one. Schematically
\begin{align}\label{eq:fonlldef}
&  \sigma^{\text{FONLL}} = \sigma^{\rm massive} +\sigma ^{\rm d}\\
&\qquad \sigma ^{\rm d}=\sigma^{\rm massless}-\sigma^{\rm massive,\,0}.\label{eq:fonlldiff}
\end{align}
This corresponds to replacing all the terms in the massless
computation which are included to finite order in $\alpha_s$ in the
massive computation with their massive counterpart.

In the simplest (original) version of  FONLL, as discussed
in Ref.~\cite{Cacciari:1998it} for $b$ production in hadronic
collisions, and in Ref.~\cite{Forte:2010ta} for deep-inelastic
scattering, in the massive scheme there is no heavy quark PDF, and the
 heavy quark can only appear as a final-state particle. In the
 massless scheme the heavy quark PDF is determined by matching
 conditions in terms of the light quarks and the gluon. These
 conditions match the massless scheme at a scale $\mu$ such that the
 heavy quark PDF only appears for scales above $\mu$. Specifically,
 at order $O(\alpha_s)$, the heavy quark PDF just vanishes
 at the scale 
 $\mu=m_b$
 and it is
 generated by perturbative evolution at higher scales, while at
 $O(\alpha^2_s)$ it has a nontrivial gluon-induced matching condition
 at all scales.

 When introducing a parametrized PDF both the massive and massless
 scheme computations change. The massless scheme changes, somewhat
 trivially,
 in that the
 heavy quark PDF, at the matching scale, instead of being given by a
 matching condition,  is freely parametrized. The massive scheme
 changes nontrivially in that there is now a heavy quark PDF also in
 this scheme, 
only it does not evolve with the scale.  The
 consequences of this were worked out in
 Refs.~\cite{Ball:2015tna,Ball:2015dpa} in the case of
 electroproduction, and we study them here for hadroproduction for the
 first time.

 \begin{figure}[tbp]
    \centering{
      \begin{minipage}{0.32\linewidth}
        \begin{tikzpicture}[line width=0.9 pt, scale=1.1]
          \draw[particle] (-1.5,1.25) -- (0,0) node[pos=0.0,below]{$b$}; ;
          \draw[antiparticle] (-1.5,-1.25) -- (0,0) node[pos=0.0,above]{$\bar{b}$};
          \draw[scalar] (0,0) -- (1.5,0) node[midway,below=0.1cm] {$H$} ;
        \end{tikzpicture}
      \end{minipage}
      \begin{minipage}{0.32\linewidth}
        \begin{tikzpicture}[line width=0.9 pt, scale=1]
          \draw[particle] (-1.5,1.0) -- (0,1.) node[pos=0.,below] {$b$};
          \draw[antiparticle] (-1.5,-1.) -- (0,-1.) node[pos=0.0,above] {$\bar{b}$};
          \draw[particle] (0,1) -- (0.,-1);
          \draw[gluon] (0.,-1) -- (1.5,-1) node[pos=.9,above=5pt] {$g$};
          \draw[scalar] (0.,1) -- (1.5,1) node[pos=.9,below] {$H$};
        \end{tikzpicture}
      \end{minipage}
      \begin{minipage}{0.32\linewidth}
        \begin{tikzpicture}[line width=0.9 pt, scale=1]
          \draw[particle] (-1.5,1.0) -- (0,1.) node[pos=0.,below] {$b$};
          \draw[gluon] (-1.5,-1.) -- (0,-1.) node[pos=0.1,above=3pt] {$g$};
          \draw[particle] (0,1) -- (0.,-1);
          \draw[particle] (0.,-1) -- (1.5,-1) node[pos=.9,above] {$b$};
          \draw[scalar] (0.,1) -- (1.5,1) node[pos=.9,below] {$H$};
        \end{tikzpicture}
      \end{minipage}
    }
    \caption{Feynman diagrams for the leading (left) and next-to-leading order real emission
      contributions to  Higgs production in bottom fusion which are present
      in the massive scheme when the $b$ quark PDF is independently
      parametrized, but absent otherwise.}
    \label{fig:massive-4fs}
  \end{figure}