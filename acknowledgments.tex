\chapter*{Acknowledgments}
%
First of all, I would like to thank my supervisor, Luigi Del Debbio.
I am deeply happy and grateful for the work done together.
It has been stimulating, inspiring, exciting and extremely fun, and it has allowed me to grow up
quite a lot.
I truly consider our long blackboard discussions the most valuable thing of the whole PhD experience, and
I am looking forward for more of that in the coming years.
%
I am grateful to Stefano Forte, for the guide and help he has provided me, for the work done together 
and for always giving a complete answer to any question, usually in less than three minutes.
I would like to thank all the members of the particle theory group in Edinburgh, for the 
great and stimulating environment they have created, in particular Einan Gardi, Roman Zwicky and Richard Ball. 
%
I thank all the members of the NNPDF collaboration, with whom I have been working a lot during the last
few years. In particular I am deeply in debts with Emanuele R. Nocera, who has helped me a countless amount of 
times in a number of different tasks and projects, providing me with an example of efficiency, hard work and human kindness which
is difficult to match. I am grateful to Juan Rojo, for the support and the opportunity he gave me to work in his group
at the end of my PhD.
%
A special thank goes to Rabah, Cameron, Rosalyn, Michael, Zahari, Stefano, Shayan, Emma
and Luca. I have learnt a lot from each of them, in terms of physics, coding and teamwork, and it has been a real pleasure
to regularly meet them in the different meetings we had during the years.
In particular I want to thank Rosalyn and Michael for having shared with me the experience of being PhD students in Edinburgh.
It has been great for me to regularly see and work with them and I am grateful for all the trips
we had together around the world. 
I am also very proud of the progresses Michael has done in learning the basics of the italian language.
%
I want to thank Simone Marzani for the help and support he gave me, Marco Bonvini and Rhorry Gauld, for 
useful and stimulating discussions about resummation.  
%
I am grateful to Nathan P. Hartland for the many questions he answered, to
Davide Napoletano for the work done together and for the important help he provided me during the very first days of my PhD.
I want to thank Guido Cossu and Ava Khamseh for the work done together during my first year
and Anatoly Radyushkin for illuminating discussions, which allowed me to better understand the relation 
between light-cone and euclidean quantities. Finally, I would like to thank the master students who, when 
in person teaching was only partially allowed, regularly came to the QFT tutorials, watching me missing
minus signs and factors two at the blackboard.
Their interest, enthusiasm and questions have been of great support for me, and I have learnt a lot from them. 

%%
A number of not-physics friends have played a central role for me during my time in Edinburgh,
making me feel home every single day. It wouldn't have been possible for me to do any work without them.
% 
A huge thank you to Stefano, for a number of things which wouldn't fit a page. For the pizza, pasta al pesto,
creative swearwords and one-arm pull up sessions among other things. 
But, most of all, for being the best possible flatmate on earth.
I am pretty sure that without him I would have already died in some stupid way. 
%
I thank Lauren, for having helped me in a number of different ways, for her kindness, her support and for always being there,
during both the good and not-so-good times.  
I thank Roxane for the great time together, for the cycling, for having shown me parts of Scotland 
I had not seen before and for her help and support in looking for a job. 
%
A special thank you to my main British climbing buddies Ed and Patrick with whom I spent countless days and nights
pulling on tiny nasty crimps, dishonest slopers and silly pinches, discussing quality climbing videos.
And of course huge thanks to the whole Garage Crimpers team,
Scott, Colin and Sinclair. I am grateful for the countless sessions on the A2, on the 45.8 in the power garage 
and of course in the County, which, together with Dumby, has now a very special place among the longlist of my
favourite climbing spots. 
%
Thanks to Andrea, for having being close during a strange time, when the pandemic started.
Finally, I would like to thank Michael, who offered me a room during my very first days in Scotland, without even knowing 
who I was.

%   
I want to thank a number of friends from Italy, who have constantly supported and helped me in different ways.
Huge thanks to Gi, for having shared with me the first years of this experience and for having always being there for me,
despite everything.
Thanks to my serious dottorandi friends Fabbri, Tommy, Mario and Anna who in one way or another are always there.
I am always looking forward for seeing them every time we are all back to Italy (especially when this happens at Silvi).
Thanks to Alice, for the help and support provided during the last months of my PhD. 
Thanks to Rick, for the chats we have every time we see each other.
Huge thanks to Lollo, for being a beast, for all the climbing we are always having together every time
I am back to Italy and for the support he is always giving me.
Thanks to Ale, for the countless routes tried together and for always taking every single whipper.
Thanks to my Passaggio Obbligato friends, in particular to Fede Montagna and Teo Nill, who 
regularly sent me supportive messages. 
Finally thanks to Luca, who in his own way has always been there for me.

Grazie ai miei fratelli, ai miei nonni e soprattutto ai miei genitori per l'aiuto e l'affetto incondizionato
che ricevo ogni singolo giorno. 
Infine grazie Sar per essere stato, a tuo modo, di guida e sostegno. Vivi in ogni piccolo traguardo quotidiano.

\blankpage
 



